 %  DOCUMENT CLASS
\documentclass[11pt]{scrartcl}

%PACKAGES
\usepackage[utf8]{inputenc}
\usepackage[ngerman]{babel}
\usepackage[reqno,fleqn]{amsmath}
\setlength\mathindent{10mm}
\usepackage{amssymb}
\usepackage[standard]{ntheorem}
\usepackage{amsfonts}
\usepackage{bbm}
\usepackage{units}
\usepackage{times, eurosym}
\usepackage{lmodern}

% FORMATIERUNG
\usepackage[paper=a4paper,left=25mm,right=25mm,top=25mm,bottom=25mm]{geometry}
\setlength{\parindent}{0cm}
\setlength{\parskip}{1mm plus1mm minus1mm}

%MATH SHORTCUTS
\newcommand*{\T}{\mathrm T}
\renewcommand{\d}{\,\mathrm d}
\newcommand*{\NN}{\mathbb N}
\newcommand*{\RR}{\mathbb R}
\newcommand*{\PP}{\mathbb P}
\newcommand*{\one}{\mathbbm 1}
\newcommand*{\eqx}{\mathbin{\overset{!}{=}}}
\newcommand*{\Gl}{\mathrm{Gl}}
\newcommand{\e}[1]{\cdot 10^{#1}}
\newcommand*{\Loes}{\mathrm{L\ddot{o}s}}
\DeclareMathOperator*{\sgn}{\mathrm{sgn}}
\DeclareMathOperator*{\rang}{\mathrm{rang}}
\DeclareMathOperator*{\diag}{\mathrm{diag}}
\DeclareMathOperator*{\Ker}{\mathrm{Ker}}
\DeclareMathOperator*{\Eig}{\mathrm{Eig}}
\newcommand{\cond}{\mathrm{cond}_{\| \cdot \|}}
\newcommand{\conda}{\mathrm{cond}_{\| \cdot \|}_1}
\newcommand{\condb}{\mathrm{cond}_{\| \cdot \|}_2}
\newcommand{\condi}{\mathrm{cond}_{\| \cdot \|}_\infty}


\theoremstyle{break}
\theorembodyfont{\normalfont}
\renewtheorem {Beispiel}{Beispiel}
\theoremstyle{remark}
\newtheorem{beh}{Behauptung}


\title{Numerische Mathematik}
\subtitle{Notizen zur Vorlesung von Prof. Dr. Timo Reis, WiSe 2014/15}
\author{Laura Schüttpelz, Damian Hofmann}
\date{\today}

\begin{document}
\maketitle

\section{Einleitung}
Was ist \emph{numerische Mathematik}? Bereitstellung konstruktiver Verfahren zur approximativen Berechnung mathematischer Sachverhalte.

Konstruktiv: Am Computer.

Mathematischer Sachverhalt: Zum Beispiel das Lösen linearer Gleichungssysteme.

\begin{Beispiel}[Lineares Gleichungssystem, LGS]

Gegeben: $A \in \RR^{n\times n} \text{ invertierbar } \Leftrightarrow A \in \Gl_n(\RR)$,
$b \in \RR^n$.
Gesucht: $x \in \RR^n$, so dass $Ax = b \Leftrightarrow x = A^{-1} b$ gilt.

Naive Methode: Lösung über Cramersche Regel.

$$x = \begin{pmatrix}x_1 \\ \vdots \\ x_n \end{pmatrix}, \qquad x_i = \frac{\det(A_i)}{\det(A)}$$
mit $$A_i :=
\begin{pmatrix}
  a_{11} & \cdots & a_{1,i-1} & b_1 & a_{1,i+1} & \cdots & a_{1n} \\
  \vdots & \ddots & \vdots & \vdots & \vdots & \ddots & \vdots \\
  a_{n1} & \cdots & a_{n,i-1} & b_n & a_{n.i+1} & \cdots & a_{nn}
\end{pmatrix}.$$

In der Leibnitz-Darstellung ist $$\det(A) = \sum_{\sigma\in S_n} \sgn(\sigma)\, a_{1\sigma(1)} \cdots a_{n\sigma(n)}.$$

Fakten:
\begin{enumerate}
\item Die Leibnitz-Darstellung von $\det(A)$ benötigt $n! (n-1)$ Multiplikationen.
\item Aus 1. folgt: Die Lösung von $Ax=b$ mit der Cramerschen Regel benötigt $2n!(n-1)n + n$
      Multiplikationen.
\end{enumerate}

Eine Intel-Core-i7-CPU berechnet etwa $3.3\e{10}$ Multiplikationen pro Sekunde.
Für ein System mit 20 Unbekannten benötigt die Cramersche Regel daher
$ \frac{20! \cdot 2 \cdot 19 \cdot +20}{3.3 \cdot 10^{10}
\cdot 60 \cdot 60 \cdot 24 \cdot 365} = 1776.3$ Jahre.
Andere Fälle:
\begin{align*}
n = 15\colon & 4.6\text{ Std}\\
n = 10\colon & 0.2\text{ Sek.}\\
n = 25\colon & 17.8\text{ Mrd. Jahre}
\end{align*}

Mithilfe "`geschickterer"' Verfahren lassen sich heute LGS mit heutzutage bis zu $10^9$
Unbekannten in vernüftiger Zeit lösen.
\end{Beispiel}

Eine vage Definition:

\begin{Definition}[Aufwand]
Sei ein numerisches Verfahren abhängig von den Größen $n_1, \ldots, n_l \in \NN$.
Man spricht vom
\emph{Aufwand von $O(f(n_1, \dots, n_l))$ Multiplikationen}
(für $f\colon \mathbb{N}^l \to \mathbb{N}$) wenn es
$c_1, c_2 > 0$ gibt, so dass das Verfahren für alle
$n_1, \dots, n_l > c_1$ mit maximal $c_2 \cdot f(n_1, \dots n_l)$
Multiplikationen auskommt (Divisionen zählen als Multiplikationen,
Additionen zählen gar nicht).
\end{Definition}

\begin{Beispiel}[Zahl der Multiplikationen in verschiedenen Rechnungen]
\begin{enumerate}
\item[(a)]
Cramersche Regel zur Lösung von $Ax = b$ mit $A \in \Gl_n(\RR),
b \in \mathbb{R}^n$: \\
$O(n! \cdot (n-1)\cdot n)$ Multiplikationen.
\item[(b)]
Euklidisches Skalarprodukt:
$$\langle x,y \rangle = \sum_{k=1}^n x_k y_k \qquad x,y \in \mathbb{R}^n:$$ \\
$O(n)$ Multiplikationen
\item[(c)]
Matrixmultiplikation: $A \cdot B \text{ für } A\in
\mathbb{R}^{n \times m}$, $B \in \mathbb{R}^{m \times k}$: \\
$O(n \cdot m \cdot k ) $ Multiplikationen
\end{enumerate}
\end{Beispiel}

Weitere Fragestellungen der Numerik:
\begin{enumerate}
\item[(a)]
Numerische Integration: Berechne $\int_a^b f(x) \d x$ numerisch
\item[(b)]
Lösung nichtlinearer Gleichungssysteme: \\
$F(x) = 0$ für gegebenes $F\colon \RR^n \supset U \rightarrow \mathbb{R}^k$. \\
Beispiel:
\begin{enumerate}
\item[(i)]
$F(x)=e^x-y = 0,\; y \in \mathbb{R}$ gegeben. \\
$\rightarrow$ geschlossene Lösung $x = \log y.$
\item[(ii)]
$F(x)=x \cdot e^x -1 = 0$ ist nicht geschlossen lösbar. \\
Eine Lösung existiert jedoch in $(0,1)$, da $f(0) = -1 < 0$, \\
$f(1)= e-1>0 $ (Rest folgt aus Zwischenwertsatz).
\end{enumerate}
\item[(c)]
Interpolation: \\
Gegeben: $(x_1,y_1), \dots, (x_n,y_n) \in \mathbb{R}^2$. \\
Finde $f: \RR \supset I \to \RR$, so dass
$f(x_1) = y_1, \dots , f(x_n)=y_n$.
\item[(d)]
Numerische Berechnung von Eigenwerten
\item[(e)]
Numerische Lösung gewöhnlicher Differentialgleichungen $x(t)=f(t,x(t))$).
\item[(f)]
Lineare Optimierung.
\end{enumerate}

\newpage
\section{Lineare Gleichungssysteme}

\subsection{Grundlagen}

\subsubsection{Notation}

$\mathbb{K} \in \{\mathbb{R}, \mathbb{C}\}$ \\

\begin{align*}
x &= \begin{pmatrix} x_1 \\ \vdots \\ x_n \end{pmatrix}
\in \mathbb{K}^n \widehat{=} \mathbb{K}^{n \times 1},
~ A=(a_{ij}) \in \mathbb{K}^{m \times n}  \\
x^\T &= (x_1, \dots, x_n) \in \mathbb{K}^{1 \times n},\;
A^\T = (a_{ji}) \in \mathbb{K}^{n \times m}  \\
x^{\ast} &= (\bar{x}_1, \dots, \bar{x}_n),
~A^{\ast} = (\bar{a}_{ji}) \in \mathbb{K}^{n \times m}  \\
\Gl_n(\mathbb K) &= \{A \in \mathbb{K}^{n \times n}: A
\text{ invertiertbar} \} & \text{"`General Linear Group"'}  \\
O_n (\mathbb{K})&= \{ U \subset \mathbb{K}^{n \times n}:
U^{\ast} U = I \} & \text{"`Orthogonale Gruppe"'} \\
& I \text{ ist die Einheitsmatrix.} \\
e_i &= \begin{pmatrix} 0 \\ \vdots \\ 1 \\ \vdots \\ 0 \end{pmatrix}
\text{:  $i$-ter kanonischer Einheitsvektor } \\
\end{align*}

\subsubsection{Grundlagen der linearen Algebra}

Lineares Gleichungssystem (LGS): $Ax=b$

gegeben: $A \in \mathbb{K}^{m \times n}$, $b \in \mathbb{K}^m$

gesucht: Lösungsmenge $\Loes(A,b) := \{x \in \mathbb{K}^n : Ax=b\} $


Def.: $r := \rang(A),\quad s := \rang(A,b)$
\begin{align*}
\text{Fälle: } b =0 : r = n &\Rightarrow \Loes(A,b)=\Ker(A)=\{0\}\\
r<n &\Rightarrow \Loes(A,b) \text{ ist $(n-r)$-dimensionaler Unterraum
von } \mathbb{K}^n \\
b \neq 0: r<s &\Rightarrow \Loes(A,b) = \emptyset \\
r=s=n &\Rightarrow \Loes(A,b) \text{ einelementig }\\
r=s<n &\Rightarrow \Loes(A,b) = x + \Ker(A)
\text{ (mit $n-r$ Parametern), $(n-r)$-dim. } \\
\text{Speziell: } n=m: b=0 &\Rightarrow r=n: \Loes(A,b) = \{0\}\\
&\Rightarrow r<n: \Loes(A,b) = \Ker(A)\\
b \neq 0: r=n &\Rightarrow \Loes(A,b) = \{A^{-1}b\}\\
	r<n, r<s &\Rightarrow \Loes(A,b) = \emptyset \\
	 r=s &\Rightarrow \text{ Lösung mit $n-r$ Parametern } \\
\end{align*}

Bekannte Lösungsmethode: Gaußsches Eliminationsverfahren

$Ax=b \rightsquigarrow \tilde{A}x = \tilde{b} \\$

\(
(\tilde{A},\tilde{b})=
\begin{pmatrix}
\tilde{a_{11}} & &\dots  &  & \tilde{a_{1n}} & \tilde{b_1} \\
0 & & \ddots &  & \vdots & \vdots \\
0 & \dots & a_{rr} & \dots & a_{rn} & \tilde{b_r} \\
0 & & \dots & & 0 & \tilde{b_n} \end{pmatrix}
\)
mit $\tilde{a_{11}}, \dots, a_{rr} \in \mathbb{K} \backslash \{0\} $

Es gilt $\rang(A)= \rang(A,b) \Leftrightarrow
\tilde{b_{r+1}} = \dots = \tilde{b_n} = 0$

\begin{enumerate}
\item[(i)]
\[
\lambda \in \mathbb{C}\text{ heißt "`Eigenwert von } A \in
\mathbb{K}^{n \times n} \text{ "'.} \]
\[
: \Leftrightarrow \exists x \in \mathbb{C}^n \backslash \{0\}: Ax = \lambda x \\
\]\[
\Leftrightarrow \Ker (\lambda I -A) \neq \{0\} \Leftrightarrow
\det (\lambda I -A) = 0, \Ker (\lambda I-A) = \Eig(A, \lambda)
\]\[
A \in \mathbb{K}^{n \times n} \text{ besitzt Eigenwerte (Korollar des
Fundamentalsatzes der Algebra)}
\]\[
\sigma (A) = \{ \lambda \ in~ \mathbb{C}: \lambda \text{ ist Eigenwert
von A } \} \text{ "`Spektrum von A"' }
\]
\item[(ii)]
\[
A \in \mathbb{K}^{n \times n} \text{ Hermitesch (dh. } A = A^\ast \text{) } a_{ij} = \bar{a_{ji}}
\]
Dann gilt
\begin{enumerate}
\item[(a)]
$ \sigma(A) \in \mathbb{R}$
\item[(b)]
Es gibt eine Orthonormalbasis (ONB) aus Eigenvektoren von A, also  $u_1, \dots, u_n \in \mathbb{K}^n$ mit $u_i^{*} u_j = \delta_{ij}$
und $Au_i = \lambda_i u_i$ für ein $\lambda_i \in \sigma(A)$
($\forall i \in \{1, \dots, n\}$).
\end{enumerate}
$\Leftrightarrow$ Für $U = (u_1, \dots, u_n) \in \mathbb{K}^{n\times n}$ gilt $U^\ast U = I$,
d.h. $U \in O_n(\mathbb{K})$, und
$U^\ast A U = \diag(\lambda_1, \dots, \lambda_n).$
\item[(iii)]
$P \in \mathbb{R}^{n \times n}$ heißt "`Permutationsmatrix"',
falls $P=(e_{\sigma(1)}, \dots, e_{\sigma(n)})$ für ein $\sigma \in S_n$.
Dann gilt:
\begin{enumerate}
  \item[(a)] Für $A = (a_1, \dots, a_n) \in \mathbb{R}^{m \times n}$ ist
  $AP = (a_{\sigma(1)},\dots, a_{\sigma(n)}).$
  \item[(b)]
  Für $A = \begin{pmatrix} a_1 \\ \vdots \\ a_m \end{pmatrix}$,
      $P = (e_{\sigma(1)}, \dots, e_{\sigma(m)})$
  ist $P^\T A = \begin{pmatrix} a_{\sigma(1)} \\ \vdots \\ a_{\sigma(n)} \end{pmatrix}.$
\end{enumerate}
\end{enumerate}

\end{document}
