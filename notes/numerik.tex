 %  DOCUMENT CLASS
\documentclass[11pt]{scrartcl}

%PACKAGES
\usepackage[utf8]{inputenc}
\usepackage[ngerman]{babel}
\usepackage[reqno,fleqn]{amsmath}
\setlength\mathindent{10mm}
\usepackage{amssymb}
\usepackage{amsthm}
\usepackage{amsfonts}
\usepackage{bbm}
\usepackage{units}
\usepackage{times, eurosym}
\usepackage{lmodern}

% FORMATIERUNG
\usepackage[paper=a4paper,left=25mm,right=25mm,top=25mm,bottom=25mm]{geometry}
\setlength{\parindent}{0cm}
\setlength{\parskip}{1mm plus1mm minus1mm}

%MATH SHORTCUTS
\newcommand*{\T}{\mathrm T}
\renewcommand{\d}{\,\mathrm d}
\newcommand*{\NN}{\mathbb N}
\newcommand*{\RR}{\mathbb R}
\newcommand*{\PP}{\mathbb P}
\newcommand*{\one}{\mathbbm 1}
\newcommand*{\eqx}{\mathbin{\overset{!}{=}}}
\newcommand*{\Gl}{\mathrm{Gl}}
\newcommand{\e}[1]{\cdot 10^{#1}}
\DeclareMathOperator*{\sign}{\mathrm{sign}}

\theoremstyle{definition}
\newtheorem{definition}{Definition}
\newtheorem{example}{Beispiel}
\theoremstyle{remark}
\newtheorem{beh}{Behauptung}


\title{Numerische Mathematik}
\subtitle{Notizen zur Vorlesung von Prof. Dr. Timo Reis, WiSe 2014/15}
\author{Damian Hofmann\footnote{\texttt{2hofmann@informatik.uni-hamburg.de}}}
\date{\today}

\begin{document}
\maketitle

\section{Einleitung}
Was ist \emph{numerische Mathematik}? Bereitstellung konstruktiver Verfahren zur approximativen Berechnung mathematischer Sachverhalte.

Konstruktiv: Am Computer.

Mathematischer Sachverhalt: Zum Beispiel das Lösen linearer Gleichungssysteme.

\begin{example}[Lineares Gleichungssystem, LGS]

Gegeben: $A \in \RR^{n\times n} \text{ invertierbar } \Leftrightarrow A \in \Gl_n(\RR)$,
$b \in \RR^n$.
Gesucht: $x \in \RR^n$, so dass $Ax = b \Leftrightarrow x = A^{-1} b$ gilt.

Naive Methode: Lösung über Cramersche Regel.

$$x = \begin{pmatrix}x_1 \\ \vdots \\ x_n \end{pmatrix}, \qquad x_i = \frac{\det(A_i)}{\det(A)}$$
mit $$A_i :=
\begin{pmatrix}
  a_{11} & \cdots & a_{1,i-1} & b_1 & a_{1,i+1} & \cdots & a_{1n} \\
  \vdots & & & \ddots & & & \vdots \\
  a_{n1} & \cdots & a_{n,i-1} & b_n & a_{n.i+1} & \cdots & a_{nn}
\end{pmatrix}.$$

In der Leibnitz-Darstellung ist $$\det(A) = \sum_{\sigma\in S_n} \sign(\sigma) a_{1\sigma(1)} \cdots a_{n\sigma(n)}.$$

Fakten:
\begin{enumerate}
\item Die Leibnitz-Darstellung von $\det(A)$ benötigt $n! (n-1)$ Multiplikationen.
\item Aus 1. folgt: Die Lösung von $Ax=b$ mit der Cramerschen Regel benötigt $2n!(n-1)n + n$
      Multiplikationen.
\end{enumerate}

Eine Intel-Core-i7-CPU berechnet etwa $3.3\e{10}$ Multiplikationen pro Sekunde.
Für ein System mit 20 Unbekannten benötigt die Cramersche Regel daher 1776.3 Jahre.

Mithilfe geschickterer Verfahren lassen sich heute LGS mit heutzutage bis zu $10^9$
Unbekannten in vernüftiger Zeit lösen.
\end{example}

\begin{definition}
Sei ein numerisches Verfahren abhängig von den Größen $n_1, \ldots, n_l \in \NN$.
\end{definition}

\end{document}