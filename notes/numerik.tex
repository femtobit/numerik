 %  DOCUMENT CLASS
\documentclass[11pt]{scrartcl}

%PACKAGES
\usepackage[utf8]{inputenc}
\usepackage[ngerman]{babel}
\usepackage[reqno,fleqn]{amsmath}
\setlength\mathindent{10mm}
\usepackage{amssymb}
\usepackage{amsthm}
\usepackage{thmtools}
\usepackage{amsfonts}
\usepackage{bbm}
\usepackage{units}
\usepackage{times, eurosym}
\usepackage{lmodern}
%Querverweise innerhalb der Datei und des Inhaltsverzeichnisses. Hyperlinks!
\usepackage[pagebackref=true,linktoc=all,bookmarks=true,bookmarksopen=true,pdfstartview={FitV}]{hyperref}

% FORMATIERUNG
\usepackage[paper=a4paper,left=25mm,right=25mm,top=25mm,bottom=25mm]{geometry}
\setlength{\parindent}{0cm}
\setlength{\parskip}{1mm plus1mm minus1mm}

%MATH SHORTCUTS
\newcommand*{\T}{\mathrm T}
\renewcommand{\d}{\,\mathrm d}
\newcommand*{\NN}{\mathbb N}
\newcommand*{\RR}{\mathbb R}
\newcommand*{\CC}{\mathbb C}
\newcommand*{\KK}{\mathbb K}
\newcommand*{\U}{\mathbb O}
\newcommand*{\PP}{\mathbb P}

\newcommand*{\one}{\mathbbm 1}
\newcommand*{\eqx}{\mathbin{\overset{!}{=}}}
\newcommand*{\Gl}{\mathrm{Gl}}
\newcommand{\e}[1]{\cdot 10^{#1}}
\newcommand*{\Loes}{\mathrm{L\ddot{o}s}}
\DeclareMathOperator*{\sgn}{\mathrm{sgn}}
\DeclareMathOperator*{\rang}{\mathrm{rang}}
\DeclareMathOperator*{\diag}{\mathrm{diag}}
\DeclareMathOperator*{\Ker}{\mathrm{Ker}}
\DeclareMathOperator*{\Eig}{\mathrm{Eig}}
\newcommand{\cond}{\mathrm{cond}_{\| \cdot \|}}
\newcommand{\conda}{\mathrm{cond}_{\| \cdot \|}_1}
\newcommand{\condb}{\mathrm{cond}_{\| \cdot \|}_2}
\newcommand{\condi}{\mathrm{cond}_{\| \cdot \|}_\infty}


\declaretheorem[style=remark,qed=$\blacklozenge$,parent=section]{Beispiel}
\declaretheorem[style=definition,qed=$\blacktriangleleft$,sibling=Beispiel]{Definition}
\declaretheorem[style=theorem,sibling=Beispiel]{Satz}
\declaretheorem[style=theorem,sibling=Beispiel]{Korollar}
\declaretheorem[style=theorem,sibling=Beispiel]{Lemma}
% Funktioneirt nicht richtig
% \declaretheorem[style=theorem,sibling=Beispiel]{Bemerkung}

\title{Numerische Mathematik}
\subtitle{Notizen zur Vorlesung von Prof. Dr. Timo Reis, WiSe 2014/15}
\author{Laura Schüttpelz, Damian Hofmann, Michael Hufschmidt}
\date{\today}

\begin{document}
\maketitle
\tableofcontents

\section{Einleitung}
Was ist numerische Mathematik?
$\rightarrow$ Bereitstellung konstruktiver Verfahren zur 
approximativen Berechnung mathematischer Sachverhalte.

\subsubsection*{Bsp: Lineares Gleichungssystem (LGS) Ax = b}
\textbf{gegeben:} $A \in \mathbb{R}^{n \times n}$ invertierbar 
$(A \in GI_{\sim} (\mathbb{R})), b \in \mathbb{R}^n$ \\
\textbf{gesucht:} $x \in \mathbb{R}^n$, so dass $Ax = b$ (also $x = A^{-1} b$)

"`Naive Möglichkeit"': Lösung über Cramersche Regel

\begin{displaymath}
x = \begin{pmatrix} x_a \\ \vdots \\ x_n \end{pmatrix}, 
~ x_i = \frac{1}{\det (A)},\\~ \det(A_i)~ mit~ 
A_i = \begin{pmatrix} a_{11} & \dots & b_1 & \dots & a_{n1} 
\\ \vdots & & \vdots & & \vdots \\ a_{n1} & \dots & b_n & 
\dots & a_{nn} \end{pmatrix}
\end{displaymath}

Leibnitz-Darstellung von $\det(A)$:
\begin{displaymath}
\det(A) = \sum_{\sigma \in S_n} sign (\sigma) a_{1\sigma(1)} 
\dots a_{n\sigma(n)}
\end{displaymath}

\subsubsection*{Fakten:}
\begin{enumerate}[(a)]
\item\label{leibnitz:multi}
Leibnitz-Darstellung von $\det(A)$ benötigt $n! \cdot (n-1)$ Multiplikationen
\item
Aus (\ref{leibnitz:multi}) folgt: Lösungen von $Ax=b$ mit Cramerschen Regel benötigt \\
$2 \cdot  n!\cdot(n-1) \cdot n + n$ Multiplikationen 
\end{enumerate}

Rechenleistung einer moderaten Maschine Intel Core i7: $3.3 \cdot 10^{10} 
\frac{\text{Multiplikationen}}{\text{Sekunde}}$ 

\begin{description}
\item[\rm Fall $n = 20$:] $\frac{2 \cdot 20! \cdot 19 + 20}{3.3 \cdot 10^{10} 
\cdot 60 \cdot 60 \cdot 24 \cdot 365} = 1776.3$ Jahre
\item[$n = 15$:] 4.6 Std
\item[$n = 10$:] 0.2 Sek.
\item[$n = 25$:] 17.8 Mrd. Jahre
\end{description}

Mithilfe anderer "`geschickter Verfahren"' kann man solche LSG 
(heutzutage bis zu $10^9$) in vernünftiger Zeit lösen.

Eine vage Definition

\begin{Definition}
Sei ein numerisches Verfahren abhängig von den 
Größen $n_1, \dots, n_l \in \mathbb{N}$. Man spricht vom 
"`Aufwand $O(f(n_1, \dots, n_l))$ Multiplikationen"'
(für $f: \mathbb{N}^l \rightarrow \mathbb{N}$) wenn es 
$c_1, c_2 > 0$ gibt, sodass das Verfahren für alle 
$n_1, \dots, n_l > c_1$ mit maximal $c_2 \cdot f(n_1, \dots n_l)$
Multiplikationen auskommt (Divisionen zählen als Multiplikationen, 
Additionen zählen gar nicht).
\end{Definition}

\subsubsection*{Beispiele:}

\begin{enumerate}[(a)]
\item Cramersche Regel zur Lösung von
      $Ax = b$ mit $A \in GI_{\sim}(\mathbb{R}), 
      b \in \mathbb{R}^n$ : \\
      O($n! \cdot (n-1)\cdot n$) Multiplikationen
\item Euklidisches Skalarprodukt:
      $\langle x,y\rangle = \sum_{k=1}^n x_k y_k$, $x,y \in \mathbb{R}^n$ :\\
      O(n) Multiplikationen
\item Matrixmultiplikation: $A \cdot B \text{ für } A\in 
      \mathbb{R}^{n \times m}, B \in \mathbb{R}^{m \times k}$ : \\
      $O(n \cdot m \cdot k ) $ Multiplikationen
\end{enumerate}  

\subsubsection*{Weitere Fragestellungen der Numerik:}

\begin{enumerate}[(a)]
\item
Numerische Integration, Berechne $\int_a^b f(x) dx$ numerisch
\item
Lösung nichtliearer Gleichungssysteme \\
$F(x) = 0$, $F: U \subset \mathbb{R}^n \rightarrow \mathbb{R}^k$ gegeben \\
Bsp.:
\begin{enumerate}[(i)]
\item
$f(x)=e^x-y$, $y \in \mathbb{R}$ gegeben. \\
$\rightarrow$ gesch! Lösung $x = \log y$ 
\item
$f(x)=x \cdot e^x -1 = 0$ nicht gesch! lösbar. \\
Lösung existiert jedoch in $(0,1)$, da $f(0) = -1 < 0$, \\
$f(1)= e-1>0 $ (Rest folgt aus Zwischenwertsatz)
\end{enumerate}
\item
Interpolation \\
Gegeben: $(x_1,y_1), \dots (x_n,y_n) \in \mathbb{R}^2$\\
Finde $f: I \subset \mathbb{R} \rightarrow \mathbb{R}$, so dass 
$f(x_1) = y_1, \dots , f(x_n)=y_n$
\item
numerische Berechnung von Eigenwerten
\item
numerische Lösung gew. Differentialgleichungen $x(t)=f(t,x(t))$
\item
lineare Optimierung
\end{enumerate}

\section{Lineare Gleichungssysteme}
\subsection{Grundlagen}
Notation: $\mathbb{K} \in \{\mathbb{R}, \mathbb{C}\}$ \\

\begin{align*}
x &= \begin{pmatrix} x_1 \\ \vdots \\ x_n \end{pmatrix} 
\in \mathbb{K}^n \widehat{=} \mathbb{K}^{n \times 1},  
~ A=(a_{ij}) \in \mathbb{K}^{m \times n}  \\
x^T &= \begin{pmatrix} x_1 & \dots & x_n \end{pmatrix} 
\in \mathbb{K}^{1 \times n}, ~A^T = (a_{ij}) \in \mathbb{K}^{n \times n}  \\ 
x^{\ast} &= \begin{pmatrix}\overline{x_1} & \dots & \overline{x_n} \end{pmatrix}, 
~A^{\ast} = (\overline{a_{ij}}) \in \mathbb{K}^{n \times m}  \\
GI_{\sim} (\mathbb{k}) &= \{A \in \mathbb{K}^{n \times n}: A 
\text{ invertierbar} \}  \\
O_n (\mathbb{K})&= \{ U \subset \mathbb{K}^{n \times n}: 
U^{\ast} U = I \} , \text{ $I$ ist Einheitsmatrix.} \\
e_i &= \begin{pmatrix} 0 \\ \vdots \\ 1 \\ \vdots \\ 0 \end{pmatrix} 
\text{: $i$-ter kanonischer Einheitsvektor } \\
\end{align*}

\subsubsection*{Lineares Gleichungssystem (LGS): Ax=b}


\textbf{gegeben:} $A \in \mathbb{K}^{m \times n}, b \in \mathbb{K}^m$

\textbf{gesucht:} Lösungsmenge $\Loes(A,b) = \{x \in \mathbb{K}^n : Ax=b\} $


Def.: $r:=\rang(A), ~ s := \rang(A,b)$
\begin{align*}
\text{Fälle: } b =0 : r = n &\Rightarrow \Loes (A,b)= \ker(A)=\{0\}\\
r<n &\Rightarrow \Loes (A,b) \text{ ist $n-r$ dimensional, Unterraum 
von } \mathbb{K}^n \\
b \neq 0: r<s &\Rightarrow \Loes (A,b) = \emptyset \\
r=s=n &\Rightarrow \Loes (A,b) \text{ einelementig }\\
r=s<n &\Rightarrow \Loes (A,b) = x + \ker(A) () \Loes  
\text{ (mit $n-r$ Parametern), $n-r$-dim. } \\
\text{Speziell: } n=m: b=0 &\Rightarrow r=n :  \Loes (A,b) = \{0\}\\
&\Rightarrow r<n : \Loes (A,b) = \ker(A)\\
b \neq 0: r=n &\Rightarrow \Loes (A,b) = \{A^{-1}b\}\\ 
	r<n, r<s &\Rightarrow \Loes (A,b) = \emptyset \\
	 r=s &\Rightarrow \text{Lösung mit $n-r$ Parametern}
\end{align*} 

\paragraph{Bekannte Lösungsmethode:} Gaußsches Eliminationsverfahren

$Ax=b \rightarrow \tilde{A}x= \tilde{b} \\$

$(\tilde{A},\tilde{b})=
\begin{pmatrix} 
\widetilde{a_{11}} & &\dots  &  & \widetilde{a_{1n}} & \widetilde{b_1} \\
0 & & \ddots &  & \vdots & \vdots \\
0 & \dots & a_{rr} & \dots & a_{rn} & \widetilde{b_r} \\
0 & & \dots & & 0 & \widetilde{b_n} \end{pmatrix} $
mit $\widetilde{a_{11}}, \dots, a_{rr} \in \mathbb{K} \backslash \{0\} $

Es gilt $\rang(A)= \rang (A,b) \Leftrightarrow 
\widetilde{b_{r+1}} = \dots = \widetilde{b_n} = 0 $

\subsubsection*{Grundlagen der linearen Algebra:}

\begin{enumerate}[(i)]
\item
\[
\lambda \in \mathbb{C}\text{ heißt "`Eigenwert von } A \in 
\mathbb{K}^{n \times n} \text{ "'.} \]
\[
: \Leftrightarrow \exists x \in \mathbb{C}^n \backslash \{0\}: Ax = \lambda x \\
\]\[
\Leftrightarrow \ker (\lambda I -A) \neq \{0\} \Leftrightarrow 
\det (\lambda I -A) =0, \ker (\lambda I-A) = Eig (A, \lambda)
\]\[
A \in \mathbb{K}^{n \times n} \text{ besitzt Eigenwerte (Korollar des 
Fundamentalsatzes der Algebra)}
\]\[
\sigma (A) = \{ \lambda \ in~ \mathbb{C}: \lambda \text{ ist Eigenwert 
von A}\} \text{ "`Spektrum von A"' }
\]
\item
\[
A \in \mathbb{K}^{n \times n} \text{ Hermitesch (dh. } A = A^\ast \text{) } a_{ij} = \overline{a_{ji}}
\]
Dann gilt 
\begin{enumerate}[(a)]
\item
$ \sigma(A) \in \mathbb{R}$
\item
$\exists$ Orthonormalbasis (ONB) aus Eigenvektoren von A. D.h. $\exists u_1, \dots, u_n \in \mathbb{K}^n \text{ mit } u_i^\ast u_j = \delta_{ij}$
$\land A~u_i = \lambda_i~u_i$ für ein $\lambda_i \in \sigma(A) 
(\forall i \in \{1, \dots, n\})$  
\end{enumerate}
$\Leftrightarrow$ Für $U = [u_1, \dots, u_n] \in \mathbb{K}^{n\times n}$ gilt $U^\ast U \in I$
(dh. $U \in \sigma_n(\mathbb{K})$) und 
$U^\ast AU = \diag(\lambda_1, \dots, \lambda_n)$  

\item
	$P \in \mathbb{R}^{n \times n}$ heißt "`Permutationsmatrix"' , falls $P=(e_{\sigma(1)} \dots e_{\sigma(n)})$ für ein $\sigma \in S_n$
	Dann 
	\begin{enumerate}[(a)]
	\item Für $A = (a_1, \dots, a_n) \in \mathbb{R}^{m \times n}$ gilt
	$AP = \begin{pmatrix} a_{\sigma(1)} & \dots & a_{\sigma(n)}\end{pmatrix}$
	\item
	Für $A = \begin{pmatrix} a_1 \\ \vdots \\ a_m \end{pmatrix}, P = (e_{\sigma(1)} \dots e_{\sigma(m)})$
	$\Rightarrow P^T A = \begin{pmatrix} a_{\sigma(1)} \\ \cdots \\ a_{\sigma(n)}  \end{pmatrix}$
\end{enumerate}
\end{enumerate}

\subsubsection{Definitheit von Matrizen}
Im Folgenden sei $A$ eine hermitesche Matrix.
\footnote{Das $A_i$ in diesem Abschnitt ist nicht zu verwechseln mit der Matrix $A_i$
in der Einleitung, Formel (\ref{cramer})}
\begin{enumerate}
  \item [a)] $A$ heißt "`positiv definit"' (Schreibweise: $A > 0$) wenn
  \begin{align*}
    &\text{\textbullet} \quad \forall x \in \KK^n \setminus \{0\}: x^* A x > 0 \\
    \Leftrightarrow \quad &\text{\textbullet} \quad \det(A_i) > 0 \; \forall \; i = 1, \dots , n \\
    \intertext{\qquad \qquad \qquad dabei ist $A_i$ der $i$-te Hauptminor
    von $A$, d.h.
    die $i\times i$ Teilmatrix}
    & \quad \quad A_i =
    \begin{pmatrix}
      a_{11} & \cdots  & a_{1i} \\
      \vdots & \ddots & \vdots \\
      a_{i1} & \cdots & a_{ii}
    \end{pmatrix} \\
    \Leftrightarrow \quad &\text{\textbullet} \quad  \sigma (A) \subset (0, \infty)
  \end{align*}
  \item[b)]$A$ heißt "`positiv semidefinit"' (Schreibweise: $A \ge 0$) wenn
    \begin{align*}
    &\text{\textbullet} \quad \forall x \in \KK \setminus \{0\}: x^* A x \ge 0 \\
    \Leftrightarrow \quad &\text{\textbullet} \quad  \sigma (A) \subset [0, \infty)
  \end{align*}
  \item[c)] $A$ heißt "`negativ (semi-) definit"': Analog.
\end{enumerate}

\subsubsection{Wurzel einer Matrix}
Sei $A \in \KK^{n \times n}$ ; $A \ge 0$ (das beinhaltet $A = A^*$). Dann ist
$\sqrt{A} = A^{\frac{1}{2}}$ definiert als eine Matrix $\sqrt{A} \ge 0$ mit
$\left(\sqrt{A} \right)^2 = A$.

\begin{proof}[Existenzbeweis]
$A \ge 0 \Rightarrow \exists \; U \in \U_n(\KK)$ ($\U_n = $ Menge
der unitären Matrizen), so dass  \newline
$A = U^* \cdot \diag(\lambda_1, \cdots , \lambda_n) \cdot U$ mit den Eigenwerten
$\lambda_1, \cdots , \lambda_n \in [0, \infty)$. \newline
Betrachte nun $\sqrt{A} = U^* \cdot \diag(\sqrt{\lambda_1}, \cdots , \sqrt{\lambda_n}) \cdot U$.
\end{proof}

\subsection{Vektor- und Matrixnormen}
\begin{Definition} [Norm]
  Sei $X$ ein $\KK$-Vektrorraum. Eine Abbildung $\| \cdot \| \rightarrow [0, \infty)$
  heißt Norm wenn
  \begin{enumerate}
    \item[(i)]    $\|x\| > 0 \quad \forall \; x \in X \setminus \lbrace 0 \rbrace$ (Positivität)
    \item[(ii)]   $\| \lambda \cdot x\| = |\lambda| \cdot \|x\| \quad \forall \; \lambda \in \KK, \forall x \in X$ (absolute Homogenität)
    \item[(iii)]  $\|x + y \| \le \|x\| + \|y\| \quad \forall \; x, y \in X$ (Dreiecksungleichung)
  \end{enumerate}
\end{Definition}

\begin{Bemerkungen}
\quad
  \begin{enumerate}
    \item[a)] Konvergenz $(x_n) \rightarrow x$ bedeutet $\|x_n - x\| \rightarrow 0$.
    \item[b)] Wenn zusätzlich $\dim (X) < \infty$, dann sind alle Normen auf $X$
    äquivalent, d.h.für jede Norm $\| \cdot \|_a$ und $\| \cdot \|_b$ auf $X$ gibt
    es Konstanten $c_1, c_2 > 0$, so dass
    $c_1 \cdot \| x \|_a \le \| x \|_b \le c_2 \cdot \| x \|_a \; \forall x \in X$.
    Es ist klar, dass $\dim(\KK^n) = n < \infty$ und
    $\dim(\KK^{m \times n}) = m \cdot n  < \infty$ .
    \item[c)] Beispiele für Normen auf $\KK^n$:
    \begin{enumerate}
      \item[(i)] $\|x\|_\infty := \max_{i = 1, \cdots , n} |x_i| \quad $Maximums-Norm
      \item[(ii)] $\|x\|_p := \left( \sum_{k = 1}^n |x_k|^p \right)^{1/p} \quad$
      $p$-Norm; $p \in [1, \cdots , \infty)$. Für $p = 2$ Euklidische Norm.
    \end{enumerate}
    Es gilt:
    $\|x\|_\infty \le \|x\|_2 < \|x\|_1 \le \sqrt{n} \cdot \|x\|_2 < n \cdot \|x\|_\infty
    \quad \forall \; x \in \KK^n$
  \end{enumerate}
\end{Bemerkungen}

Beispiele für Matrix-Normen ($A \in X = \KK^{m \times n})$:
\begin{enumerate}
  \item[a)] $\|A\|_1 :=  \max_{j = 1, \cdots , n} \left( \sum_{i = 1}^m |a_{ij}| \right) \quad$ Spaltensummen-Norm
  \item[b)] $\|A\|_\infty :=  \max_{i = 1, \cdots , m} \left( \sum_{j = 1}^n |a_{ij}| \right) \quad$ Zeilensummen-Norm
  \item[c)] $\|A\|_{\text{F}} :=  \left(\sum_{i = 1}^m  \sum_{j = 1}^n |a_{ij}|^2 \right)^{1/2} \quad$
  Frobenius-Norm
\end{enumerate}

\begin{Definition}
Eine Matrix-Norm $\|\cdot\|$ heißt
\begin{enumerate}
  \item[(i)] \textit{sub-multiplikativ}, wenn
  $$\|A \cdot B\| \le \|A\| \cdot \|B\|$$
  für alle $A \in \KK^{m \times k}, \; B \in \KK^{k \times n}$
  \item[(ii)] \textit{verträglich mit der Vektor-Norm $\| \cdot \|_V$}, wenn
  $$\|A \cdot x\| \le \|A\| \cdot \|x\|_V$$
  für alle $A \in \KK^{m \times n}, \; x \in \KK^n$.
\end{enumerate}
\end{Definition}

\begin{Beispiele}
\quad
\begin{enumerate}
  \item[(i)] $$\|A\| := \max_{\substack{i = 1 \cdots m\\j = 1 \cdots n}}|a_{ij}|$$
  ist eine Matrix-Norm, jedoch nicht submultiplikativ, denn für \\
  $A = B = \left( \begin{smallmatrix}1 & 1 \\1 & 1 \end{smallmatrix} \right)$ gilt
  $\|A\| = \|B\| = 1$, aber
  $\|A \cdot B\| = \left \| \left(\begin{smallmatrix}2 & 2 \\2 & 2 \end{smallmatrix}\right) \right\| =
  2 \nleq 1 = \|A\| \cdot \|B\|$.
  \item[(ii)] Die Frobenius-Norm ist mit der euklidischen Norm verträglich:
  \begin{align*}
    & |(A \cdot x)_i|^2 = \sum_{j = 1}^n |a_{ij} \cdot x_j |^2 \overset{(*)}{\le}
    \left( \sum_{j = 1}^n |a_{ij}|^2  \right) \cdot \left( \sum_{i = 1}^n |x_i |^2  \right) =
    \sum_{j = 1}^n |a_{ij}|^2 \cdot |x|^2.
    & \intertext{Bei $(*)$ wurde die Cauchy-Schwarzsche Ungleichung angewendet. Es folgt}
    & \|A \cdot x\|_2 \le \|A\|_{\text{F}} \cdot \|x\|_2 \quad
    \forall \; A \in \KK^{m \times n} \; , \; x \in \KK^n.
  \end{align*}
\end{enumerate}
\end{Beispiele}

\begin{Definition} [induzierte Matrixnorm]
  Sei $\| \cdot \|$ eine Vektornorm auf $\KK^n$, dann ist
  $$\|A\| := \sup_{ x\ne 0} \frac{\| A x \|}{\| x \|}$$ eine Matrixnorm
  auf $\KK^{m \times n}$, die sogenannte 
  \emph{von $\| \cdot \|$ induzierte Matrixnorm}.
  Sie wird auch als die \emph{$\| \cdot \|$ zugeordnete Matrixnorm} bezeichnet.
\end{Definition}

\begin{Bemerkungen}
\quad
  \begin{enumerate}
    \item[a)] Die Normeigenschaften lassen sich einfach nachprüfen.
    \item[b)] $$\|A\| := \sup_{ x\ne 0} \frac{\| A x \|}{\| x \|} =
    \sup_{ x\ne 0} \left\| A \frac{x}{\| x \|} \right\| =
    \sup_{ \|x \| = 1} \| A x \|
   \overset{(*)}{=} \max_{ \|x \| = 1} \| A x \|.$$
   $(*)$ gilt, weil die Menge $\{x \in \KK^n : \|x \| = 1\}$ kompakt ist.
  \end{enumerate}
\end{Bemerkungen}

\begin{Satz} Die von $\| \cdot \|$  induzierte Matrixnorm ist
  \begin{enumerate}
    \item [a)] eine Norm
    \item [b)] verträglich mit $\| \cdot \|$
    \item [c)] submultiplikativ.
  \end{enumerate}
\end{Satz}
\begin{proof}
\quad
\begin{enumerate}
  \item [a)] Norm:
    \begin{itemize}
      \item Positivität: $A \ne 0 \; \Rightarrow \; \exists x \text{ so dass }
    A x \ne 0 \; \Rightarrow \; \frac{\| A x \|}{\| x \|} > 0
    \; \Rightarrow \;  \sup_{ x\ne 0} \frac{\| A x \|}{\| x \|} > 0$
      \item Homogenität: $\lambda \in \KK:\; \forall A \in \KK^{m \times n} \;
      \Rightarrow \| \lambda A \| = \sup_{ x\ne 0} \frac{\| \lambda A x \|}{\| x \|} =
      \sup_{ x\ne 0} |\lambda|  \frac{\| A x \|}{\| x \|} =
      |\lambda| \sup_{ x\ne 0} \frac{\| A x \|}{\| x \|}$
      \item Dreiecks-Ungleichung: $\forall A, B  \in \KK^{m \times n} \; \Rightarrow \;
      \sup_{ x\ne 0} \frac{\| (A + B) x \|}{\| x \|} \le
      \sup_{ x\ne 0} \frac{\| A x \| + \| B x \|}{\| x \|} \le
      \sup_{ x\ne 0} \frac{\| A x \|}{\| x \|} + \sup_{ x\ne 0} \frac{\| B x \|}{\| x \|}$
    \end{itemize}
  \item [b)] Verträglichkeit: Sei $A \in \KK^{m \times n} \; , \; x \in \KK$.\\
    1. Fall $x = 0  : \quad \Rightarrow \; \| A x \| = 0 = \|A \| \; \|x\| $\\
    2. Fall $x \ne 0  : \quad \Rightarrow \; \| A x \| =
    \| A \frac{x}{\|x\|}\| \cdot \| x \| \le
    \max_{y = 1} \| A y \| \cdot \| x \| = \| A \| \cdot \| x \|$
  \item [c)] Submultiplikativität: $\forall A, B  \in \KK^{m \times n} \text{ gilt: }
  \sup_{ x\ne 0} \frac{\| A \cdot B x \|}{\| x \|} =
  \sup_{ x\ne 0} \| A \| \frac{\| B x \|}{\| x \|} = \\
  \text{ (wegen Verträglichkeit) } = \| A \|  \sup_{ x\ne 0}  \| \frac{\| B x \|}{\| x \|}
  = \|A\| \cdot \|B\|$
\end{enumerate}
\end{proof}

\begin{Satz}
  Die Spaltensummennorm wird durch $\| \cdot \|_1$ induziert, die Zeilensummennorm
  wird durch $\| \cdot \|_\infty$ induziert.
\end{Satz}
\begin{proof}
  Übungsblatt 2 oder J. Werner, Numerische Mathematik 1, Seite 19.
\end{proof}

\begin{Definition}[Spektralradius]
Sei $A \in \KK^{m \times n}$.
$$\rho(A) := \max |\sigma(A)| =  \max_{\lambda \in \sigma(A)} |\lambda|$$
heißt \emph{Spektralradius von $A$}.
\end{Definition}

\begin{Bemerkung}
Für  $A \in \KK^{m \times n} \; \Rightarrow \; A^* A \ge 0 $ d.h.
$A^* A$ ist positiv semidefinit,
denn $x^* A^* A x = (A x)^* (A x) = \|A x\|_2^2 \ge 0  \; \Rightarrow \;
\sigma(A^* A) \subset [0, \infty)$ d.h. das Spektrum ist nicht negativ.
\end{Bemerkung}

\begin{Satz}
  Die der euklidischen Norm $\| \cdot \|_2$ zugeordnete Matrixnorm ist
  die \emph{Spektralnorm} $$\|A\|_2 = \sqrt{\rho(A^* A)}.$$
\end{Satz}
\begin{proof}
Sei $A^* A = U^* \diag(\lambda_1, \cdots , \lambda_n) U \text{ mit }
U \in \U_n(\KK), \; \lambda_1 \ge  \lambda_2 \ge \cdots \ge \lambda_n \ge 0$. Dann gilt:
\begin{align}
  \nonumber
  \| U x \|_2^2 &= (U x)^* (U x) = x^* U^* U x = x^* x = \| x \|_2^2
  \intertext{Das bedeutet: Unitäre Transformationen erhalten die Länge. Daher gilt für}
  \nonumber
  \| A x \|_2^2 &= (A x)^* (A x) =  x^* A^* A x  =
  (U x)^* \diag(\lambda_1, \cdots , \lambda_n) U x \\
  \nonumber
  &= \sum_{k = 1}^n \lambda_k \|(U x)_k\|_2^2 \le \lambda_1 \sum_{k = 1}^n \|(U x)_k\|_2^2 =
  \lambda_1 \|U x \|_2^2 =  \lambda_1 \|x\|_2^2 = \rho(A^* A) \|x\|_2^2 \\
  \nonumber
  &\Rightarrow \quad \|A x \|_2 \le \rho(A^* A)^{1/2} \cdot \|x\|_2 \\
 \label{euklid1}
  &\Rightarrow \quad \rho(A^* A)^{\frac{1}{2}} \ge \|A \|_2 =
  \sup_{x\ne 0} \frac{ \|A x\|_2}{ \|x\|_2}
\intertext{ Sei nun $\hat x \in \KK^n$ mit $\|\hat x\|_2 = 1$ und
  $A^* A \hat x = \lambda_1 \hat x$. Dann gilt:}
  \nonumber
  \|A\|_2^2 &\ge \| A \hat x\|_2^2 = \hat x^* A^* A \hat x =
  \lambda_1 \hat x^* \hat x = \lambda_1 \| \hat x\|_2^2 = \lambda_1 = \rho(A) \\
  \label{euklid2}
  &\Rightarrow \quad \|A\|_2 \ge \rho(A^* A)^{1/2}
\end{align}
Aus (\ref{euklid1}) und (\ref{euklid2}) folgt $\|A\|_2 = \rho(A^* A)^{1/2}$
\end{proof}




\subsection{Konditionen einer Matrix}

Sei $A \in \Gl_n(\KK), \; b \in \KK^n $ mit einem linearen Gleichungsssystem (LGS)
$A \cdot x = b$. Wir betrachten nun ein "`gestörtes Problem"' $A \cdot \tilde{x} = \tilde{b}$,
wobei $\tilde{b}$ von $b$ durch Meß- oder Rundungsfehler abweicht: Wir definieren
$x_d := \tilde{x} - x$ und $b_d := \tilde{b} - b$. Damit wird:
\begin{align*}
  & A \cdot x_d = b_d \quad \Rightarrow \quad x_d = A^{-1} \cdot b_d \\
  & \Rightarrow \quad \|x_d\| \le \|A^{-1}\| \cdot \|b_d\| \\
  & \| b \| = \| A \cdot x \| \le \|A\| \cdot \|x\|\\
  & \Rightarrow \quad \|x \| \ge \|A^{-1}\| \cdot \|b\|
\intertext{damit kann man den relativen Fehler abschätzen:}
 & \frac{\|x_d\|}{\|x\|} \le  \|A\| \cdot \|A^{-1}\| \cdot \frac{\|b_d\|}{\|b\|}
\end{align*}

\begin{Definition}[Kondition]
  Für $A \in \Gl_n(\KK)$ heißt $\cond(A) :=  \|A^{-1}\|\cdot\|A\|$ die Kondition von $A$.
\end{Definition}
\begin{Bemerkungen}
\quad \\
  \begin{enumerate}
    \item[a)] Sprechweise: Eine Matrix $A$ heißt gut / schlecht konditioniert, wenn
    $\cond(A)$ klein / groß ist.
    \item[b)] Wegen $1 = \|I\| = \|A^{-1} \cdot A\| \le \|A^{-1}\|\cdot\|A\| = \cond(A)$
    gilt stets $\cond(A) \ge 1 \; \forall A \in \Gl_n(\KK)$ und für alle
    Matrizennormen $\|\cdot\|$.
    \item[c)] Unitäre Matrizen $U \in \U_n(\KK)$ haben stets $\cond(U) = 1$
    \item[d)] Für das obige Problem gilt: $\frac{\|x_d\|}{\|x\|} \le
      \cond (A) \cdot \frac{\|b_d\|}{\|b\|}$.
    \item[e)] Veranschaulichung: [Bild: Einheitskreis im $\RR^n$ wird durch
    eine Matrix $A$ in eine schmale schräg gestellte Ellipse mit der großen bzw.
    kleinen Halbachse $a$ bzw. $b$ transformiert]. Die "`maximale Streckung"'
    ist $a = \max_{\|x\| = 1} \| A \cdot x\| = \|A\|$, die "`minimale Streckung"' ist
    \begin{align*}
    b &= \min_{\|x\| = 1} \| A \cdot x\| =
    \left( \max_{\|x\| = 1} \frac{1}{\| A \cdot x\|} \right)^{-1} =
    \left( \max_{\|x\| \ne 0} \frac{\|x\|}{\| A \cdot x\|} \right)^{-1} \\
    \intertext{Substitution  $y = A^{-1} \cdot x$}
     &= \left( \max_{\|y\| \ne 0} \frac{\|A^{-1} \cdot y\|}{\|y\|} \right)^{-1}
     = \frac{1}{\|A^{-1}\|}
    \end{align*}
  \end{enumerate}
\end{Bemerkungen}

Konsequenz: $\cond(A) = \frac{\text{maximale Streckung des Einheitskreises}}
{\text{minimale Streckung des Einheitskreises}}$

\begin{Lemma}[Störungslemma]
  Sei $A \in \Gl_n(\KK), \; S \in \KK^{n \times n}$ so dass
  $\|A^{-1}\| \cdot \|S\| < 1$. Dann gilt: $A + S \in \Gl_n(\KK)$ mit
  $\|(A + S)^{-1}\| \le \frac{\|A^{-1}\|}{1 - \|A^{-1}\| \cdot \|S\|}$.
\end{Lemma}
\begin{proof}
Es gilt $A + S = A(I + A^{-1} S)$ und ferner:
\begin{align}
   \nonumber
   \|(I + A^{-1} S) \cdot y\| &\ge \|y\| - \|A^{-1} \cdot S \cdot y\|  \ge \\
   \label{stoer1}
   & \ge \|y\| - \|A^{-1}\| \cdot \| S \| \cdot \|y\| =
   \underbrace {\left( 1 - \|A^{-1}\| \cdot \|S\| \right)}_{> 0}  \cdot \|y\| \\
   \nonumber
   & \Rightarrow \quad I + A^{-1} \cdot S  \in \Gl_n(\KK) \; \text{ ist invertierbar}\\
   \nonumber
   & \Rightarrow \quad A + S \in \Gl_n(\KK) \; \text{ ist invertierbar}
\end{align}
Mit $y = (I + A^{-1} \cdot S) \cdot x$ folgt aus Gleichung (\ref{stoer1}:
\begin{align*}
   \|x\| & \ge \left( 1 - \|A^{-1}\| \cdot \|S\| \right) \cdot
     \left\| (I + A^{-1} \cdot S)^{-1} \cdot x \right\| \\
  & \Rightarrow \; \left\| (I + A^{-1} \cdot S)^{-1} \cdot x \right\| \le
     \frac{1}{1 - \|A^{-1}\| \cdot \|S\|} \cdot \|x\|\\
     \left\|(A+S)^{-1}\right\|  & = \left\| (I + A^{-1} \cdot S)^{-1} A^{-1} \right\| \le
     \left\| (I + A^{-1} \cdot S)^{-1} \right\|\cdot \| A^{-1} \| \le
       \frac{\|A^{-1}\|}{1 - \|A^{-1}\| \cdot \|S\|}
\end{align*}
\end{proof}

\begin{Satz}
Sei $A \in \Gl_n(\KK), \; A_d \in \KK^{n \times n}$  mit  $\|A^{-1}\| \cdot \|A_d\| \le 1$.
Seien $b, b_d \in \KK^n$  und  $x, x_d \in \KK^n$, so dass $A \cdot x = b$ und
$(A + A_d)(x + x_d) = (b + b_d)$. Dann gilt:
\begin{align*}
\frac{\|x_d\|}{\|x\|} = \frac{\cond(A)}{1 - \cond(A) \cdot \|A_d\|/\|A\|} \cdot
\left( \frac{\|A_d\|}{\|A\|} + \frac{\|b_d\|}{\|b\|} \right)
\end{align*}
\end{Satz}
\begin{proof}
Aus $\|A^{-1}\| \cdot \|A_d\| \le 1$ folgt nach dem Störungslemma
$A + A_d \in \Gl_n(\KK)$. Dann gilt:
\begin{align*}
A \cdot x & = b \;  ; \quad (x + x_d) = (A + A_d)^{-1} (b + b_d) \quad \Rightarrow \\
\|x_d\| & = \| (A + A_d)^{-1} (b + b_d) - A^{-1} b \| =
   \| (A + A_d)^{-1} (b + b_d -(A +  A_d) A^{-1} b) \| \\
  & = \| (A + A_d)^{-1} (b + b_d - b - A_d A^{-1} b) \| \\
  & \le \| (A + A_d)^{-1}\| \cdot \| b_d - A_d x \| \le
  \| (A + A_d)^{-1}\| \cdot \left( \|b_d\| + \|A_d\| \cdot \|x\|  \right)
\intertext{Mit dem Störungslemma ergibt sich daraus}
  & \le \frac{\|A^{-1}\|}{ 1 - \|A^{-1}\| \cdot \|A_d\|} \left( \|b_d\| + \|A_d\| \cdot \|x\| \right) \\
  & =  \frac{\cond(A)}{ 1 - \cond(A) \cdot \|A_d\|/\|A\|}
      \cdot \left( \frac{\|b_d\|}{\|A\|} + \frac{\|A_d\|}{\|A\|} \cdot \|x\| \right) \\
  & = \|x\| \cdot \frac{\cond(A)}{ 1 - \cond(A) \cdot \|A_d\|/\|A\|}
     \cdot \left( \frac{\|b_d\|}{\|A\| \cdot \|x\| } + \frac{\|A_d\|}{\|A\|} \right) \\
  & \le \|x\| \cdot \frac{\cond(A)}{ 1 - \cond(A) \cdot \|A_d\|/\|A\|}
     \cdot \left( \frac{\|b_d\|}{\|b\| } + \frac{\|A_d\|}{\|A\|} \right)
\end{align*}
\end{proof}

\subsection{Elementarmatrizen}

\begin{Definition}[Dyade, Elementarmatrix]
Eine Matrix $W \in \KK^{n \times n}$ heißt
\begin{enumerate}
  \item[a)] "`Dyade"' falls $\rang(W) = 1$ ist und
  \item[b)] "`Elementarmatrix"' falls $W - I$ eine Dyade ist.
\end{enumerate}
\end{Definition}
\begin{Bemerkungen}
\quad \\
  \begin{enumerate}
    \item[a)] Dyaden haben die Form $W = u \cdot v^*$ mit $u, v \in \KK^n\setminus \lbrace 0 \rbrace$
    \item[b)] Elementamatrizen haben die Form $W = I + u \cdot v^*$
    \item[c)] Rechenregeln: $A(u \cdot v*) = (A u) \cdot v^*$ und
      $A (I + u \cdot v^*) = A + (A \cdot u) \cdot v^*$
  \end{enumerate}
Aufwand: Die Berechnung von $(A \cdot u) \cdot v^*$ erfordert $O(n^2)$ Mulitplikationen.
Beachte: $A\cdot (u \cdot v*)$ erfordert $O(n^3)$ Mulitplikationen, analog
$u \cdot v^* \cdot A$ und $ (I + u \cdot v^*) \cdot A$.
\end{Bemerkungen}

\begin{Lemma}
Sei $A \in \Gl_n(\KK), \; u, v \in \KK^n$, dann gilt:
  \begin{enumerate}
    \item[(i)] $A + u  v* \in \Gl_n(\KK) \quad \Rightarrow \quad 1 + v^* A^{-1} u \ne 0$
    \item[(ii)] $A + v^* A u \ne 0  \quad \Rightarrow \quad
      (A + uv^*) = A^{-1} - \frac{A^{-1} u v^* A}{1 + v^*A^{-1}u}$
  \end{enumerate}
Die letzte Identität ist ein Spezialfall der Sherman-Morrison-Woodbury-Formel.
\end{Lemma}
\begin{proof}
\quad \\
  \begin{enumerate}
    \item["`$\Rightarrow$"'] Sei $1 + v^* A^{-1} u \ne 0 \quad \Rightarrow $
    \begin{align*}
    & (1 + u v^*)\cdot \left(A^{-1} - \frac{A^{-1} u v^* A}{1 + v^*A^{-1}u}  \right) =
    I + u v^*A^{-1} - \frac{u v^* A^{-1}}{1 + v^*A^{-1} u} -
       \frac{u v^* A^{-1}u v^* A^{-1}}{1 +  v^*A^{-1} u} \\
    & = I + \frac{1}{1 + v^*A^{-1}u} \left( u v^* A^{-1} + u v^* A^{-1} u v^* A^{-1} -
      u v^* A^{-1} - u v^* A^{-1} u v^* A^{-1} \right) = I
    \end{align*}

    \item["`$\Leftarrow$"'] Angenommen $1 + v^* A^{-1} u = 0 \quad \Rightarrow $
    \begin{align*}
      & u \ne 0 \Rightarrow A^{-1} u \ne 0 \quad \Rightarrow \\
      & (A + u v^*) (A^{-1} u) = u + \underbrace{v* A^{-1} u}_{= -1} = 0 \quad \Rightarrow \\
     &  A + u v^* \notin \Gl_n(\KK) \qquad \text{Widerspruch!}
    \end{align*}
  \end{enumerate}
\end{proof}

\begin{Beispiele}[Elementarmatrizen]
\quad
\begin{itemize}
\item[a)] \emph{Vertauschungsmatrix.}
  $P_{rs} := (e_1,\ldots,e_{r-1},e_s,e_{r+1},\ldots,e_{s-1},e_r,e_{s+1},\ldots,e_n).$

  Einheitsmatrix mit $r$-ter und $s$-ter Spalte vertauscht.

  Es gilt $P_{rs} = I - (e_r - e_s)(e_r - e_s)^\T$ und $P_{rs} = R_{rs}^\T$,
  $P_{rs}^2 = P_{rs}$.
  
  $P_{rs} A$ vertauscht $r$-te und $s$-te Spalte in A. Aufwand: 0 Mult.
  
  Es gilt $\det P_{rs} = -1$, es handelt sich also um eine Spiegelung.

\item[b)] \emph{Gauß-Matrix.}
  Sei $1 \le k \le n$, $u = (0, \ldots, 0, u_{k+1}, \ldots, u_n)^\T$.
  \[
    G_k := I - ue_k^\T = \begin{pmatrix}
      1 & & & \\
      & \ddots & & & \\
      & & 1 & & \\
      & & -u_{k+1} & \ddots & \\
      & & \vdots & & \\
      & & -u_n & & 1 \\
    \end{pmatrix}.
  \]

  Beachte: $G^{-1}_k = (I - ue_k^\T)^{-1} = I + ue_k^\T$.

  Für $A = \begin{pmatrix} a_1 \\ \vdots \\ a_n \end{pmatrix} \in \KK^{n\times n}$
  gilt \[
    G_k A = (I - ue_k^\T) A = A - u(e_k^\T A) = A - ua_k
          = \begin{pmatrix} a_1 \\ \vdots \\ a_k \\ a_{k+1} - u_{k+1} a_k \\ \vdots \\ a_n - u_n a_k \end{pmatrix}.
  \]
  Für $j = k+1, \ldots, n$ Subtraktion von $a_j$ mit $u_j a_k$.

  Aufwand (für $A \in \RR^{n\times m}$): $(n-k) \cdot m$ Mult.

\item[c)] \emph{Householder-Spiegelung.}
  $v \in \KK^n \setminus \{0\}$, $Q_v := I - \frac{2}{v^{*}v}vv^{*}.$
  Es gilt $Q_v^{*} = Q_v$ und \[
    Q_v^2 = (I - \frac{2}{v^{*}v}vv^{*})(I - \frac{2}{v^{*}v}vv^{*})
          = I - \frac{4}{v^*v}vv^* + \frac{4}{(v^*v)^2}vv^*vv^* = I.    
  \]

  Also ist $Q_v$ hermitesch und unitär (denn die Eigenwerte sind reell und
  haben Betrag 1).

  Es gilt $Q_v v = -v$ und $Q_v w = w$ für alle $w \in \vspan\{v\}^\perp$.
  Es handelt sich bei $Q_v$ also um eine Spiegelung an der Hyperebene
  $\vspan\{v\}^\perp$.

  \item[d)] \(
    \lambda \in \KK \setminus \{0\}$, $S_k := I + (\lambda - 1)e_ke_k^\T
      = \diag(1, \ldots, \underset{k-1}{1}, \underset{k}{\lambda}, \underset{k+1}{1}, \ldots, 1)
    \).

    $S_k^{-1} = I + (\frac{1}{\lambda} - 1) e_k e_k^\T.$

    \[
      S_k \begin{pmatrix} a_1 \\ \vdots \\ a_n \end{pmatrix}
        = \begin{pmatrix} a_1 \\ \vdots \\ a_{k-1} \\ \lambda a_k \\ a_{k+1} \\ \vdots \\ a_n \end{pmatrix}.
    \]

    Aufwand: $m$ Mult. für $A \in \KK^{n\times m}$.
\end{itemize}
\end{Beispiele}

\subsection{Gaußsches Eliminationsverfahren}

Gegeben: LGS $Ax = b$, $A \in \Gl_n(\KK)$, $b \in \KK^n$.

Ziel: Sukzessive Multiplikation von links mit Gaußmatrizen $G_i$ und Vertauschungsmatrizen $P_i$ ($i \in \{1, \ldots, n\}$),
so dass \[
  \underset{=: R\text{ (obere Dreiecksmatrix)}}{\underbrace{
    G_nP_n \cdots G_1 P_1 A x}}
  = \underset{=: c \in \KK^n}{\underbrace{
    G_nP_n \cdots G_1 P_1 b}}.
\]

Danach: Lösung von $Rx = c$ durch \emph{Rückwärtseinsetzen}:
\begin{codebox}
  \li \For $i = n, n-1, \ldots, 1$:
    \Do
      \li $x_i \gets \frac{1}{r_{ij}} \left( c_i - \sum_{j=i+1}^n r_{ij} x_j \right)$
    \End
\end{codebox}

Das Rückeinsetzen benötigt insgesamt \[
  \sum_{i=1}^n (n - i + 1) = n^2 + n - \frac{n(n+1)}{2} = \frac{n(n+1)}{2}
    = O(n^2)
\] Multiplikationen.

Transformation $(A \mid b) \to (R \mid c)$ über folgenden Prozess:
% TODO

Anschaulich: %TODO

Nach $n-1$ Schritten erhalten wir eine obere Dreiecksmatrix.

\begin{Bemerkung}
Das Verfahren ist für $A \in \Gl_n(\KK)$ immer durchführbar.
Wäre dies nämlich im $k$-ten Schritt ($1 \le k < n$) nicht der Fall, so
hätte die zugehörige Matrix $A^{(k)}$ die Form \[
  A^{(k)} = \left( \begin{array}{ccc|cccc}
  a_{11} & \cdots & a_{1k} & a_{1,k+1} & a_{1,k+2} & \cdots & a_{1n} \\
  & \ddots & \vdots & \vdots & \vdots & \ddots & \vdots \\
  0 & & a_{kk} & a_{k,k+1} & a_{k,k+2} & \cdots & a_{kn} \\ \hline
  0 & \cdots & 0 & 0 & a_{k+1,k+2} & \cdots & a_{k+1,n} \\
  \vdots & \ddots & \vdots & \vdots & \vdots & \ddots &  \vdots \\
  0 & \cdots & 0 & 0 & a_{n,k+1} & \cdots & a_{nn}
  \end{array} \right)
\] also \(
  A^{(k)} = \left( \begin{array}{c|c} \Lambda_1 & \Lambda_2 \\\hline 0 & \Lambda_3 \end{array} \right)
\) und damit
$$\pm \det(A) = \det(A^{(k)}) = \det(\Lambda_1) \det(\Lambda_3) = 0\text{, da }\det (\Lambda_3) = 0.$$
\end{Bemerkung}

\begin{Algorithmus}[Gauß-Verfahren]
\quad
\begin{codebox}
\Procname{$\proc{Gauss-Elimination}(A,b)$}
\li \For $k = 1, \ldots, n$:
\Do
  \li Bestimme $r(k) \in \{k, \ldots, n\}$, so dass
      $|a_{r(k),k}| = \max_{i=k,\ldots,n} |a_{ik}|.$
  \li Vertausche $k$-te und $r(k)$-te Zeile von $(A \mid b)$.
  \li \For $i \gets k + 1, \ldots, n$:
  \Do
    \li $b_i \gets b_i - \frac{a_{ik}}{a_{kk}} b_k$.
    \li \For $j \gets k+1, \ldots, n$:
    \Do
      \li $a_{ij} \gets a_{ij} - \frac{a_{ik}}{a_{kk}} a_{kj}$.
    \End
    \li $a_{ik} \gets 0$.
  \End
\End
\end{codebox}
\end{Algorithmus}

\begin{Bemerkungen}
\begin{itemize}
  \item[a)] Das Gauß-Verfahren ist genau dann ohne Pivotisierung möglich,
    wenn alle Hauptabschnittsdeterminanten von 0 verschieden sind. (Beweis in den Übungen.)

  \item[b)] Die Pivotisierung trägt zur numerischen Stabilität bei.

  Beispiel: Betrachte 
\end{itemize}
\end{Bemerkungen}
\documentclass[a4paper]{scrartcl}

\usepackage[utf8]{inputenc} \usepackage[ngerman]{babel}

\usepackage{amssymb} \usepackage{amsmath}
\usepackage{latexsym}
\usepackage{enumerate}

\newcounter{mydefctr}
\newtheorem{mydef}[mydefctr]{Definition}
%\newtheorem{vmydef}[mydefctr]{(vage) Definition}

%\newcommand{\rang}{\mathop{\mathrm{rang}}}
%\newcommand{\loes}{\mathop{\mathrm{L\ddot{o}s}}}
%\newcommand{\diag}{\mathop{\mathrm{diag}}}

\parindent 0pt
\parskip 0.25\baselineskip

\title{Vorlesung Numerik 27.10.2014}

\author{Laura Schüttpelz}

\begin{document


\end{document}



\begin{Algorithmus}[Cholesky-Zerlegung]\hfill\newline
	Eingabe: $A > 0, A \in \KK^{n\times n}$ (benutzt die \glqq untere linke Hälfte \grqq von A),\\
	Ausgabe: $ L \in \Gl_n(\KK)$ untere $\triangle$-Matrix mit $l_{kk} >0 \forall k\in \{1,\cdots,n\} \text{ und } A=LL^*$  Berechne über den Weg $(i,j) = (1,1), (2,1), \cdots , (n,1), (2,2), (2,3), \cdots , (2,n),
	(3,3), \cdots , (3,n), \cdots , (n,n)$
	\quad
	\begin{codebox}
		\Procname{$\proc{Cholesky-Zerlegung}$}
		\li $L := 0 \in \KK^{n\times n}$
		\li \For $k = 1, \ldots, n$:
		\Do
		\li $l_{kk} = \left(a_{kk} - \sum_{j=1}^{k-1}|l_{kj}|^{2}\right)^{\frac{1}{2}}$
		\li \For $i \gets k + 1, \ldots, n$:
		\Do
		\li $l_{ik} =\frac{1}{l_{kk}} \left(a_{ik} - \sum_{j=1}^{k-1}l_{ij}\overline{l_{kj}}\right)$
		\End
		\End
	\end{codebox}
\end{Algorithmus}
Aufwand:\begin{itemize}
	\item[a)] Berechnung von $l_{ij}$: $j$ Multiplikationen + 1 Wurzel falls $i=j$ 	\item[b)]$\Rightarrow$ Berechnung von L:\begin{align*}
	\sum_{i=1}^{n}\sum_{j=1}^{i}j = \sum_{i=1}^{n} \frac{i(i+1)}{2}= \frac{1}{2}\sum_{i=1}^{n} i^2 + \frac{1}{2} \sum_{i=1}^{n} i = \frac{1}{12}n(n+1)(2n+1)+\frac{n(n+1)}{4}  \\= O(n^3) + n\text{ Wurzel}
	\end{align*}
\end{itemize}
\begin{Beispiel}[Cholesky-Zerlegung]
	\quad
	$A = \begin{pmatrix}
	1 & 2 & 1\\
	2 & 5 & 2\\
	1 & 2 & 10
	\end{pmatrix}$\\
	\begin{align*}&l_{11}= \sqrt{a_{11}} = 1 \quad\Rightarrow l_{21}=\frac{a_{21}}{l_{11}}=2\\
	&l_{22}= \sqrt{a_{22}-l_{21}^2} = 1\\
	&l_{31}= \frac{a_{31}}{l_{11}} = 1, l_{32}= \frac{a_{32}-l_{31}l_{21}}{l_{22}} = 0\\
	&l_{33}= \sqrt{a_{33}-l_{31}^2-l_{32}^2} = \sqrt{9}=3\\
	&\Rightarrow L = \begin{pmatrix}
	1 & 0 & 0\\
	2 & 1 & 0\\
	1 & 0 & 3
	\end{pmatrix}
	\end{align*}
\end{Beispiel}

\subsection{QR-Zerlegung}
$A \in \RR^{m \times n} \text{ mit rang}(A) = n$\\
\newline
Ziel: Berechne $Q \in O_n(\RR)\quad
				R=\begin{pmatrix}
				R_1\\0
				\end{pmatrix}$ wobei $R_1$ obere $\triangle$-Matrix,
				$A=QR$ "QR-Zerlegung"\\
\newline
Nutzen für LGS $Ax=b: QRx = b\\ \Rightarrow \begin{pmatrix} R_1\\0\end{pmatrix}x= Q^Tb = \begin{pmatrix}c\\d\end{pmatrix} \text{ mit } c\in \RR^n, d \in \RR^{m-n}$\\
\newline
Dann: \begin{itemize}
	\item[(i)]Lösbarkeit $\Leftrightarrow d=0$
	\item[(ii)]Lösung erfüllt dann $R_1x=c$ ($\rightsquigarrow$Rückwärtseinsetzen)
\end{itemize}
Erinnerung: Householder-Spiegelung
\begin{align*}
&Q_v=I -\frac{2}{v^Tv}vv^T \;(v\in\RR^m)\\
&Q_v=Q_v^T, Q_v^2=I \quad (\Rightarrow Q_v \in O_m(\RR))
\end{align*}\\
Vorgehensweise: Sukzessive Multiplikation von links mit Householder-Matrizen s.d.\\ $ Q_{v_n}, \cdots, Q_{v_1} A =\begin{pmatrix}R_1\\0\end{pmatrix}\\
\Rightarrow A =QR \text{ für } R= \begin{pmatrix}R_1\\0\end{pmatrix} \text{ und } Q=Q_{v_1} \cdots Q_{v_n} $
\begin{Lemma}
	$x \in \RR^m \text{ mit } x_1 \neq 0$
	\newline Dann gilt für $v = \frac{x}{\|x\|_2}+sign(x_1)e_1$
	\begin{itemize}
		\item[a)]$\|v\|=\sqrt{2}\left(1+\frac{|x_1|}{\|x\|_2}\right)^2$
		\item[b)]$Q_vx=-sign(x_1)\|x\|_2e_1$
	\end{itemize}
\end{Lemma}
\begin{proof}
	Übungsaufgabe Aufgabe 8
\end{proof}
Konsequenz: Für $x \in \RR^m$ wähle $v=\frac{x}{\|x\|_2}+ sign(x_1)e_1
\Rightarrow Q_vx \in \text{span}\{e_1\}$\\
\newline
Nun Durchführung der QR-Zerlegung nach Householder $A=(a_1,\cdots,a_n); a_1,\cdots,a_n \in \RR^m$\\
\begin{itemize}
	\item[1.]QR-Schritt: Bestimme $v_1 \in\RR^m$ mit $Q_{v_1}\cdot a_1 = r_{11}\cdot e_1$ für ein $r_{11} \in \RR$\\\newline
	$\Rightarrow Q_{v_1}A=(Q_{v_1}a_1,\cdots,Q_{v_1}a_n) =\left(
	\begin{array}{c|ccc} r_{11} & * & \cdots & *\\ \hline
						 0 & & & \\
						 \vdots & &A^{(2)} &\\
						 0 & & &
	\end{array}\right)\\ Q=Q_{v_1}$
	\item[2.]QR-Schritt: $A^{(2)}= \left(a_1^{(2)},\cdots,a_{n-1}^{(2)}\right)$\\
			Bestimme $v_2^{(2)} \in \RR^{m-1}$ mit $Q_{v_2^{(2)}}a_1^{(2)} = r_{22}e_1\\
			\Rightarrow \text{für } v_2 = \begin{pmatrix}0\\v_2^{(2)}\end{pmatrix}\in \RR^m \text{ gilt } Q_{v_2}Q_{v_1}A=\left(
			\begin{array}{c|ccc} r_{11} & * & \cdots & *\\ \hline
			0 & & & \\
			\vdots & &Q_{v_2^{(2)}}A^{(2)} &\\
			0 & & &
			\end{array}\right) = \left(
			\begin{array}{ccccc} r_{11} & * & \cdots &\cdots & *\\
			0 & r_{22} & * & \cdots & *\\
			\vdots & 0 & &\\
			\vdots & \vdots& &A^{(3)}\\
			0 & 0 & &
			\end{array}\right)$
			$Q=Q\cdot Q_{v_2}$\\
\end{itemize}
Aufwand: Im k-ten Schritt $(k=1, \cdots,n)$
		\begin{itemize}
			\item[-]Bestimmen von $v_k: m-k+2$ Multiplikationen + 1 Wurzel
			\item[-]Berechnen von $Q_{v_k^{(k)}}A^k: (m-k+1)(n-k+1)$ Multiplikationen
			\item[-] Berechnen von $QQ_{v_k}: (m-k+1)n$ Multiplikationen\\
			\item[$\Rightarrow$]$\sum O(mn^2) \text{ Multiplikationen } + O(n) Wurzeln$
		\end{itemize}\hfill\\

\begin{Bemerkung}
	Vorherige Schrittfolge erfordert, dass $a_{11}^{(k)} \neq 0 \;\forall k=1,\cdots,n$.
	Dies kann mit Pivotisierung umgangen werden.
\end{Bemerkung}\hfill
\begin{weitere Bemerkungen}\hfill
	\begin{itemize}
		\item[a)]QR-Zerlegung gehört zu \glqq stabilen Algorithmus\grqq, d.h. es gibt keine zusätzliche Fehlerverstärkung.
Beweis: Für $A \in \Gl_n(\RR); \; A = Q \cdot R$ gilt:
\begin{align*}
  \condb(A) &= \sqrt{\frac{\lambda_{max}(A^T \cdot A)}{\lambda_{min}(A^T \cdot A)}} =
\sqrt{\frac{\lambda_{max}(R^T \cdot Q^T \cdot Q \cdot R)}{\lambda_{min}(R^T \cdot Q^T \cdot Q \cdot R)}} \\
&= \sqrt{\frac{\lambda_{max}(R^T \cdot R)}{\lambda_{min}(R^T \cdot R)}} = \condb(R)
\end{align*}
		\item[b)]$R^T$ ist \glqq Cholesky-artiger\grqq Faktor von $A^TA$ da $R^TR=A^TA$ (vgl. A)\\ i.A. $R^T \neq L$, jedoch $R^T=DL$ für $D=(sign(r_{11}),\cdots,sign(r_{nn}))$
		\item[c)]Ausblick: lineare Ausgleichsrechnung (später mehr)\\
		$A \in\RR^{m\times n}, rang(A)=n$\\
		Problem: Finde $\hat{x} \in \RR^n$ s.d. $\|A\hat{x}-b\|_2=\underset{x\in\RR^n}{min}\|Ax-b\|_2$\\
		Mit $A=QR$ und $\|Qy\|_2=\|y\|_2 \;\forall < \in \RR^n \text{ gilt }\\
      \|Ax-b\|_2^2=\|Q^T(Ax-b)\|_2^2=\left\|\begin{pmatrix}R_1\\0\end{pmatrix}x-\begin{pmatrix}c\\d\end{pmatrix}\right\|_2^2 = \|R_1x-c\|_2^2 + \|d\|^2$. Dieser Ausdruck ist minimal, wenn $R_1x=c$
	\end{itemize}
\end{weitere Bemerkungen}

\subsection{Lineare Ausgleichsprobleme, überbestimmte Gleichungssysteme}
Geg: $A \in \RR^{m\times n}, b\in \RR^n$\\
Oftmals: $\rang(A,b)>\rang(A) \rightsquigarrow \text{Lös}(A,b) = \emptyset$\\

Lineares Ausgleichsproblem: Finde
\begin{align*}
\tag{LA}
\hat{x} \in \RR \text{ mit } \|A\hat{x}-b\|_2 = \underset{x \in \RR^n}{\mathrm{min}}\|Ax-b\|_2
\end{align*}

\subsubsection{Exkurs: Approximation in Skalarprodukträumen}
\begin{Definition}
$L \subset \RR^n$ heißt \glqq affiner Unterraum\grqq wenn $L=v+U$, wobei $v \in \RR^n$, U Untervektorraum (UVR)
\end{Definition}
\begin{Beispiel}\hfill
	\begin{itemize}
	\item[a)] UVR ist affiner Unterraum
	\item[b)] Lös$(A,b)$ ist affiner Unterraum falls nicht leer
	\end{itemize}
\end{Beispiel}
\textbf{Problem:}\\
Gegeben: \begin{itemize}
	 \item[a)] affiner UR $L=v+U$
	 \item[b)] $b \in \RR^n$
	 \end{itemize}
Finde
\begin{align*}
\tag{MIN}
\hat{y} \in L \text{ so dass } \|\hat{y}-b\|_2 = \underset{y \in L}{min} \|y-b\|_2
\end{align*}

\begin{Satz}
Für einen affinen Unterraum $L=v+U\quad(v\in \RR^n, U\subset \RR^n UVR)$ und ein $b\in \RR^n$ gilt:\\
\begin{itemize}
\item[a)]$\hat{y} \in L$ erfüllt (MIN) $\Leftrightarrow \hat{y}-b\in U^{\perp}$
\item[b)]$\hat{y}$ mit (MIN) ist eindeutig bestimmt
\end{itemize}
\end{Satz}
\begin{proof}\hfill
\begin{itemize}
\item[a)] Sei $y \in L \Rightarrow y=v+u, u\in U$\\Wegen $\RR^n=U \oplus U^{\perp} \text{ gilt } b-v=u_1+U_2, \text{ wobei } u_1\in U, u_2 \in U^{\perp}$\\
$\|b-y\|_2^{2} =\|(b-v)-(y-v)\|_2^{2}=\|u_1+u_2-u\|_2^{2}=\|-(u+u_1)+u_2\|_2^{2} \overset{Pythagoras}{=}\|u_1-u\|_2^{2} + \|u_2\|^{2}$\\
Also gilt: $\|b-y\|_2^{2}$ minimal $\Leftrightarrow u_1=u \Leftrightarrow b-y = u_2 \in U^{\perp}$\\
\item[b)] Sei $\hat{y}$ Lösung von (MIN). Sei $y_2 \in L \text{ mit }\|\hat{y}-b\|_2 =\|y_2-b_2\| \overset{a)}{\Rightarrow} \hat{y}-b,y_2-b \in U^{\perp} \Rightarrow \hat{y}-y_2 \in U^{\perp}$.
Andererseits gilt $\hat{y}-y_2 \in U \Rightarrow \hat{y}-y_2=0$
\end{itemize}
\end{proof}
\begin{Satz}
$A \in \RR^{m \times n}$. Dann gilt:\\\begin{itemize}
\item[a)]$(\ker A)^{\perp} = \im A$ \quad \text{(Übungsaufgabe3)}
\item[b)]$A^TA$\; \text{ symmetrisch }
\item[c)]$\ker A^TA = \ker A$
\item[d)]$\im A^TA = \im A^T$
\item[e)]$A^TA \ge 0 \text{ mit } (A^TA>0 \Leftrightarrow ker A = {0})$
\item[f)] $\forall \lambda \in \RR\backslash\{0\}: \dim \ker(\lambda I-A^TA) = \dim \ker(\lambda I-AA^T)$
\end{itemize}
\end{Satz}
\begin{proof}\hfill
\begin{itemize}
\item[c)]$"\supset"$ \; \text{klar}\\
$" \subset" $ Sei $x \in \ker A^TA \Rightarrow 0=x^TA^TAx=\|Ax\|_2^{2} \Rightarrow Ax=0 \Rightarrow \in \ker A$
\item[d)]$\im AA^T \overset{a)}{=} (\ker(A^TA)^T)^{\perp} = (ker A^TA)^{\perp} \overset{c)}{=} (\ker A)^{\perp} \overset{a)}{=} im A^T$
\item[e)]$A^TA\ge 0  \; \text{klar}$\\$" \Rightarrow"\text{ Sei } \ker A^TA =\{0\} \overset{c)}{\Rightarrow}ker\{0\}$\\$"\Leftarrow" \text{ Sei } \ker A=\{0\}, x\in \RR^n\backslash\{0\} \Rightarrow x^TA^TAx = \|Ax\|_2^{2} > 0$
\item[f)] folgt aus ÜA 3
\end{itemize}
\end{proof}
\subsubsection{Zurück zum linearen Ausgleichsproblem}
\begin{align*}
&A \in \RR^{m \times n}, b \in \RR^n\\
\tag{LA}
&\text{Finde } \hat{x} \in \RR^n s.d. \|A\hat{x}-b\|_2 = \underset{x \in \RR}{\mathrm{min}}\|Ax-b\|_2\\
&\Leftrightarrow \hat{y}-b \in (\im A)^{\perp} = \ker A^T\\
&\Leftrightarrow A^T(\hat{y}-b)=0 \Leftrightarrow A^T(Ax-b)=0\\
&\Leftrightarrow A^TAx=A^Tb \quad \text{(NGL:  "`Normalengleichung"')}
\end{align*}
\begin{Bemerkung}
$A^TAx=A^Tb$ ist immer lösbar, da $\im A^TA=\im A^T$
\end{Bemerkung}
\begin{Satz}
$A \in \RR^{m \times n}, b \in \RR^m.$ Dann gilt:
\begin{itemize}
\item[a)]$\|A\hat{x}-b\|_2= \underset{x \in \RR^n}{\mathrm{min}}\|Ax-b\|_2 \Leftrightarrow A^TAx=A^Tb$ (NGL)
\item[b)] $\mathrm{NGL\;ist\;loesbar\;} (\Loes(A^TA,A^Tb) = \emptyset$)
\item[c)] $\mathrm{Es\;gibt\;ein\;eindeutiges\;} \hat{x_{*}} \in \RR^n$ mit $ \|\hat{x_{*}}\|_2 = \underset{\hat{x} \text{ erfüllt NGL }}{\mathrm{min}}\|\hat{x}\|$
\end{itemize}
\end{Satz}
\begin{proof}
	\begin{itemize}
		\item[a) und b)] folgt aus vorigen Überlegungen
		\item[c)] Es gilt $L=\{\hat{x}\in \RR^n:A^TA\hat{x}=A^Tb\}$ ist affiner UR, Rest folgt dann aus 2.9.1
	\end{itemize}
\end{proof}
\begin{Bemerkung}
	(LA) wird auch häufig als Lösung "Lösung im kleinste-Quadrat-Sinne"
	\textit{(least squares solution)} bezeichnet.
\end{Bemerkung}
\begin{Beispiel}
	Gesucht ist eine Gerade $y=\alpha x+\beta$ durch (0,1),(1,1),(2,1),(5,1)\\
	$\rightsquigarrow$ gibts nicht\\
	Suche stattdessen eine Gerade, die die Summe der Quadrate der Abstände zu den Punkten minimiert.\\
	Minimiere \begin{align*}
	&\left\|\underbrace{\begin{pmatrix}
	0 & 1\\1 & 1\\2 & 1\\ 5  & 1
	\end{pmatrix}}_{=A}\underbrace{\begin{pmatrix}
	\alpha\\\beta
	\end{pmatrix}}_{=x}-\underbrace{\begin{pmatrix}
	2\\1\\1\\1
	\end{pmatrix}}_{=b}\right\|_2\\\\
	&\overset{NGL}{\Rightarrow} \begin{pmatrix}
	30 & 8\\8 & 4
	\end{pmatrix}\begin{pmatrix}
	\alpha\\\beta
	\end{pmatrix}=\begin{pmatrix}
	8\\5
	\end{pmatrix}\quad
	Lsg.\begin{pmatrix}
	\frac{-1}{7}\\\frac{43}{28}
	\end{pmatrix}=\begin{pmatrix}
	\alpha\\\beta
	\end{pmatrix}
	\end{align*}
\end{Beispiel}
\begin{Bemerkung}
	Wenn $\rang A = n$, dann kann (LA) ohne (NGL) mit QR-Zerlegung gelöst werden
\end{Bemerkung}

\subsubsection{Singulärwertzerlegung}
\begin{Satz}
	Sei $A \in \RR^{m \times n}, r=rang(A)$. Dann existieren $U \in O_m(\RR), V \in O_n(\RR) \text{ sowie } \sigma_1 \ge \ldots \ge \sigma_r>0$, sodass für $\sum=\diag(\sigma_1,\ldots,\sigma_r)$ gilt $U^*AV=\begin{pmatrix}
	\sum & 0_{r,(n-1)}\\
	0_{m-r,r} & 0_{m-r,n-r}
	\end{pmatrix}$\\\glqq Singulärwertzerlegung"\\
	Die Singulärwerte $\sigma_1,\ldots,\sigma_r$ sind durch $A$ eindeutig bestimmt.
\end{Satz}
Vorbemerkungen zum Beweis:\\
\begin{itemize}
	\item[(i)]$\sigma(A^TA)\backslash\{0\} = \sigma(AA^T)\backslash\{0\}\\ \text{dim ker}(\lambda I-A^TA)=\text{dim ker}(\lambda I-AA^T)$
	\item[(ii)]$\rang(A^TA) = \rang(AA^T)=r\\ \sigma(A^TA) \subset [0,\infty)$
\end{itemize}
\begin{proof}
	Seien $\lambda_1 \ge \ldots \ge \lambda_r\ge\lambda_{r+1}=\ldots=\lambda_n=0$ die Eigenwerte von $A^TA$ (inkl. Vielfache)\\
	$v_1,\ldots,v_n$ sei ONS im $R^n$ mit $A^TAv_i=\lambda_iv_i \quad i=1,\ldots,n$\\
	Setze $\sigma_i =\sqrt{\lambda_i} \quad i=1,\ldots,r \text{ und } u_i=\frac{1}{\sigma_i}Av_i$\\
	\begin{itemize}
		\item[a)]Für $k=1,\ldots,r$ gilt $AA^Tu_i=\frac{1}{\sigma_i}A^TAv_i=\frac{1}{\sigma_i}A(\lambda_iv_i)=\lambda_i\left(\frac{1}{\sigma_i}Av_i\right)=\lambda_iu_i$
		\item[b)]Für $j,k=1,\ldots,r$ gilt $u_j^{T}u_k=\frac{1}{\sigma_j\sigma_k}v_j^{T}A^TAv_k=\frac{\lambda_k}{\sigma_j\sigma_k}v_j^{T}v_k=\delta_{j,k}$
	\end{itemize}
	$\Rightarrow \{u_1,\ldots,u_r\}$ ist ein ONS in $\RR^m$ bestehend aus EV von $AA^T$.\\\newline
	Ergänze zu ONB $\{u_1,\ldots,u_m\}$. Dann gilt $A^Tu_{r+1}=\ldots=A^Tu_m=0$.\\
	Betrachte $U=(u_1,\ldots,u_m) \in O_m(\RR), V=(v_1,\ldots,v_n) \in O_n(\RR)$\\
	Betrachte $U^TAv$: Es gilt $(U^TAv)_{i,j}=u_i^{T}Av_j$\\\\
	Für $i=r+1,\ldots,m$ gilt $u_i^{T}Av_j=\underset{=0}{(Au_i)^T}v_j=0$\\
	Für $i=1,\ldots,r $ gilt $u_i^{T}Av_j=\frac{1}{\sigma_i}(Av_i)^TAv_j=\frac{1}{\sigma_i}v_i^{T}A^TAv_j=\sigma_i\delta_{ij}$\\\newline
	Also $U^TAV= \left(\begin{array}{c|c}
		\begin{matrix}\sigma_1 & &  \\&\ddots & \\ & & \sigma_r\end{matrix} &\begin{matrix}& &\\ &\bigzero& \\&&\end{matrix}\\\hline
		\bigzero &\begin{matrix}& &\\ &\bigzero& \\&&\end{matrix}
		\end{array}\right)$
\end{proof}

Es gibt also $U \in \U_m(\RR), V \in \U_n(\RR)$ mit
\begin{align*}
  & A = U \left(\begin{array}{c|c} \varSigma & 0  \\ \hline 0 & 0 \end{array} \right) V^T \; \text{ mit } \; \varSigma = \diag(\sigma_1, \cdots , \sigma_r) \text{ und }\\
  & r = \rang(A) \; ; \quad \sigma_1\ge \sigma_2\ge\sigma_1\ge  \cdots , \sigma_r > 0
\end{align*}
Noch zu zeigen: Die Singulärwerte sind eindeutig bestimmt durch $A$.
(Achtung! $U$ und $V$ sind nicht eindeutig., Das gitl nurr wenn die $\sigma_i$
paarweise verscheiden sind.)

Beweis: Sei
\begin{align*}
  & U_1 \left(\begin{array}{c|c} \varSigma_1 & 0  \\ \hline 0 & 0 \end{array} \right) V_1^T =
A = U_2 \left(\begin{array}{c|c} \varSigma_2 & 0  \\ \hline 0 & 0 \end{array} \right) V_2^T \\
& \Rightarrow A^T A = V_1^T \left(\begin{array}{c|c} \varSigma_1^2 & 0  \\ \hline 0 & 0 \end{array} \right) V_1 = V_2^T \left(\begin{array}{c|c} \varSigma_2^2 & 0  \\ \hline 0 & 0 \end{array} \right) V_2 \\
& \Rightarrow \; \text{Die Quadrate der Singulärwerte sind Eigenwerte von } A^T A\\
& \Rightarrow \; \varSigma_1^2 = \varSigma_1^2  \quad \text{Die Eigenwerte sind der Größe nach sortiert}\\
& \Rightarrow \;  \varSigma_1 = \varSigma_1 \quad \text{weil A positiv definit ist}
\end{align*}
Die Singulärwertzerlegung (SVD = single value decomposition) findet ihre Anwendung im
linearen Ausgleichsproblem (LA): Gesucht ist ein $\hat{x}\in \RR^n$ so dass
\begin{align*}
\tag{LA}
\|A\hat{x}-b\|_2 & = \min_{x \in \RR^n}\|Ax-b\|_2
\intertext{Mit geeigneten Abbildungen $U, V \in \U^n$ gilt dann, da $U, V$ linientreu abbilden:}
\|A\hat{x}-b\|_2^2 & = \| U^T A x - U^T b\|_2^2 =  \| U^T A V (V^T x) - U^T b\|_2^2
= \left\| \begin{pmatrix}\varSigma & 0 \\ 0 & 0 \end{pmatrix} V^T x - U^T b \right\|_2^2\\
& = \sum_{i = 1}^r \left(\sigma_i(V^T x)_i - (U^T b_i)\right)^2 + \sum_{i = r +1}^n \left(U^T b_i\right)^2\\
&= \sum_{i = 1}^r \left(\sigma_i v_i^T x_i - U^T b \right)^2 +  \sum_{i = r +1}^n \left(U^T b_i\right)^2
\end{align*}
Der Ausruck wird minimal wenn der erste Summand verschwindet. Folgerungen:
\begin{itemize}
  \item[a)]
  \begin{align*}
    & \hat{x} \text{ löst das lineare Ausgleichsproblem}\\
    & \Leftrightarrow \quad v_i^T \hat{x} = \frac{1}{\sigma_i} U^T b \quad i = 1, \cdots , r\\
    & \Leftrightarrow \quad V^T \hat{x} =
    \begin{pmatrix}
    \frac{1}{\sigma_1} u_1^T b\\ \vdots \\ \frac{1}{\sigma_r} u_r^T b \\
    \alpha_{r + 1} \\ \vdots \\ \alpha_n
    \end{pmatrix} \quad \text{für $\alpha_{r + 1}, \cdots , \alpha_n \in \RR$ beliebig}\\
    & \Leftrightarrow \quad \hat{x} = V \cdot
    \begin{pmatrix}
    \frac{1}{\sigma_1} u_1^T b\\ \vdots \\ \frac{1}{\sigma_r} u_r^T b \\
    \alpha_{r + 1} \\ \vdots \\ \alpha_n
    \end{pmatrix}
  \end{align*}
  \item[b)]
  \begin{align*}
    & \|\hat{x}\|_2^2 = \sum_{i = 1}^r \left(\frac{1}{\sigma_i} u_i^T b \right)^2 + \sum_{i = r + 1}^n \alpha_i^2
    \intertext{Daraus folgt: Die eindeutig bestimmte Lösung mit minimaler Norm ist}
    &\hat{x} = V \cdot
    \begin{pmatrix}
    \frac{1}{\sigma_1} u_1^T b\\ \vdots \\ \frac{1}{\sigma_r} u_r^T b \\
    0 \\ \vdots \\ 0
    \end{pmatrix} = V  \begin{pmatrix}\varSigma^{-1} & 0 \\ 0 & 0 \end{pmatrix} U^T b = A^+ b
  \end{align*}
\end{itemize}
Dabei ist $A^+$ die Pseudoinverse von $A$, siehe nächster Abschnitt.


\subsection{Pseudoinverse (Moore-Penrose-Inverse)}

Hier mache ich weiter: Michael

\subsection{Iterations-Verfahren}


TODO

TODO: Fortsetzung Iterations-Verfahren

\subsubsection{Gesamt- und Einzelschrittverfahren}

TODO:

TODO: Beweis Konvergenz EV und GV

\subsection{Abstiegsverfahren}

TODO

\subsubsection{Gradientenverfahren}

TODO

\subsubsection{Conjugate-Gradient-Verfahren}
\begin{Lemma}
$A \in \KK^{n\times n}$, $A > 0$. Dann ist die Abbildung $$(x,y) \mapsto x^* A y$$
ein Skalarprodukt im $\KK^n$.
\end{Lemma}
\begin{proof}
Nachrechnen.
\end{proof}
Wir schreiben $\langle x, y \rangle_A := x^*Ay$ und
${\|x\|}_A := \sqrt{\langle x, y \rangle_A}$.
${\|\cdot\|}_A$ ist eine Norm.

\begin{Algorithmus}[CG-Verfahren] \quad\\\
Eingabe: $A \in \KK^{n\times n}$, $A > 0$, $b \in \KK^n$, 
  Startwert $x^0 \in \KK^n$.\\
Ausgabe: Approximation von $x = A^{-1}b$.
\begin{codebox}
\Procname{$\proc{CG-Verfahren}$}
\li $r^0 \gets b - Ax^0$
\li $d^0 \gets r^0$
\li $k \gets 0$
\li \Repeat
    \li $\alpha^k \gets (r^{k\,\T} r^k) / (d^{k\,\T} A d^k)$
    \li $x^{k+1} \gets x^k + \alpha^k d^k$
    \li $r^{k+1} \gets b - A x^{k+1}$
    %\li \If $\|r^{k+1}\| < \varepsilon$
    %  \Then
    %  \li \Return $x^{k+1}$
    %\End
    \li $\gamma^k \gets (r^{k+1\,\T} A d^k) / (d^{k\,\T} A d^k)$
    \li $d^{k+1} \gets r^{k+1} + \gamma^k d^k$
    \li $k \gets k + 1$
  \End
\end{codebox}
\end{Algorithmus}

TODO Fortsetzung cg-Verfahren und Beweis

\section{Nichtlineare Gleichungen}

\subsection{Problemstellung, Erinnerung}

TODO

TODO Fortsetzung Erinnerungen, Mittelwertsatz im $\RR^n$

\subsection{Newton-Verfahren}

TODO

TODO: Fortsetzung Newton-Verfahren, Konvergenzanalyse

\subsection{Sekantenverfahren}

TODO
TODO: Fortsetzung Sekantenverfahren (Beweis)

\subsection{Interpolation}

TODO



\end{document}
