\documentclass[a4paper]{scrartcl}

\usepackage[utf8]{inputenc} \usepackage[ngerman]{babel}

\usepackage{amssymb} \usepackage{amsmath}
\usepackage{latexsym}
\usepackage{enumerate}

\newcounter{mydefctr}
\newtheorem{mydef}[mydefctr]{Definition}
\newtheorem{vmydef}[mydefctr]{(vage) Definition}

\newcommand{\rang}{\mathop{\mathrm{rang}}}
\newcommand{\loes}{\mathop{\mathrm{L\ddot{o}s}}}
\newcommand{\diag}{\mathop{\mathrm{diag}}}

\parindent 0pt
\parskip 0.25\baselineskip

\title{Numerik-Skript Teil 1}

\author{Laura Schüttpelz}

\begin{document}

\maketitle

\section{Einleitung}
Was ist numerische Mathematik?
$\rightarrow$ Bereitstellung konstruktiver Verfahren zur 
approximativen Berechnung mathematischer Sachverhalte.

\subsubsection*{Bsp: Lineares Gleichungssystem (LGS) Ax = b}
\textbf{gegeben:} $A \in \mathbb{R}^{n \times n}$ invertierbar 
$(A \in GI_{\sim} (\mathbb{R})), b \in \mathbb{R}^n$ \\
\textbf{gesucht:} $x \in \mathbb{R}^n$, so dass $Ax = b$ (also $x = A^{-1} b$)

"`Naive Möglichkeit"': Lösung über Cramersche Regel

\begin{displaymath}
x = \begin{pmatrix} x_a \\ \vdots \\ x_n \end{pmatrix}, 
~ x_i = \frac{1}{\det (A)},\\~ \det(A_i)~ mit~ 
A_i = \begin{pmatrix} a_{11} & \dots & b_1 & \dots & a_{n1} 
\\ \vdots & & \vdots & & \vdots \\ a_{n1} & \dots & b_n & 
\dots & a_{nn} \end{pmatrix}
\end{displaymath}

Leibnitz-Darstellung von $\det(A)$:
\begin{displaymath}
\det(A) = \sum_{\sigma \in S_n} sign (\sigma) a_{1\sigma(1)} 
\dots a_{n\sigma(n)}
\end{displaymath}

\subsubsection*{Fakten:}
\begin{enumerate}[(a)]
\item\label{leibnitz:multi}
Leibnitz-Darstellung von $\det(A)$ benötigt $n! \cdot (n-1)$ Multiplikationen
\item
Aus (\ref{leibnitz:multi}) folgt: Lösungen von $Ax=b$ mit Cramerschen Regel benötigt \\
$2 \cdot  n!\cdot(n-1) \cdot n + n$ Multiplikationen 
\end{enumerate}

Rechenleistung einer moderaten Maschine Intel Core i7: $3.3 \cdot 10^{10} 
\frac{\text{Multiplikationen}}{\text{Sekunde}}$ 

\begin{description}
\item[\rm Fall $n = 20$:] $\frac{2 \cdot 20! \cdot 19 + 20}{3.3 \cdot 10^{10} 
\cdot 60 \cdot 60 \cdot 24 \cdot 365} = 1776.3$ Jahre
\item[$n = 15$:] 4.6 Std
\item[$n = 10$:] 0.2 Sek.
\item[$n = 25$:] 17.8 Mrd. Jahre
\end{description}

Mithilfe anderer "`geschickter Verfahren"' kann man solche LSG 
(heutzutage bis zu $10^9$) in vernünftiger Zeit lösen.

\begin{vmydef}
Sei ein numerisches Verfahren abhängig von den 
Größen $n_1, \dots, n_l \in \mathbb{N}$. Man spricht vom 
"`Aufwand $O(f(n_1, \dots, n_l))$ Multiplikationen"'
(für $f: \mathbb{N}^l \rightarrow \mathbb{N}$) wenn es 
$c_1, c_2 > 0$ gibt, sodass das Verfahren für alle 
$n_1, \dots, n_l > c_1$ mit maximal $c_2 \cdot f(n_1, \dots n_l)$
Multiplikationen auskommt (Divisionen zählen als Multiplikationen, 
Additionen zählen gar nicht).
\end{vmydef}

\subsubsection*{Beispiele:}

\begin{enumerate}[(a)]
\item Cramersche Regel zur Lösung von
      $Ax = b$ mit $A \in GI_{\sim}(\mathbb{R}), 
      b \in \mathbb{R}^n$ : \\
      O($n! \cdot (n-1)\cdot n$) Multiplikationen
\item Euklidisches Skalarprodukt:
      $\langle x,y\rangle = \sum_{k=1}^n x_k y_k$, $x,y \in \mathbb{R}^n$ :\\
      O(n) Multiplikationen
\item Matrixmultiplikation: $A \cdot B \text{ für } A\in 
      \mathbb{R}^{n \times m}, B \in \mathbb{R}^{m \times k}$ : \\
      $O(n \cdot m \cdot k ) $ Multiplikationen
\end{enumerate}  

\subsubsection*{Weitere Fragestellungen der Numerik:}

\begin{enumerate}[(a)]
\item
Numerische Integration, Berechne $\int_a^b f(x) dx$ numerisch
\item
Lösung nichtliearer Gleichungssysteme \\
$F(x) = 0$, $F: U \subset \mathbb{R}^n \rightarrow \mathbb{R}^k$ gegeben \\
Bsp.:
\begin{enumerate}[(i)]
\item
$f(x)=e^x-y$, $y \in \mathbb{R}$ gegeben. \\
$\rightarrow$ gesch! Lösung $x = \log y$ 
\item
$f(x)=x \cdot e^x -1 = 0$ nicht gesch! lösbar. \\
Lösung existiert jedoch in $(0,1)$, da $f(0) = -1 < 0$, \\
$f(1)= e-1>0 $ (Rest folgt aus Zwischenwertsatz)
\end{enumerate}
\item
Interpolation \\
Gegeben: $(x_1,y_1), \dots (x_n,y_n) \in \mathbb{R}^2$\\
Finde $f: I \subset \mathbb{R} \rightarrow \mathbb{R}$, so dass 
$f(x_1) = y_1, \dots , f(x_n)=y_n$
\item
numerische Berechnung von Eigenwerten
\item
numerische Lösung gew. Differentialgleichungen $x(t)=f(t,x(t))$
\item
lineare Optimierung
\end{enumerate}

\section{Lineare Gleichungssysteme}
\subsection{Grundlagen}
Notation: $\mathbb{K} \in \{\mathbb{R}, \mathbb{C}\}$ \\

\begin{align*}
x &= \begin{pmatrix} x_1 \\ \vdots \\ x_n \end{pmatrix} 
\in \mathbb{K}^n \widehat{=} \mathbb{K}^{n \times 1},  
~ A=(a_{ij}) \in \mathbb{K}^{m \times n}  \\
x^T &= \begin{pmatrix} x_1 & \dots & x_n \end{pmatrix} 
\in \mathbb{K}^{1 \times n}, ~A^T = (a_{ij}) \in \mathbb{K}^{n \times n}  \\ 
x^{\ast} &= \begin{pmatrix}\overline{x_1} & \dots & \overline{x_n} \end{pmatrix}, 
~A^{\ast} = (\overline{a_{ij}}) \in \mathbb{K}^{n \times m}  \\
GI_{\sim} (\mathbb{k}) &= \{A \in \mathbb{K}^{n \times n}: A 
\text{ invertierbar} \}  \\
O_n (\mathbb{K})&= \{ U \subset \mathbb{K}^{n \times n}: 
U^{\ast} U = I \} , \text{ $I$ ist Einheitsmatrix.} \\
e_i &= \begin{pmatrix} 0 \\ \vdots \\ 1 \\ \vdots \\ 0 \end{pmatrix} 
\text{: $i$-ter kanonischer Einheitsvektor } \\
\end{align*}

\subsubsection*{Lineares Gleichungssystem (LGS): Ax=b}


\textbf{gegeben:} $A \in \mathbb{K}^{m \times n}, b \in \mathbb{K}^m$

\textbf{gesucht:} Lösungsmenge $\loes(A,b) = \{x \in \mathbb{K}^n : Ax=b\} $


Def.: $r:=\rang(A), ~ s := \rang(A,b)$
\begin{align*}
\text{Fälle: } b =0 : r = n &\Rightarrow \loes (A,b)= \ker(A)=\{0\}\\
r<n &\Rightarrow \loes (A,b) \text{ ist $n-r$ dimensional, Unterraum 
von } \mathbb{K}^n \\
b \neq 0: r<s &\Rightarrow \loes (A,b) = \emptyset \\
r=s=n &\Rightarrow \loes (A,b) \text{ einelementig }\\
r=s<n &\Rightarrow \loes (A,b) = x + \ker(A) () \loes  
\text{ (mit $n-r$ Parametern), $n-r$-dim. } \\
\text{Speziell: } n=m: b=0 &\Rightarrow r=n :  \loes (A,b) = \{0\}\\
&\Rightarrow r<n : \loes (A,b) = \ker(A)\\
b \neq 0: r=n &\Rightarrow \loes (A,b) = \{A^{-1}b\}\\ 
	r<n, r<s &\Rightarrow \loes (A,b) = \emptyset \\
	 r=s &\Rightarrow \text{Lösung mit $n-r$ Parametern}
\end{align*} 

\paragraph{Bekannte Lösungsmethode:} Gaußsches Eliminationsverfahren

$Ax=b \rightarrow \tilde{A}x= \tilde{b} \\$

$(\tilde{A},\tilde{b})=
\begin{pmatrix} 
\widetilde{a_{11}} & &\dots  &  & \widetilde{a_{1n}} & \widetilde{b_1} \\
0 & & \ddots &  & \vdots & \vdots \\
0 & \dots & a_{rr} & \dots & a_{rn} & \widetilde{b_r} \\
0 & & \dots & & 0 & \widetilde{b_n} \end{pmatrix} $
mit $\widetilde{a_{11}}, \dots, a_{rr} \in \mathbb{K} \backslash \{0\} $

Es gilt $\rang(A)= \rang (A,b) \Leftrightarrow 
\widetilde{b_{r+1}} = \dots = \widetilde{b_n} = 0 $

\subsubsection*{Grundlagen der linearen Algebra:}

\begin{enumerate}[(i)]
\item
\[
\lambda \in \mathbb{C}\text{ heißt "`Eigenwert von } A \in 
\mathbb{K}^{n \times n} \text{ "'.} \]
\[
: \Leftrightarrow \exists x \in \mathbb{C}^n \backslash \{0\}: Ax = \lambda x \\
\]\[
\Leftrightarrow \ker (\lambda I -A) \neq \{0\} \Leftrightarrow 
\det (\lambda I -A) =0, \ker (\lambda I-A) = Eig (A, \lambda)
\]\[
A \in \mathbb{K}^{n \times n} \text{ besitzt Eigenwerte (Korollar des 
Fundamentalsatzes der Algebra)}
\]\[
\sigma (A) = \{ \lambda \ in~ \mathbb{C}: \lambda \text{ ist Eigenwert 
von A}\} \text{ "`Spektrum von A"' }
\]
\item
\[
A \in \mathbb{K}^{n \times n} \text{ Hermitesch (dh. } A = A^\ast \text{) } a_{ij} = \overline{a_{ji}}
\]
Dann gilt 
\begin{enumerate}[(a)]
\item
$ \sigma(A) \in \mathbb{R}$
\item
$\exists$ Orthonormalbasis (ONB) aus Eigenvektoren von A. D.h. $\exists u_1, \dots, u_n \in \mathbb{K}^n \text{ mit } u_i^\ast u_j = \delta_{ij}$
$\land A~u_i = \lambda_i~u_i$ für ein $\lambda_i \in \sigma(A) 
(\forall i \in \{1, \dots, n\})$  
\end{enumerate}
$\Leftrightarrow$ Für $U = [u_1, \dots, u_n] \in \mathbb{K}^{n\times n}$ gilt $U^\ast U \in I$
(dh. $U \in \sigma_n(\mathbb{K})$) und 
$U^\ast AU = \diag(\lambda_1, \dots, \lambda_n)$  

\item
	$P \in \mathbb{R}^{n \times n}$ heißt "`Permutationsmatrix"' , falls $P=(e_{\sigma(1)} \dots e_{\sigma(n)})$ für ein $\sigma \in S_n$
	Dann 
	\begin{enumerate}[(a)]
	\item Für $A = (a_1, \dots, a_n) \in \mathbb{R}^{m \times n}$ gilt
	$AP = \begin{pmatrix} a_{\sigma(1)} & \dots & a_{\sigma(n)}\end{pmatrix}$
	\item
	Für $A = \begin{pmatrix} a_1 \\ \vdots \\ a_m \end{pmatrix}, P = (e_{\sigma(1)} \dots e_{\sigma(m)})$
	$\Rightarrow P^T A = \begin{pmatrix} a_{\sigma(1)} \\ \cdots \\ a_{\sigma(n)}  \end{pmatrix}$
\end{enumerate}
\end{enumerate}

\end{document}

