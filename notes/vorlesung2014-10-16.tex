\subsubsection{Definitheit von Matrizen}
Im Folgenden sei $A$ eine hermitesche Matrix.
\footnote{Das $A_i$ in diesem Abschnitt ist nicht zu verwechseln mit der Matrix $A_i$
in der Einleitung, Formel (\ref{cramer})}
\begin{enumerate}
  \item [a)] $A$ heißt "`positiv definit"' (Schreibweise: $A > 0$) wenn
  \begin{align*}
    &\text{\textbullet} \quad \forall x \in \KK^n \setminus \{0\}: x^* A x > 0 \\
    \Leftrightarrow \quad &\text{\textbullet} \quad \det(A_i) > 0 \; \forall \; i = 1, \dots , n \\
    \intertext{\qquad \qquad \qquad dabei ist $A_i$ der $i$-te Hauptminor
    von $A$, d.h.
    die $i\times i$ Teilmatrix}
    & \quad \quad A_i =
    \begin{pmatrix}
      a_{11} & \cdots  & a_{1i} \\
      \vdots & \ddots & \vdots \\
      a_{i1} & \cdots & a_{ii}
    \end{pmatrix} \\
    \Leftrightarrow \quad &\text{\textbullet} \quad  \sigma (A) \subset (0, \infty)
  \end{align*}
  \item[b)]$A$ heißt "`positiv semidefinit"' (Schreibweise: $A \ge 0$) wenn
    \begin{align*}
    &\text{\textbullet} \quad \forall x \in \KK \setminus \{0\}: x^* A x \ge 0 \\
    \Leftrightarrow \quad &\text{\textbullet} \quad  \sigma (A) \subset [0, \infty)
  \end{align*}
  \item[c)] $A$ heißt "`negativ (semi-) definit"': Analog.
\end{enumerate}

\subsubsection{Wurzel einer Matrix}
Sei $A \in \KK^{n \times n}$ ; $A \ge 0$ (das beinhaltet $A = A^*$). Dann ist
$\sqrt{A} = A^{\frac{1}{2}}$ definiert als eine Matrix $\sqrt{A} \ge 0$ mit
$\left(\sqrt{A} \right)^2 = A$.

\begin{proof}[Existenzbeweis]
$A \ge 0 \Rightarrow \exists \; U \in \U_n(\KK)$ ($\U_n = $ Menge
der unitären Matrizen), so dass  \newline
$A = U^* \cdot \diag(\lambda_1, \cdots , \lambda_n) \cdot U$ mit den Eigenwerten
$\lambda_1, \cdots , \lambda_n \in [0, \infty)$. \newline
Betrachte nun $\sqrt{A} = U^* \cdot \diag(\sqrt{\lambda_1}, \cdots , \sqrt{\lambda_n}) \cdot U$.
\end{proof}

\subsection{Vektor- und Matrixnormen}
\begin{Definition} [Norm]
  Sei $X$ ein $\KK$-Vektrorraum. Eine Abbildung $\| \cdot \| \rightarrow [0, \infty)$
  heißt Norm wenn
  \begin{enumerate}
    \item[(i)]    $\|x\| > 0 \quad \forall \; x \in X \setminus \lbrace 0 \rbrace$ (Positivität)
    \item[(ii)]   $\| \lambda \cdot x\| = |\lambda| \cdot \|x\| \quad \forall \; \lambda \in \KK, \forall x \in X$ (absolute Homogenität)
    \item[(iii)]  $\|x + y \| \le \|x\| + \|y\| \quad \forall \; x, y \in X$ (Dreiecksungleichung)
  \end{enumerate}
\end{Definition}

\begin{Bemerkungen}
\quad
  \begin{enumerate}
    \item[a)] Konvergenz $(x_n) \rightarrow x$ bedeutet $\|x_n - x\| \rightarrow 0$.
    \item[b)] Wenn zusätzlich $\dim (X) < \infty$, dann sind alle Normen auf $X$
    äquivalent, d.h.für jede Norm $\| \cdot \|_a$ und $\| \cdot \|_b$ auf $X$ gibt
    es Konstanten $c_1, c_2 > 0$, so dass
    $c_1 \cdot \| x \|_a \le \| x \|_b \le c_2 \cdot \| x \|_a \; \forall x \in X$.
    Es ist klar, dass $\dim(\KK^n) = n < \infty$ und
    $\dim(\KK^{m \times n}) = m \cdot n  < \infty$ .
    \item[c)] Beispiele für Normen auf $\KK^n$:
    \begin{enumerate}
      \item[(i)] $\|x\|_\infty := \max_{i = 1, \cdots , n} |x_i| \quad $Maximums-Norm
      \item[(ii)] $\|x\|_p := \left( \sum_{k = 1}^n |x_k|^p \right)^{1/p} \quad$
      $p$-Norm; $p \in [1, \cdots , \infty)$. Für $p = 2$ Euklidische Norm.
    \end{enumerate}
    Es gilt:
    $\|x\|_\infty \le \|x\|_2 < \|x\|_1 \le \sqrt{n} \cdot \|x\|_2 < n \cdot \|x\|_\infty
    \quad \forall \; x \in \KK^n$
  \end{enumerate}
\end{Bemerkungen}

Beispiele für Matrix-Normen ($A \in X = \KK^{m \times n})$:
\begin{enumerate}
  \item[a)] $\|A\|_1 :=  \max_{j = 1, \cdots , n} \left( \sum_{i = 1}^m |a_{ij}| \right) \quad$ Spaltensummen-Norm
  \item[b)] $\|A\|_\infty :=  \max_{i = 1, \cdots , m} \left( \sum_{j = 1}^n |a_{ij}| \right) \quad$ Zeilensummen-Norm
  \item[c)] $\|A\|_{\text{F}} :=  \left(\sum_{i = 1}^m  \sum_{j = 1}^n |a_{ij}|^2 \right)^{1/2} \quad$
  Frobenius-Norm
\end{enumerate}

\begin{Definition}
Eine Matrix-Norm $\|\cdot\|$ heißt
\begin{enumerate}
  \item[(i)] \textit{sub-multiplikativ}, wenn
  $$\|A \cdot B\| \le \|A\| \cdot \|B\|$$
  für alle $A \in \KK^{m \times k}, \; B \in \KK^{k \times n}$
  \item[(ii)] \textit{verträglich mit der Vektor-Norm $\| \cdot \|_V$}, wenn
  $$\|A \cdot x\| \le \|A\| \cdot \|x\|_V$$
  für alle $A \in \KK^{m \times n}, \; x \in \KK^n$.
\end{enumerate}
\end{Definition}

\begin{Beispiele}
\quad
\begin{enumerate}
  \item[(i)] $$\|A\| := \max_{\substack{i = 1 \cdots m\\j = 1 \cdots n}}|a_{ij}|$$
  ist eine Matrix-Norm, jedoch nicht submultiplikativ, denn für \\
  $A = B = \left( \begin{smallmatrix}1 & 1 \\1 & 1 \end{smallmatrix} \right)$ gilt
  $\|A\| = \|B\| = 1$, aber
  $\|A \cdot B\| = \left \| \left(\begin{smallmatrix}2 & 2 \\2 & 2 \end{smallmatrix}\right) \right\| =
  2 \nleq 1 = \|A\| \cdot \|B\|$.
  \item[(ii)] Die Frobenius-Norm ist mit der euklidischen Norm verträglich:
  \begin{align*}
    & |(A \cdot x)_i|^2 = \sum_{j = 1}^n |a_{ij} \cdot x_j |^2 \overset{(*)}{\le}
    \left( \sum_{j = 1}^n |a_{ij}|^2  \right) \cdot \left( \sum_{i = 1}^n |x_i |^2  \right) =
    \sum_{j = 1}^n |a_{ij}|^2 \cdot |x|^2.
    & \intertext{Bei $(*)$ wurde die Cauchy-Schwarzsche Ungleichung angewendet. Es folgt}
    & \|A \cdot x\|_2 \le \|A\|_{\text{F}} \cdot \|x\|_2 \quad
    \forall \; A \in \KK^{m \times n} \; , \; x \in \KK^n.
  \end{align*}
\end{enumerate}
\end{Beispiele}

\begin{Definition} [induzierte Matrixnorm]
  Sei $\| \cdot \|$ eine Vektornorm auf $\KK^n$, dann ist
  $$\|A\| := \sup_{ x\ne 0} \frac{\| A x \|}{\| x \|}$$ eine Matrixnorm
  auf $\KK^{m \times n}$, die sogenannte
  \emph{von $\| \cdot \|$ induzierte Matrixnorm}.
  Sie wird auch als die \emph{$\| \cdot \|$ zugeordnete Matrixnorm} bezeichnet.
\end{Definition}

\begin{Bemerkungen}
\quad
  \begin{enumerate}
    \item[a)] Die Normeigenschaften lassen sich einfach nachprüfen.
    \item[b)] $$\|A\| := \sup_{ x\ne 0} \frac{\| A x \|}{\| x \|} =
    \sup_{ x\ne 0} \left\| A \frac{x}{\| x \|} \right\| =
    \sup_{ \|x \| = 1} \| A x \|
   \overset{(*)}{=} \max_{ \|x \| = 1} \| A x \|.$$
   $(*)$ gilt, weil die Menge $\{x \in \KK^n : \|x \| = 1\}$ kompakt ist.
  \end{enumerate}
\end{Bemerkungen}

\begin{Satz} Die von $\| \cdot \|$  induzierte Matrixnorm ist
  \begin{enumerate}
    \item [a)] eine Norm
    \item [b)] verträglich mit $\| \cdot \|$
    \item [c)] submultiplikativ.
  \end{enumerate}
\end{Satz}
\begin{proof}
\quad
\begin{enumerate}
  \item [a)] Norm:
    \begin{itemize}
      \item Positivität: $A \ne 0 \; \Rightarrow \; \exists x \text{ so dass }
    A x \ne 0 \; \Rightarrow \; \frac{\| A x \|}{\| x \|} > 0
    \; \Rightarrow \;  \sup_{ x\ne 0} \frac{\| A x \|}{\| x \|} > 0$
      \item Homogenität: $\lambda \in \KK:\; \forall A \in \KK^{m \times n} \;
      \Rightarrow \| \lambda A \| = \sup_{ x\ne 0} \frac{\| \lambda A x \|}{\| x \|} =
      \sup_{ x\ne 0} |\lambda|  \frac{\| A x \|}{\| x \|} =
      |\lambda| \sup_{ x\ne 0} \frac{\| A x \|}{\| x \|}$
      \item Dreiecks-Ungleichung: $\forall A, B  \in \KK^{m \times n} \; \Rightarrow \;
      \sup_{ x\ne 0} \frac{\| (A + B) x \|}{\| x \|} \le
      \sup_{ x\ne 0} \frac{\| A x \| + \| B x \|}{\| x \|} \le
      \sup_{ x\ne 0} \frac{\| A x \|}{\| x \|} + \sup_{ x\ne 0} \frac{\| B x \|}{\| x \|}$
    \end{itemize}
  \item [b)] Verträglichkeit: Sei $A \in \KK^{m \times n} \; , \; x \in \KK$.\\
    1. Fall $x = 0  : \quad \Rightarrow \; \| A x \| = 0 = \|A \| \; \|x\| $\\
    2. Fall $x \ne 0  : \quad \Rightarrow \; \| A x \| =
    \| A \frac{x}{\|x\|}\| \cdot \| x \| \le
    \max_{y = 1} \| A y \| \cdot \| x \| = \| A \| \cdot \| x \|$
  \item [c)] Submultiplikativität: $\forall A, B  \in \KK^{m \times n} \text{ gilt: }
  \sup_{ x\ne 0} \frac{\| A \cdot B x \|}{\| x \|} =
  \sup_{ x\ne 0} \| A \| \frac{\| B x \|}{\| x \|} = \\
  \text{ (wegen Verträglichkeit) } = \| A \|  \sup_{ x\ne 0}  \| \frac{\| B x \|}{\| x \|}
  = \|A\| \cdot \|B\|$
\end{enumerate}
\end{proof}

\begin{Satz}
  Die Spaltensummennorm wird durch $\| \cdot \|_1$ induziert, die Zeilensummennorm
  wird durch $\| \cdot \|_\infty$ induziert.
\end{Satz}
\begin{proof}
  Übungsblatt 2 oder J. Werner, Numerische Mathematik 1, Seite 19.
\end{proof}

\begin{Definition}[Spektralradius]
Sei $A \in \KK^{m \times n}$.
$$\rho(A) := \max |\sigma(A)| =  \max_{\lambda \in \sigma(A)} |\lambda|$$
heißt \emph{Spektralradius von $A$}.
\end{Definition}

\begin{Bemerkung}
Für  $A \in \KK^{m \times n} \; \Rightarrow \; A^* A \ge 0 $ d.h.
$A^* A$ ist positiv semidefinit,
denn $x^* A^* A x = (A x)^* (A x) = \|A x\|_2^2 \ge 0  \; \Rightarrow \;
\sigma(A^* A) \subset [0, \infty)$ d.h. das Spektrum ist nicht negativ.
\end{Bemerkung}

\begin{Satz}
  Die der euklidischen Norm $\| \cdot \|_2$ zugeordnete Matrixnorm ist
  die \emph{Spektralnorm} $$\|A\|_2 = \sqrt{\rho(A^* A)}.$$
\end{Satz}
\begin{proof}
Sei $A^* A = U^* \diag(\lambda_1, \cdots , \lambda_n) U \text{ mit }
U \in \U_n(\KK), \; \lambda_1 \ge  \lambda_2 \ge \cdots \ge \lambda_n \ge 0$. Dann gilt:
\begin{align}
  \nonumber
  \| U x \|_2^2 &= (U x)^* (U x) = x^* U^* U x = x^* x = \| x \|_2^2
  \intertext{Das bedeutet: Unitäre Transformationen erhalten die Länge. Daher gilt für}
  \nonumber
  \| A x \|_2^2 &= (A x)^* (A x) =  x^* A^* A x  =
  (U x)^* \diag(\lambda_1, \cdots , \lambda_n) U x \\
  \nonumber
  &= \sum_{k = 1}^n \lambda_k \|(U x)_k\|_2^2 \le \lambda_1 \sum_{k = 1}^n \|(U x)_k\|_2^2 =
  \lambda_1 \|U x \|_2^2 =  \lambda_1 \|x\|_2^2 = \rho(A^* A) \|x\|_2^2 \\
  \nonumber
  &\Rightarrow \quad \|A x \|_2 \le \rho(A^* A)^{1/2} \cdot \|x\|_2 \\
 \label{euklid1}
  &\Rightarrow \quad \rho(A^* A)^{\frac{1}{2}} \ge \|A \|_2 =
  \sup_{x\ne 0} \frac{ \|A x\|_2}{ \|x\|_2}
\intertext{ Sei nun $\hat x \in \KK^n$ mit $\|\hat x\|_2 = 1$ und
  $A^* A \hat x = \lambda_1 \hat x$. Dann gilt:}
  \nonumber
  \|A\|_2^2 &\ge \| A \hat x\|_2^2 = \hat x^* A^* A \hat x =
  \lambda_1 \hat x^* \hat x = \lambda_1 \| \hat x\|_2^2 = \lambda_1 = \rho(A) \\
  \label{euklid2}
  &\Rightarrow \quad \|A\|_2 \ge \rho(A^* A)^{1/2}
\end{align}
Aus (\ref{euklid1}) und (\ref{euklid2}) folgt $\|A\|_2 = \rho(A^* A)^{1/2}$
\end{proof}



