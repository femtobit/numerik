\subsubsection{Definitheit von Matrizen}
Im Folgenden sei $A$ eine hermitesche Matrix.
\begin{enumerate}
  \item [a)] $A$ heißt "`positiv definit"' ($A > 0$) wenn
  \begin{align*}
    &\text{\textbullet} \quad \forall x \in \KK \setminus \{0\}: x^* A x > 0 \\
    \Leftrightarrow \quad &\text{\textbullet} \quad \det(A_i) > 0 \; \forall \; i = 1, \dots , n \\
    \intertext{\qquad \qquad \qquad dabei ist $A_i$ der $i$-te Hauptminor von $A$, d.h. die $i\times i$ Teilmatrix}
    & \quad \quad A_i =
    \begin{pmatrix}
      a_{11} & \cdots  & a_{1n} \\
      \vdots & \ddots & \vdots \\
      a_{i1} & \cdots & a_{ii}
    \end{pmatrix} \\
    \Leftrightarrow \quad &\text{\textbullet} \quad  \sigma (A) \subset (0, \infty)
  \end{align*}
  \item[b)]$A$ heißt "`positiv semidefinit"' ($A \ge 0$) wenn
    \begin{align*}
    &\text{\textbullet} \quad \forall x \in \KK \setminus \{0\}: x^* A x \ge 0 \\
    \Leftrightarrow \quad &\text{\textbullet} \quad  \sigma (A) \subset [0, \infty)
  \end{align*}
  \item[c)] $A$ heißt "`negativ (semi-) definit"': Analog.
\end{enumerate}

\subsubsection{Wurzel einer Matrix}
Sei $A \in \KK^{n \times n}$ ; $A \ge 0$ (das beinhaltet $A = A^*$). Dann ist
$\sqrt{A} = A^{\frac{1}{2}}$ definiert als eine Matrix $\sqrt{A} \ge 0$ mit
$\left(\sqrt{A} \right)^2 = A$.

\begin{proof}[Existenzbeweis]
$A \ge 0 \Rightarrow \exists \; U \in \U_n(\KK)$ ($\U_n = $ Menge
der unitären Matrizen), so dass  \newline
$A = U^* \cdot \diag(\lambda_1, \dots \lambda_n) \cdot U$ mit den Eigenwerten
$\lambda_1, \dots \lambda_n \in [0, \infty)$. \newline
Betrachte nun $\sqrt{A} = U^* \cdot \diag(\sqrt{\lambda_1}, \dots, \sqrt{\lambda_n}) \cdot U$.
\end{proof}

\subsection{Vektor- und Matrixnormen}
\begin{Definition} [Norm]
  Sei $X$ ein $\KK$-Vektrorraum. Eine Abbildung $\| \cdot \| \rightarrow [0, \infty)$
  heißt Norm wenn
  \begin{enumerate}
    \item[(i)]    $\|x\| > 0 \quad \forall \; x \in X \setminus \lbrace 0 \rbrace$ (Positivität)
    \item[(ii)]   $\| \lambda \cdot x\| = |\lambda| \cdot \|x\| \quad \forall \; \lambda \in \KK, \forall x \in X$ (absolute Homogenität)
    \item[(iii)]  $\|x + y \| \le \|x\| + \|y\| \quad \forall \; x, y \in X$ (Dreiecksungleichung)
  \end{enumerate}
\end{Definition}

\begin{Bemerkungen}
\quad
  \begin{enumerate}
    \item[a)] Konvergenz $(x_n) \rightarrow x$ bedeutet $\|x_n - x\| \rightarrow 0$.
    \item[b)] Wenn zusätzlich $\dim (X) < \infty$, dann sind alle Normen auf $X$
    äquivalent, d.h.für jede Norm $\| \cdot \|_a$ und $\| \cdot \|_b$ auf $X$ gibt
    es Konstanten $c_1, c_1 > 0$, so dass
    $c_1 \cdot \| x \|_a \le \| x \|_b \le c_2 \cdot \| x \|_a \; \forall x \in X$.
    Es ist klar, dass $\dim(\KK^n) = n < \infty$ und
    $\dim(\KK^{m \times n}) = m \cdot n  < \infty$ .
    \item[c)] Beispiele für Normen auf $\KK^n$:
    \begin{enumerate}
      \item[(i)] $\|x\|_\infty := \max_{i = 1, \cdots , n} |x_i| \quad $Maximums-Norm
      \item[(ii)] $\|x\|_p := \left( \sum_{i = 1}^n |x_i|^p \right)^\frac{1}{p} \quad$
      $p$-Norm; $p \in [1, \cdots , \infty)$. Für $p = 2$ Euklidische Norm.
    \end{enumerate}
    Es gilt:
    $\|x\|_\infty \le \|x\|_2 < \|x\|_1 \le \sqrt{n} \cdot \|x\|_2 < n \cdot \|x\|_\infty
    \quad \forall \; x \in \KK^n$
  \end{enumerate}
\end{Bemerkungen}

Beispiele für Matrix-Normen ($A \in X = \KK^{m \times n})$:
\begin{enumerate}
  \item[a)] $\|A\|_1 :=  \max_{j = 1, \cdots , n} \left( \sum_{i = 1}^m |a_{ij}| \right) \quad$ Spaltensummen-Norm
  \item[b)] $\|A\|_\infty :=  \max_{i = 1, \cdots , m} \left( \sum_{j = 1}^n |a_{ij}| \right) \quad$ Zeilensummen-Norm
  \item[c)] $\|A\|_{\text{F}} :=  \left(\sum_{i = 1}^m  \sum_{j = 1}^n |a_{ij}|^2 \right)^{\frac{1}{2}} \quad$
  Frobenius-Norm
\end{enumerate}

\begin{Definition}
\quad \\
\begin{enumerate}
  \item[(i)] Eine Matrix-Norm $\|\cdot\|$ heißt \textit{sub-multiplikativ}, wenn
  $\|A \cdot B\| \le \|A\| \cdot \|B\| \newline \forall \; A \in \KK^{m \times k}, \; B \in \KK^{k \times n}$
  \item[(ii)] \textit{verträglich mit der Vektor-Norm $\| \cdot \|_V$}, wenn
  $\|A \cdot x\| \le \|A\| \cdot \|x\|_V \quad \forall \; A \in \KK^{m \times n}, \; x \in \KK^n$
\end{enumerate}
\end{Definition}

Beispiele:
\begin{enumerate}
  \item[(i)] $\|A\| := \max_{\substack{i = 1 \cdots m\\j = 1 \cdots n}}|a_{ij}|$
  ist Matrix-Norm, jedoch nicht sub.multiplikativ, denn für \\
  $A = B = \left( \begin{smallmatrix}1 & 1 \\1 & 1 \end{smallmatrix} \right)$ gilt
  $\|A\| = \|B\| = 1$, aber
  $\|A \cdot B\| = \left \| \left(\begin{smallmatrix}2 & 2 \\2 & 2 \end{smallmatrix}\right) \right\| =
  2 \nleq 1 = \|A\| \cdot \|B\|$
  \item[(ii)] Die Frobenius-Norm ist mit der euklidischen Norm verträglich:
  \begin{align*}
    & |(A \cdot x)_i|^2 = \sum_{j = 1}^n |a_{ij} \cdot x_j |^2 \overset{(*)}{\le}
    \left( \sum_{j = 1}^n |a_{ij}|^2  \right) \cdot \left( \sum_{i = 1}^n |x_i |^2  \right) =
    \sum_{j = 1}^n |a_{ij}|^2 \cdot |x|^2
    & \intertext{Bei $(*)$ wurde die Cauchy-Schwarzsche Ungleichung angewendet}
    & \Rightarrow \|A \cdot x\|_2 \le \|A\|_{\text{F}} \cdot \|x\|_2 \quad
    \forall \; A \in \KK^{m \times n} \; , \; x \in \KK^n
  \end{align*}
\end{enumerate}



