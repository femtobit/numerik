\subsubsection{Definitheit von Matrizen}
Im Folgenden sei $A$ eine hermitesche Matrix.
\begin{enumerate}
  \item [a)] $A$ heißt "`positiv definit"' ($A > 0$) wenn
  \begin{align*}
    &\text{\textbullet} \quad \forall x \in \KK \setminus \{0\}: x^* A x > 0 \\
    \Leftrightarrow \quad &\text{\textbullet} \quad \det(A_i) > 0 \; \forall \; i = 1, \dots , n \\
    \intertext{\qquad \qquad \qquad dabei ist $A_i$ der $i$-te Hauptminor von $A$, d.h. die $i\times i$ Teilmatrix}
    & \quad \quad A_i =
    \begin{pmatrix}
      a_{11} & \cdots  & a_{1n} \\
      \vdots & \ddots & \vdots \\
      a_{i1} & \cdots & a_{ii}
    \end{pmatrix} \\
    \Leftrightarrow \quad &\text{\textbullet} \quad  \sigma (A) \subset (0, \infty)
  \end{align*}
  \item[b)]$A$ heißt "`positiv semidefinit"' ($A \ge 0$) wenn
    \begin{align*}
    &\text{\textbullet} \quad \forall x \in \KK \setminus \{0\}: x^* A x \ge 0 \\
    \Leftrightarrow \quad &\text{\textbullet} \quad  \sigma (A) \subset [0, \infty)
  \end{align*}
  \item[c)] $A$ heißt "`negativ (semi-) definit"': Analog.
\end{enumerate}

\subsubsection{Wurzel}
Sei $A \in \KK^{n \times n}$ ; $A \ge 0$ (das beinhaltet $A = A^*$). Dann ist
$\sqrt{A} = A^{\frac{1}{2}}$ definiert als eine Matrix $\sqrt{A} \ge 0$ mit
$\left(\sqrt{A} \right)^2 = A$.

Existenszbeweis: $A \ge 0 \Rightarrow \exists \; U \in \U_n(\KK)$ ($\U_n = $ Menge
der unitären Matrizen), so dass  \newline
$A = U^* \cdot \diag(\lambda_1, \dots \lambda_n) \cdot U$ mit den Eigenwerten
$\lambda_1, \dots \lambda_n \in [0, \infty)$. \newline
Betrachte dazu $\sqrt{A} = U^* \cdot \diag(\sqrt{\lambda_1}, \dots \sqrt{\lambda_n}) \cdot U$.

\subsection{Vektor- und Matrixnormen}
\begin{Definition} [Norm]
  Sei $X$ ein $\KK$-Vektrorraum. Eine Abbildung $\| \cdot \| \rightarrow [0, \infty)$
  heißt Norm wenn
  \begin{enumerate}
    \item[(i)]    $\|x\| > 0 \quad \forall \; x \in X \setminus \lbrace 0 \rbrace$
    \item[(ii)]   $\| \lambda \cdot x\| = |\lambda| \cdot \|x\| \quad \forall \; \lambda \in \KK, \forall x \in X$
    \item[(iii)]  $\|x + y \| \le \|x\| + \|y\| \quad \forall \; x, y \in X$
  \end{enumerate}
\end{Definition}

% \begin{Bemerkung}  %funktioniert nicht richtig
\textbf{Bemerkungen:}
\begin{enumerate}
    \item[a)] Konvergenz $(x_n) \rightarrow x$ bedeutet $\|x_n - x\| \rightarrow 0$.
    \item[b)] Wenn zusätzlich $\dim (X) < \infty$, dann sind alle Normen auf $X$
    äquivalent, d.h.für jede Norm $\| \cdot \|_a$ und $\| \cdot \|_b$ auf $X$ gibt
    es Konstanten $c_1, c_1 > 0$, so dass
    $c_1 \cdot \| x \|_a \le \| x \|_b \le c_2 \cdot \| x \|_a \; \forall x \in X$.
  \end{enumerate}
