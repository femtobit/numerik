\subsection{Konditionen einer Matrix}

Sei $A \in \Gl_n(\KK), \; b \in \KK^n $ mit einem linearen Gleichungsssystem (LGS)
$A \cdot x = b$. Wir betrachten nun ein "`gestörtes Problem"' $A \cdot \tilde{x} = \tilde{b}$,
wobei $\tilde{b}$ von $b$ durch Meß- oder Rundungsfehler abweicht: Wir definieren
$x_d := \tilde{x} - x$ und $b_d := \tilde{b} - b$. Damit wird:
\begin{align*}
  & A \cdot x_d = b_d \quad \Rightarrow \quad x_d = A^{-1} \cdot b_d \\
  & \Rightarrow \quad \|x_d\| \le \|A^{-1}\| \cdot \|b_d\| \\
  & \| b \| = \| A \cdot x \| \le \|A\| \cdot \|x\|\\
  & \Rightarrow \quad \|x \| \ge \|A^{-1}\| \cdot \|b\|
\intertext{damit kann man den relativen Fehler abschätzen:}
 & \frac{\|x_d\|}{\|x\|} \le  \|A\| \cdot \|A^{-1}\| \cdot \frac{\|b_d\|}{\|b\|}
\end{align*}

\begin{Definition}[Kondition]
  Für $A \in \Gl_n(\KK)$ heißt $\cond(A) :=  \|A^{-1}\|\cdot\|A\|$ die Kondition von $A$.
\end{Definition}
\begin{Bemerkungen}
\quad \\
  \begin{enumerate}
    \item[a)] Sprechweise: Eine Matrix $A$ heißt gut / schlecht konditioniert, wenn
    $\cond(A)$ klein / groß ist.
    \item[b)] Wegen $1 = \|I\| = \|A^{-1} \cdot A\| \le \|A^{-1}\|\cdot\|A\| = \cond(A)$
    gilt stets $\cond(A) \ge 1 \; \forall A \in \Gl_n(\KK)$ und für alle
    Matrizennormen $\|\cdot\|$.
    \item[c)] Unitäre Matrizen $U \in \U_n(\KK)$ haben stets $\cond(U) = 1$
    \item[d)] Für das obige Problem gilt: $\frac{\|x_d\|}{\|x\|} \le
      \cond (A) \cdot \frac{\|b_d\|}{\|b\|}$.
    \item[e)] Veranschaulichung: [Bild: Einheitskreis im $\RR^n$ wird durch
    eine Matrix $A$ in eine schmale schräg gestellte Ellipse mit der großen bzw.
    kleinen Halbachse $a$ bzw. $b$ transformiert]. Die "`maximale Streckung"'
    ist $a = \max_{\|x\| = 1} \| A \cdot x\| = \|A\|$, die "`minimale Streckung"' ist
    \begin{align*}
    b &= \min_{\|x\| = 1} \| A \cdot x\| =
    \left( \max_{\|x\| = 1} \frac{1}{\| A \cdot x\|} \right)^{-1} =
    \left( \max_{\|x\| \ne 0} \frac{\|x\|}{\| A \cdot x\|} \right)^{-1} \\
    \intertext{Substitution  $y = A^{-1} \cdot x$}
     &= \left( \max_{\|y\| \ne 0} \frac{\|A^{-1} \cdot y\|}{\|y\|} \right)^{-1}
     = \frac{1}{\|A^{-1}\|}
    \end{align*}
  \end{enumerate}
\end{Bemerkungen}

Konsequenz: $\cond(A) = \frac{\text{maximale Streckung des Einheitskreises}}
{\text{minimale Streckung des Einheitskreises}}$

\begin{Lemma}[Störungslemma]
  Sei $A \in \Gl_n(\KK), \; S \in \KK^{n \times n}$ so dass
  $\|A^{-1}\| \cdot \|S\| < 1$. Dann gilt: $A + S \in \Gl_n(\KK)$ mit
  $\|(A + S)^{-1}\| \le \frac{\|A^{-1}\|}{1 - \|A^{-1}\| \cdot \|S\|}$.
\end{Lemma}
\begin{proof}
Es gilt $A + S = A(I + A^{-1} S)$ und ferner:
\begin{align}
   \nonumber
   \|(I + A^{-1} S) \cdot y\| &\ge \|y\| - \|A^{-1} \cdot S \cdot y\|  \ge \\
   \label{stoer1}
   & \ge \|y\| - \|A^{-1}\| \cdot \| S \| \cdot \|y\| =
   \underbrace {\left( 1 - \|A^{-1}\| \cdot \|S\| \right)}_{> 0}  \cdot \|y\| \\
   \nonumber
   & \Rightarrow \quad I + A^{-1} \cdot S  \in \Gl_n(\KK) \; \text{ ist invertierbar}\\
   \nonumber
   & \Rightarrow \quad A + S \in \Gl_n(\KK) \; \text{ ist invertierbar}
\end{align}
Mit $y = (I + A^{-1} \cdot S) \cdot x$ folgt aus Gleichung (\ref{stoer1}:
\begin{align*}
   \|x\| & \ge \left( 1 - \|A^{-1}\| \cdot \|S\| \right) \cdot
     \left\| (I + A^{-1} \cdot S)^{-1} \cdot x \right\| \\
  & \Rightarrow \; \left\| (I + A^{-1} \cdot S)^{-1} \cdot x \right\| \le
     \frac{1}{1 - \|A^{-1}\| \cdot \|S\|} \cdot \|x\|\\
     \left\|(A+S)^{-1}\right\|  & = \left\| (I + A^{-1} \cdot S)^{-1} A^{-1} \right\| \le
     \left\| (I + A^{-1} \cdot S)^{-1} \right\|\cdot \| A^{-1} \| \le
       \frac{\|A^{-1}\|}{1 - \|A^{-1}\| \cdot \|S\|}
\end{align*}
\end{proof}

\begin{Satz}
Sei $A \in \Gl_n(\KK), \; A_d \in \KK^{n \times n}$  mit  $\|A^{-1}\| \cdot \|A_d\| \le 1$.
Seien $b, b_d \in \KK^n$  und  $x, x_d \in \KK^n$, so dass $A \cdot x = b$ und
$(A + A_d)(x + x_d) = (b + b_d)$. Dann gilt:
\begin{align*}
\frac{\|x_d\|}{\|x\|} = \frac{\cond(A)}{1 - \cond(A) \cdot \frac{\|A_d\|}{\|A\|}} \cdot
\left( \frac{\|A_d\|}{\|A\|} + \frac{\|b_d\|}{\|b\|} \right)
\end{align*}
\end{Satz}
\begin{proof}
  TODO Michael
\end{proof}

\subsection{Elementarmatrizen}

\begin{Definition}[Dyade, Elementarmatrix]
  TODO Michael
\end{Definition}

\begin{Bemerkungen}
\quad \\
  \begin{enumerate}
    \item[a)]
  \end{enumerate}
\end{Bemerkungen}

\begin{Lemma}
  TODO Michael
\end{Lemma}
\begin{proof}
  TODO Michael
\end{proof}


