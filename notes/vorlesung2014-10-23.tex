\begin{Beispiele}[Elementarmatrizen]
\quad
\begin{itemize}
\item[a)] \emph{Vertauschungsmatrix.}
  $P_{rs} := (e_1,\ldots,e_{r-1},e_s,e_{r+1},\ldots,e_{s-1},e_r,e_{s+1},\ldots,e_n).$

  Einheitsmatrix mit $r$-ter und $s$-ter Spalte vertauscht.

  Es gilt $P_{rs} = I - (e_r - e_s)(e_r - e_s)^\T$ und $P_{rs} = R_{rs}^\T$,
  $P_{rs}^2 = P_{rs}$.
  
  $P_{rs} A$ vertauscht $r$-te und $s$-te Spalte in A. Aufwand: 0 Mult.
  
  Es gilt $\det P_{rs} = -1$, es handelt sich also um eine Spiegelung.

\item[b)] \emph{Gauß-Matrix.}
  Sei $1 \le k \le n$, $u = (0, \ldots, 0, u_{k+1}, \ldots, u_n)^\T$.
  \[
    G_k := I - ue_k^\T = \begin{pmatrix}
      1 & & & \\
      & \ddots & & & \\
      & & 1 & & \\
      & & -u_{k+1} & \ddots & \\
      & & \vdots & & \\
      & & -u_n & & 1 \\
    \end{pmatrix}.
  \]

  Beachte: $G^{-1}_k = (I - ue_k^\T)^{-1} = I + ue_k^\T$.

  Für $A = \begin{pmatrix} a_1 \\ \vdots \\ a_n \end{pmatrix} \in \KK^{n\times n}$
  gilt \[
    G_k A = (I - ue_k^\T) A = A - u(e_k^\T A) = A - ua_k
          = \begin{pmatrix} a_1 \\ \vdots \\ a_k \\ a_{k+1} - u_{k+1} a_k \\ \vdots \\ a_n - u_n a_k \end{pmatrix}.
  \]
  Für $j = k+1, \ldots, n$ Subtraktion von $a_j$ mit $u_j a_k$.

  Aufwand (für $A \in \RR^{n\times m}$): $(n-k) \cdot m$ Mult.

\item[c)] \emph{Householder-Spiegelung.}
  $v \in \KK^n \setminus \{0\}$, $Q_v := I - \frac{2}{v^{*}v}vv^{*}.$
  Es gilt $Q_v^{*} = Q_v$ und \[
    Q_v^2 = (I - \frac{2}{v^{*}v}vv^{*})(I - \frac{2}{v^{*}v}vv^{*})
          = I - \frac{4}{v^*v}vv^* + \frac{4}{(v^*v)^2}vv^*vv^* = I.    
  \]

  Also ist $Q_v$ hermitesch und unitär (denn die Eigenwerte sind reell und
  haben Betrag 1).

  Es gilt $Q_v v = -v$ und $Q_v w = w$ für alle $w \in \vspan\{v\}^\perp$.
  Es handelt sich bei $Q_v$ also um eine Spiegelung an der Hyperebene
  $\vspan\{v\}^\perp$.

  \item[d)] \(
    \lambda \in \KK \setminus \{0\}$, $S_k := I + (\lambda - 1)e_ke_k^\T
      = \diag(1, \ldots, \underset{k-1}{1}, \underset{k}{\lambda}, \underset{k+1}{1}, \ldots, 1)
    \).

    $S_k^{-1} = I + (\frac{1}{\lambda} - 1) e_k e_k^\T.$

    \[
      S_k \begin{pmatrix} a_1 \\ \vdots \\ a_n \end{pmatrix}
        = \begin{pmatrix} a_1 \\ \vdots \\ a_{k-1} \\ \lambda a_k \\ a_{k+1} \\ \vdots \\ a_n \end{pmatrix}.
    \]

    Aufwand: $m$ Mult. für $A \in \KK^{n\times m}$.
\end{itemize}
\end{Beispiele}

\subsection{Gaußsches Eliminationsverfahren}

Gegeben: LGS $Ax = b$, $A \in \Gl_n(\KK)$, $b \in \KK^n$.

Ziel: Sukzessive Multiplikation von links mit Gaußmatrizen $G_i$ und Vertauschungsmatrizen $P_i$ ($i \in \{1, \ldots, n\}$),
so dass \[
  \underset{=: R\text{ (obere Dreiecksmatrix)}}{\underbrace{
    G_nP_n \cdots G_1 P_1 A}x}
  = \underset{=: c \in \KK^n}{\underbrace{
    G_nP_n \cdots G_1 P_1 b}}.
\]

Danach: Lösung von $Rx = c$ durch \emph{Rückwärtseinsetzen}:
\begin{codebox}
  \li \For $i = n, n-1, \ldots, 1$:
    \Do
      \li $x_i \gets \frac{1}{r_{ij}} \left( c_i - \sum_{j=i+1}^n r_{ij} x_j \right)$
    \End
\end{codebox}

Das Rückeinsetzen benötigt insgesamt \[
  \sum_{i=1}^n (n - i + 1) = n^2 + n - \frac{n(n+1)}{2} = \frac{n(n+1)}{2}
    = O(n^2)
\] Multiplikationen.

Transformation $(A \mid b) \to (R \mid c)$ über folgenden Prozess:
Für $k=n-1, \ldots, 1$:
\begin{itemize}
  \item[a)] Bestimme $r(k) \in \{k, \ldots, n\}$, so dass
    $$|a_{r(k),k}| = \max_{i=k,\ldots,n}|a_ik|.$$
    $a_{r(k),k}$ heißt \emph{Pivotelement}.

    Es ist $a_{r(k),k} \ne 0$, da sonst $A \notin \Gl_n(\KK)$.
    Vertausche die Spalten $k$ und $r(k)$ (\emph{Pivotisierung}).

    \item[b)] Für $$l_k := (0, \ldots, 0, l_{k+1,k}, \ldots, l_{n,k})^\T$$
    mit $l_{ik} := \frac{a_{ik}}{a_{kk}}$, setze $G_k := I - l_ke_k^\T$
    und $(A \mid b) := (G_k A \mid G_k b)$.
\end{itemize}

% Anschaulich: TODO

Nach $n-1$ Schritten erhalten wir eine obere Dreiecksmatrix.

\begin{Bemerkung}
Das Verfahren ist für $A \in \Gl_n(\KK)$ immer durchführbar.
Wäre dies nämlich im $k$-ten Schritt ($1 \le k < n$) nicht der Fall, so
hätte die zugehörige Matrix $A^{(k)}$ die Form \[
  A^{(k)} = \left( \begin{array}{ccc|cccc}
  a_{11} & \cdots & a_{1k} & a_{1,k+1} & a_{1,k+2} & \cdots & a_{1n} \\
  & \ddots & \vdots & \vdots & \vdots & \ddots & \vdots \\
  0 & & a_{kk} & a_{k,k+1} & a_{k,k+2} & \cdots & a_{kn} \\ \hline
  0 & \cdots & 0 & 0 & a_{k+1,k+2} & \cdots & a_{k+1,n} \\
  \vdots & \ddots & \vdots & \vdots & \vdots & \ddots &  \vdots \\
  0 & \cdots & 0 & 0 & a_{n,k+1} & \cdots & a_{nn}
  \end{array} \right)
\] also \(
  A^{(k)} = \left( \begin{array}{c|c} \Lambda_1 & \Lambda_2 \\\hline 0 & \Lambda_3 \end{array} \right)
\) und damit
$$\pm \det(A) = \det(A^{(k)}) = \det(\Lambda_1) \det(\Lambda_3) = 0\text{, da }\det (\Lambda_3) = 0.$$
\end{Bemerkung}

\begin{Algorithmus}[Gauß-Verfahren]
\quad
\begin{codebox}
\Procname{$\proc{Gauss-Elimination}(A,b)$}
\li \For $k = 1, \ldots, n$:
\Do
  \li Bestimme $r(k) \in \{k, \ldots, n\}$, so dass
      $|a_{r(k),k}| = \max_{i=k,\ldots,n} |a_{ik}|.$
  \li Vertausche $k$-te und $r(k)$-te Zeile von $(A \mid b)$.
  \li \For $i \gets k + 1, \ldots, n$:
  \Do
    \li $b_i \gets b_i - \frac{a_{ik}}{a_{kk}} b_k$.
    \li \For $j \gets k+1, \ldots, n$:
    \Do
      \li $a_{ij} \gets a_{ij} - \frac{a_{ik}}{a_{kk}} a_{kj}$.
    \End
    \li $a_{ik} \gets 0$.
  \End
\End
\end{codebox}
\end{Algorithmus}

\begin{Bemerkungen}
\quad
\begin{itemize}
  \item[a)] Das Gauß-Verfahren ist genau dann ohne Pivotisierung möglich,
    wenn alle Hauptabschnittsdeterminanten von 0 verschieden sind. (Beweis in den Übungen.)

  \item[b)] Die Pivotisierung trägt zur numerischen Stabilität bei.

  Beispiel: Löse \[
    A = \left( \begin{array}{ll}
      1.00\e{-3} & 1.00\e{0} \\
      1.00\e{0} &  1.00\e{0}
    \end{array} \right),
    \qquad
    b = \begin{pmatrix}
      1.00\e{0} \\
      2.00\e{0}
    \end{pmatrix}
  \]
  mit zweistelliger Gleitpunktarithmetik (d.h. schneide nach zweiter Stelle ab).

  \begin{itemize}
    \item[(i)] Ohne Pivotisierung: \[
        \rightsquigarrow \left( \begin{array}{lr|r}
          1.00\e{-3} & 1.00\e{0} & 1.00\e{0} \\
          0         & -9.99\e{3} & -9.98\e{3}
        \end{array} \right)
      \] also $x_2 = 1.00\e{0}$, $x_1 = 0$.

    \item[(ii)] Mit Pivotisierung: \[
        \rightsquigarrow \left( \begin{array}{ll|l}
          1.00\e{0} & 1.00\e{0} & 2.00\e{0} \\
          0         & 1.00\e{0} & 1.00\e{0}
        \end{array} \right)
      \] also $x_2 = 1.00\e{0}$, $x_1 = 1.00\e{0}$.
  \end{itemize}
  Die tatsächliche Lösung ist $x_1 = 1.00101\ldots$, $x_2 = 0.99899\ldots$.
\end{itemize}
\end{Bemerkungen}
