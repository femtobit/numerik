\subsection{LR-Zerlegung}

\paragraph{Gauß-Verfahren (formal)}
$$G_{n-1} P_{n-1,r(n-1)}G_{n-2}P_{n-2,r(n-2)} \cdots G_1P_{1,r(1)}(A \mid b) = (R \mid c)$$
Nachrechnen ergibt: Für $i>j$ ist $P_{i,r(i)} G_jP_{i,r(i)}$ eine Gauß-Matrix
(vertausche $l_{ij}$ und $l_{r(i),j}$).

\begin{itemize}
\item[$\Rightarrow$] Für $P = P_{n-1,r(n-1)} \dots P_{1,r(1)}$ gilt 
    $$\widetilde{G}_{n-1} \cdots \widetilde{G}_1 P (A \mid b) = (R \mid c)$$
    mit Gauß-Matrizen $\widetilde{G}_{n-1}, \dots, \widetilde{G}_1$.

\item[$\Rightarrow$] $P (A \mid b) = 
    \widetilde{G}_1^{-1} \cdots \widetilde{G}_{n-1}^{-1} (R \mid c)$
    mit $\widetilde{G}_i = I - \widetilde{l_i} e_i^\T 
         \Rightarrow \widetilde{G}_i^{-1} = I + \widetilde{l_i} e_i^\T$
\end{itemize}
