
\subsection{LR-Zerlegung}

\paragraph{Gauß-Verfahren (formal)}
$$G_{n-1} P_{n-1,r(n-1)}G_{n-2}P_{n-2,r(n-2)} \cdots G_1P_{1,r(1)}(A \mid b) = (R \mid c)$$
Nachrechnen ergibt: Für $i>j$ ist $P_{i,r(i)} G_jP_{i,r(i)}$ eine Gauß-Matrix
(vertausche $l_{ij}$ und $l_{r(i),j}$).

\begin{itemize}
\item[$\Rightarrow$] Für $P = P_{n-1,r(n-1)} \dots P_{1,r(1)}$ gilt
    $$\widetilde{G}_{n-1} \cdots \widetilde{G}_1 P (A \mid b) = (R \mid c)$$
    mit Gauß-Matrizen $\widetilde{G}_{n-1}, \dots, \widetilde{G}_1$.

\item[$\Rightarrow$] $P (A \mid b) =
    \widetilde{G}_1^{-1} \cdots \widetilde{G}_{n-1}^{-1} (R \mid c)$
    mit $\widetilde{G}_i = I - \widetilde{l_i} e_i^\T
         \Rightarrow \widetilde{G}_i^{-1} = I + \widetilde{l_i} e_i^\T$
\end{itemize}

Nachrechnen ergibt
$L := \widetilde{G}_1^{-1} \dots \widetilde{G}_{n-1}^{-1} =
\begin{pmatrix}
\frac{1}{l_{11}} & & 0 \\
\vdots & & \vdots \\
& & 1 & \\
\widetilde{l}_{n1} & \dots & \widetilde{l}_{nn-1} & 1 \\
\end{pmatrix}$
Insgesamt: PA = LR ~~ "`LR-Zerlegung"'
\begin{itemize}
\item P: Permutationsmatrix
\item L: untere Dreiecksmatrix mit Einsen auf Diagonale
\item R: obere Dreiecksmatrix
\end{itemize}

Lösung von $Ax = b$ mit LR-Zerlegung $PA=LR$ \\
$Ax=b \Rightarrow PAx = Pb \Rightarrow LRx = Pb$
\begin{enumerate}
\item[(1)]
Löse $Ly = b$ durch Vorwärtseinsetzen \\
Für $i = 1, \dots ,n$ \\
$y_i = \left( b_{\pi(i)} - \sum_{k = 1}^{i -1} l_{ik} y_k  \right) / l_{ii}$ \\
Ende für  ~~ (O($n^2$) Mult. )\\
($P = (e_{\pi(1)}) \dots e_{\pi(n)}), \pi \in S_n$)
\item[(2)]
Löse LGS $Rx = y$ durch Rückwärtseinsetzen (O($n^2$) Mult. )
\end{enumerate}


\underline{Algorithmus} \\
Eingabe: $A \in Gl_n (\KK)$ \\
Ausgabe: LR-Zerlegung von A:  Matrix $ A \in \KK^{n \times n}$ und
$\pi \in S_n$ \\
so dass $PA = LR$ (ursprüngliches A) für \\
$P = (e_{\pi(1)} \dots e_{\pi(n)})$ mit \\
$L = (l_{ij}) \text{ mit } l_{ij} =
\begin{cases} \delta_{ij} &: i \le j \\  a_{ij} &: i > j  \end{cases} $ \\
$R = (r_{ij}) mit r_{ij} =
\begin{cases} a_{ij} &: i \le j \\ 0 &: i > r  \end{cases}$

Initialisieren: Permutationsvektor $\pi = (1,2, \dots, n)$\\
Für $k = 1, \dots, n-1$ \\

\begin{enumerate}
\item[(a)] Bestimme $r \in \{k, \dots, n \} $ mit $ |a_{ik}|
= max_{i= k, \dots, n} |a_{in}|$ (Pivotelement)
\item[(b)] "`Test auf Singularität"': Falls $a_{rk}=0$, 
	dann $A \notin Gl_n(\mathbb{K})$ \textsc{stopp}
\item[(c)] 
Falls $k \notin r$: Vertausche k-te und r-te Zeile in A:

%\begin{algorithm}
%\begin{algorithmic}
%\Function{Zeilenvertauschen}{A}
%\For {$j \gets 1, \dots, n$}
	%\State \Call{$Vertausche$}{$a_{kj}, a_{rj}$}
%\EndFor
%\State \Call{Vertausche}{$\pi(k),\pi(r)$}
%\EndFunction
%\end{algorithmic}
%\end{algorithm}

\item[(d)] 
%\begin{algorithm}
%\begin{algorithmic}
%\For {$i = k+1, \dots, n$}
%\State $a_{ik} \gets \frac{a_{ik}}{a_{kk}}$ 
%\For {$j = k+1, \dots, n$}
%\State $a_{ij} \gets a_{ij} - a_{ik} \cdot a_{kj}$
%\EndFor 
%\EndFor 
%\end{algorithmic}
%\end{algorithm}
\item[(e)]
Letzter Test auf Singularität\\
Falls $a_{nn}=0$, dann $A \notin Gl_n(\mathbb{K})$ \textsc{stopp}
\end{enumerate}
Bem.:
\begin{enumerate}
\item[(a)]
Vorteile von LR gegenüber Gauß-Verfahren: Auch wenn 
$Ax^{(i)}= b^{(i)} \in \mathbb{K}^n$ gelöst werden muss, dann muss 
LR-Zerlegung nur einmal berechnet werden (danach nur Vorwärts und 
Rückwärts Einsetzen)
\item[(b)]
Aufwand von LR-Zerlegung: $O(n^3)$ Mult.
\item[(c)]
LR-Zerl. kann zur Bestimmung von $A^{-1}$ benutzt werden, da 
$A^{-1}= (x^{(1)} \cdots x^{(n)})$ wobei $Ax^{(i)}= e_i$ \\
(Aufwand $O(n^3) + O(n^2)= O(n^3)$ Mult. 
\end{enumerate}

Bsp.:
\begin{align*}
A= 
&\begin{pmatrix}
1  &  2  & 1 \\
-3 & -5  & 1 \\
-7 & -12 & -2
\end{pmatrix} \\
\text{Pivot} \rightarrow
&\begin{pmatrix}
-7 & -12 & -2 \\
-3 & -5  & -1 \\
1 & 2 & 1
\end{pmatrix}
 \pi = \begin{pmatrix} 3 & 3 & 1\end{pmatrix} \\
\text{1. Schritt:} 
&\begin{pmatrix}
-7           & -12         &           -2 \\
\frac{3}{7}  & \frac{1}{7} & -\frac{1}{7} \\
-\frac{1}{7} & \frac{2}{7} & \frac{5}{7}  
\end{pmatrix}
\rightarrow^{\text{Pivot}}
\begin{pmatrix}
-7           & -12         &           -2  \\
-\frac{1}{7} & \frac{2}{7} &  \frac{5}{7}  \\
\frac{3}{7}  & \frac{1}{7} & -\frac{1}{7} 
\end{pmatrix} 
 \pi = \begin{pmatrix} 3 & 1 & 2 \end{pmatrix} \\
\text{2. Schritt}
&\begin{pmatrix}
-7           & -12         &          -2  \\
-\frac{1}{7} & \frac{2}{7} &  \frac{5}{7} \\
\frac{3}{7}  & \frac{1}{7} & -\frac{1}{2} 
\end{pmatrix} \\
\text{Also}
&\begin{pmatrix}
0 & 0 & 1\\
1 & 0 & 0\\
0 & 1 & 0
\end{pmatrix} 
\begin{pmatrix}
1  &   2 &  1 \\
-3 &  -5 & -1 \\
-7 & -12 & -2 
\end{pmatrix} \\
=
&\begin{pmatrix}
           1 &           0 & 0 \\
-\frac{1}{7} &           1 & 0 \\
 \frac{3}{7} & \frac{1}{2} & 1
\end{pmatrix} 
\begin{pmatrix}
-7 &         -12 &           1 \\
 0 & \frac{2}{7} &  \frac{5}{7}\\
 0 &          0  & -\frac{1}{2}
\end{pmatrix}
\end{align*}

Bem.:
\begin{enumerate}
\item[(a)]
LR-Zerlegung ist one Pivotisierung möglich $\Leftrightarrow$
 Alle Hauptabschnittsdeterminanten $\neq 0$
\item[(b)]
Falls existent, dann ist LR-Zerlegung ohne Pivotisierung eindeutg (Übung)
\end{enumerate}
\subsection{Cholesky-Zerlegung}

