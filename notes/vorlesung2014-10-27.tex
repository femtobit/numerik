\subsection{LR-Zerlegung}

\paragraph{Gauß-Verfahren (formal)}
$$G_{n-1} P_{n-1,r(n-1)}G_{n-2}P_{n-2,r(n-2)} \cdots G_1P_{1,r(1)}(A \mid b) = (R \mid c)$$
Nachrechnen ergibt: Für $i>j$ ist $P_{i,r(i)} G_jP_{i,r(i)}$ eine Gauß-Matrix
(vertausche $l_{ij}$ und $l_{r(i),j}$).

\begin{itemize}
\item[$\Rightarrow$] Für $P = P_{n-1,r(n-1)} \dots P_{1,r(1)}$ gilt 
    $$\widetilde{G}_{n-1} \cdots \widetilde{G}_1 P (A \mid b) = (R \mid c)$$
    mit Gauß-Matrizen $\widetilde{G}_{n-1}, \dots, \widetilde{G}_1$.

\item[$\Rightarrow$] $P (A \mid b) = 
    \widetilde{G}_1^{-1} \cdots \widetilde{G}_{n-1}^{-1} (R \mid c)$
    mit $\widetilde{G}_i = I - \widetilde{l_i} e_i^\T 
         \Rightarrow \widetilde{G}_i^{-1} = I + \widetilde{l_i} e_i^\T$
\end{itemize}

Nachrechnen ergibt
$L := \widetilde{G}_1^{-1} \dots \widetilde{G}_{n-1}^{-1} = 
\begin{pmatrix} 
\frac{1}{l_{11}} & & 0 \\ 
\vdots & & \vdots \\ 
& & 1 & \\
\widetilde{l}_{n1} & \dots & \widetilde{l}_{nn-1} & 1 \\ 
\end{pmatrix}$
Insgesamt: PA = LR ~~ "`LR-Zerlegung"'
\begin{itemize}
\item P: Permutationsmatrix
\item L: untere Dreiecksmatrix mit Einsen auf Diagonale
\item R: obere Dreiecksmatrix
\end{itemize}

Lösung von $Ax = b$ mit LR-Zerlegung $PA=LR$ \\
$Ax=b \Rightarrow PAx = Pb \Rightarrow LRx = Pb$
\begin{enumerate}
\item[(1)]
Löse $Ly = b$ durch Vorwärtseinsetzen \\
Für $i = 1, \dots ,n$ \\
$y_i = b_{\pi(i)} - \sum_{k = 1}^{i} l_{in} y_n $\\
Ende für  ~~ (O($n^2$) Mult. )\\
($P = (e_{\pi(1)}) \dots e_{\pi(n)}), \pi \in S_n$) 
\item[(2)]
Löse LGS $Rx = y$ durch Rückwärtseinsetzen (O($n^2$) Mult. )
\end{enumerate}

\underline{Algorithmus} \\
Eingabe: $A \in Gl_n (mathbb{K})$ \\
Ausgabe: LR-Zerlegung von A:  Matrix $ A \in \mathbb{K}^{n \times n}$ und 
$\pi \in S_n$ \\
so dass $PA = LR$ (ursprüngliches A) für \\
$P = (e_{\pi(1)}) \dots e_{\pi(n)})$ mit \\
$L = (l_{ij}) \text{ mit } l_{ij} = 
\begin{cases} \delta_{ij} &: i \le j \\  a_{ij} &: i > j  \end{cases} $ \\
$R = (r_{ij}) mit r_{ij} = 
\begin{cases} a_{ij} &: i \le j \\ 0 &: i > r  \end{cases}$
