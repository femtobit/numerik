\documentclass[a4paper]{scrartcl}

\usepackage[utf8]{inputenc} \usepackage[ngerman]{babel}

\usepackage{amssymb} \usepackage{amsmath}
\usepackage{latexsym}
\usepackage{enumerate}

\newcounter{mydefctr}
\newtheorem{mydef}[mydefctr]{Definition}
%\newtheorem{vmydef}[mydefctr]{(vage) Definition}

%\newcommand{\rang}{\mathop{\mathrm{rang}}}
%\newcommand{\loes}{\mathop{\mathrm{L\ddot{o}s}}}
%\newcommand{\diag}{\mathop{\mathrm{diag}}}

\parindent 0pt
\parskip 0.25\baselineskip

\title{Vorlesung Numerik 27.10.2014}

\author{Laura Schüttpelz}

\begin{document}

\maketitle

\section{II.6 LR-Zerlegung}
Gauß-Verfahren (formal) \\
$G_{n-1} P_{n-1~r(n-1)}G_{n-2}P_{n-2~r(n-2)} \dots G_1P_{1~r(1)}(A|b) = (R|c)$ \\
Nachrechnen ergibt: Für $i>j$ ist $P_{i~r(i)} G_jP_{i~r(i)}$ eine Gauß-Matrix \\
(vertausche $l_{ij} ~~ l_{r(i)j}$)

\begin{align*}
\Rightarrow &\text{ Für } P = P_{n-1~r(n-1)} \dots P_{1r(1)} \\
&\text{gilt } \widetilde{G}_{n-1} \dots \widetilde{G}_1 P (A|b) = (R|c) \\
&\text{für Gauß-Matrizen } \widetilde{G}_{n-1}, \dots, \widetilde{G}_1 \\
\Rightarrow &P (A|b) = \widetilde{G}_1^{-1}, \dots, \widetilde{G}_{n-1}^{-1} (R|c) \\
&\widetilde{G}_i = I - \widetilde{l_i} e_i^T \Rightarrow \widetilde{G}_i^{-1} = I + \widetilde{l_i} e_i^T\\
 \end{align*}
\end{document}



