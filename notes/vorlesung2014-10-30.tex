\begin{Algorithmus}[Cholesky-Zerlegung]\hfill\newline
	Eingabe: $A > 0, A \in \KK^{n\times n}$ (benutzt die \glqq untere linke Hälfte \grqq von A),\\
	Ausgabe: $ L \in \Gl_n(\KK)$ untere $\triangle$-Matrix mit $l_{kk} >0 \forall k\in \{1,\cdots,n\} \text{ und } A=LL^*$ 
	\quad
	\begin{codebox}
		\Procname{$\proc{Cholesky-Zerlegung}$}
		\li $l := 0 \in \KK^{n\times n}$
		\li \For $k = 1, \ldots, n$:
		\Do
		\li $l_{kk} = \left(a_{kk} - \sum_{j=1}^{k-1}|l_{kj}|^{2}\right)^{\frac{1}{2}}$
		\li \For $i \gets k + 1, \ldots, n$:
		\Do
		\li $l_{ik} =\frac{1}{l_{kk}} \left(a_{ik} - \sum_{j=1}^{k-1}l_{ij}\overline{l_{kj}}\right)$
		\End
		\End
	\end{codebox}
\end{Algorithmus}
Aufwand:\begin{itemize}
	\item[a)] Berechnung von $l_{ij}$: $j$ Multiplikationen + 1 Wurzel falls $i=j$ 	\item[b)]$\Rightarrow$ Berechnung von L:\begin{align*}
	\sum_{i=1}^{n}\sum_{j=1}^{i}j = \sum_{i=1}^{n} \frac{i(i+1)}{2}= \frac{1}{2}\sum_{i=1}^{n} i^2 + \frac{1}{2} \sum_{i=1}^{n} i = \frac{1}{12}n(n+1)(2n+1)+\frac{n(n+1)}{4}  \\= O(n^3) + n\text{ Wurzel}
	\end{align*}
\end{itemize}
\begin{Beispiel}[Cholesky-Zerlegung]
	\quad
	$A = \begin{pmatrix}
	1 & 2 & 1\\
	2 & 5 & 2\\
	1 & 2 & 10
	\end{pmatrix}$\\
	\begin{align*}&l_{11}= \sqrt{a_{11}} = 1 \quad\Rightarrow l_{21}=\frac{a_{21}}{l_{11}}=2\\
	&l_{22}= \sqrt{a_{22}-l_{21}^2} = 1\\
	&l_{31}= \frac{a_{31}}{l_{11}} = 1, l_{32}= \frac{a_{32}-l_{31}l_{21}}{l_{22}} = 0\\
	&l_{33}= \sqrt{a_{33}-l_{31}^2-l_{32}^2} = \sqrt{9}=3\\
	&\Rightarrow L = \begin{pmatrix}
	1 & 0 & 0\\
	2 & 1 & 0\\
	1 & 0 & 3
	\end{pmatrix}
	\end{align*}
\end{Beispiel}

\subsection{QR-Zerlegung}
$A \in \RR^{m \times n} \text{ mit rang}(A) = n$\\
\newline
Ziel: Berechne $Q \in O_n(\RR)\quad
				R=\begin{pmatrix}
				R_1\\0
				\end{pmatrix}$ wobei $R_1$ obere $\triangle$-Matrix,
				$A=QR$ "QR-Zerlegung"\\
\newline
Nutzen für LGS $Ax=b: QRx = b\\ \Rightarrow \begin{pmatrix} R_1\\0\end{pmatrix}x= Q^Tb = \begin{pmatrix}c\\d\end{pmatrix} \text{ mit } c\in \RR^n, d \in \RR^{m-n}$\\
\newline
Dann: \begin{itemize}
	\item[(i)]Lösbarkeit $\Leftrightarrow d=0$
	\item[(ii)]Lösung erfüllt dann $R_1x=c$ ($\rightsquigarrow$Rückwärtseinsetzen)
\end{itemize}
Erinnerung: Householder-Spiegelung
\begin{align*}
&Q_v=I -\frac{2}{v^Tv}vv^T \;(v\in\RR^m)\\
&Q_v=Q_v^T, Q_v^2=I \quad (\Rightarrow Q_v \in O_m(\RR))
\end{align*}\\
Vorgehensweise: Sukzessive Multiplikation von links mit Householder-Matrizen s.d.\\ $ Q_{v_n}, \cdots, Q_{v_1} A =\begin{pmatrix}R_1\\0\end{pmatrix}\\
\Rightarrow A =QR \text{ für } R= \begin{pmatrix}R_1\\0\end{pmatrix} \text{ und } Q=Q_{v_1} \cdots Q_{v_n} $
\begin{Lemma}
	$x \in \RR^m \text{ mit } x_1 \neq 0$
	\newline Dann gilt für $v = \frac{x}{\|x\|_2}+sign(x_1)e_1$
	\begin{itemize}
		\item[a)]$\|v\|=\sqrt{2}\left(1+\frac{|x_1|}{\|x\|_2}\right)^2$
		\item[b)]$Q_vx=-sign(x_1)\|x\|_2e_1$
	\end{itemize}
\end{Lemma}
\begin{proof}
	Übungsaufgabe Aufgabe 8
\end{proof}
Konsequenz: Für $x \in \RR^m$ wähle $v=\frac{x}{\|x\|_2}+ sign(x_1)e_1
\Rightarrow Q_vx \in \text{span}\{e_1\}$\\
\newline
Nun Durchführung der QR-Zerlegung nach Householder $A=(a_1,\cdots,a_n); a_1,\cdots,a_n \in \RR^m$\\
\begin{itemize}
	\item[1.]QR-Schritt: Bestimme $v_1 \in\RR^m$ mit $Q_{v_1}\cdot a_1 = r_{11}\cdot e_1$ für ein $r_{11} \in \RR$\\\newline
	$\Rightarrow Q_{v_1}A=(Q_{v_1}a_1,\cdots,Q_{v_1}a_n) =\left(
	\begin{array}{c|ccc} r_{11} & * & \cdots & *\\ \hline
						 0 & & & \\
						 \vdots & &A^{(2)} &\\
						 0 & & &
	\end{array}\right)\\ Q=Q_{v_1}$
	\item[2.]QR-Schritt: $A^{(2)}= \left(a_1^{(2)},\cdots,a_{n-1}^{(2)}\right)$\\
			Bestimme $v_2^{(2)} \in \RR^{m-1}$ mit $Q_{v_2^{(2)}}a_1^{(2)} = r_{22}e_1\\
			\Rightarrow \text{für } v_2 = \begin{pmatrix}0\\v_2^{(2)}\end{pmatrix}\in \RR^m \text{ gilt } Q_{v_2}Q_{v_1}A=\left(
			\begin{array}{c|ccc} r_{11} & * & \cdots & *\\ \hline
			0 & & & \\
			\vdots & &Q_{v_2^{(2)}}A^{(2)} &\\
			0 & & &
			\end{array}\right) = \left(
			\begin{array}{ccccc} r_{11} & * & \cdots &\cdots & *\\
			0 & r_{22} & * & \cdots & *\\
			\vdots & 0 & &\\
			\vdots & \vdots& &A^{(3)}\\
			0 & 0 & &
			\end{array}\right)$
			$Q=Q\cdot Q_{v_2}$\\
\end{itemize}
Aufwand: Im k-ten Schritt $(k=1, \cdots,n)$
		\begin{itemize}
			\item[-]Bestimmen von $v_k: m-k+2$ Multiplikationen + 1 Wurzel
			\item[-]Berechnen von $Q_{v_k^{(k)}}A^k: (m-k+1)(n-k+1)$ Multiplikationen
			\item[-] Berechnen von $QQ_{v_k}: (m-k+1)n$ Multiplikationen\\
			\item[$\Rightarrow$]$\sum O(mn^2) \text{ Multiplikationen } + O(n) Wurzeln$
		\end{itemize}\hfill\\
		
\begin{Bemerkung}
	Vorherige Schrittfolge erfordert, dass $a_{11}^{(k)} \neq 0 \;\forall k=1,\cdots,n$.
	Dies kann mit Pivotisierung umgangen werden. 
\end{Bemerkung}\hfill
\begin{weitere Bemerkungen}\hfill
	\begin{itemize}
		\item[a)]QR-Zerlegung gehört zu \glqq stabilen Algorithmus\grqq, d.h. es gibt keine zusätzliche Fehlerverstärkung.
		\item[b)]$R^T$ ist \glqq Cholesky-artiger\grqq Faktor von $A^TA$ da $R^TR=A^TA$ (vgl. A)\\ i.A. $R^T \neq L$, jedoch $R^T=DL$ für $D=(sign(r_{11}),\cdots,sign(r_{nn}))$
		\item[c)]Ausblick: lineare Ausgleichsrechnung (später mehr)\\
		$A \in\RR^{m\times n}, rang(A)=n$\\
		Problem: Finde $\hat{x} \in \RR^n$ s.d. $\|A\hat{x}-b\|_2=\underset{x\in\RR^n}{min}\|Ax-b\|_2$\\
		Mit $A=QR$ und $\|Qy\|_2=\|y\|_2 \;\forall < \in \RR^n \text{ gilt }\\
      \|Ax-b\|_2^2=\|Q^T(Ax-b)\|_2^2=\left\|\begin{pmatrix}R_1\\0\end{pmatrix}x-\begin{pmatrix}c\\d\end{pmatrix}\right\|_2^2 = \|R_1x-c\|_2^2 + \|d\|^2$. Dieser Ausdruck ist minimal, wenn $R_1x=c$
	\end{itemize}
\end{weitere Bemerkungen}