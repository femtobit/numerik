\subsection{Lineare Ausgleichsprobleme, überbestimmte Gleichungssysteme}
Geg: $A \in \RR^{m\times n}, b\in \RR^n$\\
Oftmals:$rang(A,b)>rang(A) \rightsquigarrow \text{Lös}(A,b) = \emptyset$\\

Lineares Ausgleichsproblem:\\
Finde $\hat{x} \in \RR \text{ mit } \|A\hat{x}-b\|_2 = \underset{x \in \RR^n}{\mathrm{min}}\|Ax-b\|_2$ (LA)\\
\begin{Exkurs}\textbf{Approximation in Skalarprodukträumen}\\
\begin{Definition}
$L \subset \RR^n$ heißt \glqq affiner Unterraum\grqq wenn $L=v+U$, wobei $v \in \RR^n$, U Untervektorraum (UVR)
\end{Definition}
\begin{Beispiel}\hfill
	\begin{itemize}
	\item[a)] UVR ist affiner Unterraum
	\item[b)] Lös$(A,b)$ ist affiner Unterraum falls nicht linear
	\end{itemize}
\end{Beispiel}
\textbf{Problem:}\\
Geg: \begin{itemize}
	 \item[a)] affiner UR $L=v+U$
	 \item[b)] $b \in \RR^n$
	 \end{itemize}
Finde $\hat{y} \in L s.d. \|\hat{y}-b\|_2 = \underset{y \in L}{min} \|y-b\|_2$ (MIN)\\
\begin{Satz}
TODO
\end{Satz}
\end{Exkurs} 
