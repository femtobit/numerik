\subsection{Lineare Ausgleichsprobleme, überbestimmte Gleichungssysteme}
Geg: $A \in \RR^{m\times n}, b\in \RR^n$\\
Oftmals:$rang(A,b)>rang(A) \rightsquigarrow \text{Lös}(A,b) = \emptyset$\\

Lineares Ausgleichsproblem:\\
Finde $\hat{x} \in \RR \text{ mit } \|A\hat{x}-b\|_2 = \underset{x \in \RR^n}{\mathrm{min}}\|Ax-b\|_2$ (LA)
\subsubsection{Exkurs: Approximation in Skalarprodukträumen}
\begin{Definition}
$L \subset \RR^n$ heißt \glqq affiner Unterraum\grqq wenn $L=v+U$, wobei $v \in \RR^n$, U Untervektorraum (UVR)
\end{Definition}
\begin{Beispiel}\hfill
	\begin{itemize}
	\item[a)] UVR ist affiner Unterraum
	\item[b)] Lös$(A,b)$ ist affiner Unterraum falls nicht linear
	\end{itemize}
\end{Beispiel}
\textbf{Problem:}\\
Geg: \begin{itemize}
	 \item[a)] affiner UR $L=v+U$
	 \item[b)] $b \in \RR^n$
	 \end{itemize}
Finde $\hat{y} \in L s.d. \|\hat{y}-b\|_2 = \underset{y \in L}{min} \|y-b\|_2$ (MIN)\\
\begin{Satz}
Für einen affinen Unterraum $L=v+U\quad(v\in \RR^n, U\subset \RR^n UVR)$ und ein $b\in \RR^n$ gilt:\\
\begin{itemize}
\item[a)]$\hat{y} \in L$ erfüllt (MIN) $\Leftrightarrow \hat{y}-b\in\U^{\perp}$
\item[b)]$\hat{y}$ mit (MIN) ist eindeutig bestimmt
\end{itemize}
\end{Satz}
\begin{proof}\hfill
\begin{itemize}
\item[a)] Sei $y \in L \Rightarrow y=v+u, u\in U$\\Wegen $\RR^n=U \oplus U^{\perp} \text{ gilt } b-v=u_1+U_2, \text{ wobei } u_1\in U, u_2 \in U^{\perp}$\\
$\|b-y\|_2^{2} =\|(b-v)-(y-v)\|_2^{2}=\|u_1+u_2-u\|_2^{2}=\|-(u+u_1)+u_2\|_2^{2} \overset{Pythagoras}{=}\|u_1-u\|_2^{2} + \|u_2\|^{2}$\\
Also gilt: $\|b-y\|_2^{2}$ minimal $\Leftrightarrow u_1=u \Leftrightarrow b-y = u_2 \in U^{\perp}$\\
\item[b)] Sei $\hat{y}$ Lösung von (MIN). Sei $y_2 \in L \text{ mit }\|\hat{y}-b\|_2 =\|y_2-b_2\| \overset{a)}{\Rightarrow} \hat{y}-b,y_2-b \in U^{\perp} \Rightarrow \hat{y}-y_2 \in U^{\perp}$\\
Andererseits gilt $\hat{y}-y_2 \in U \Rightarrow \hat{y}-y_2=0$
\end{itemize}
\end{proof}
\begin{Satz}
$A \in \RR^{m \times n}$.Dann gilt:\\\begin{itemize}
\item[a)]$(ker A)^{\perp} = im A$ (Übungsaufgabe3)
\item[b)]$A^TA$ symmetrisch
\item[c)]$ker A^TA = ker A$
\item[d)]$im A^TA im A^T$
\item[e)]$A^TA \ge 0 \text{ mit } (A^TA>0 \Leftrightarrow ker A = {0})$
\item[f)] $\forall \lambda \in \RR\backslash\{0\}: dim ker(\lambda I-A^TA) = dim ker(\lambda I-AA^T)$
\end{itemize}
\end{Satz}
\begin{proof}\hfill
\begin{itemize}
\item[c)]$"\supset"$ klar\\
$" \subset" $ Sei $x \in ker A^TA \Rightarrow 0=x^TA^TAx=\|Ax\|_2^{2} \Rightarrow Ax=0 \Rightarrow \in ker A$
\item[d)]$im AA^T \overset{a)}{=} (ker(A^TA)^T)^{\perp} = (ker A^TA)^{\perp} \overset{c)}{=} (ker A)^{\perp} \overset{a)}{=} im A^T$
\item[e)]$A^TA\ge 0 klar$\\$" \Rightarrow"\text{ Sei } ker A^TA =\{0\} \overset{c)}{\Rightarrow}ker\{0\}$\\$"\Leftarrow" \text{ Sei } ker A=\{0\}, x\in \RR^n\backslash\{0\} \Rightarrow x^TA^TAx = \|Ax\|_2^{2} > 0$
\item[f)] folgt aus ÜA 3
\end{itemize}
\end{proof}
\subsubsection{Zurück zum linearen Ausgleichsproblem}
TODO