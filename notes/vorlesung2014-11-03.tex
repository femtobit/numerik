\subsection{Lineare Ausgleichsprobleme, überbestimmte Gleichungssysteme}
Geg: $A \in \RR^{m\times n}, b\in \RR^n$\\
Oftmals: $\rang(A,b)>\rang(A) \rightsquigarrow \text{Lös}(A,b) = \emptyset$\\

Lineares Ausgleichsproblem: Finde
\begin{align*}
\tag{LA}
\hat{x} \in \RR \text{ mit } \|A\hat{x}-b\|_2 = \underset{x \in \RR^n}{\mathrm{min}}\|Ax-b\|_2
\end{align*}

\subsubsection{Exkurs: Approximation in Skalarprodukträumen}
\begin{Definition}
$L \subset \RR^n$ heißt \glqq affiner Unterraum\grqq wenn $L=v+U$, wobei $v \in \RR^n$, U Untervektorraum (UVR)
\end{Definition}
\begin{Beispiel}\hfill
	\begin{itemize}
	\item[a)] UVR ist affiner Unterraum
	\item[b)] Lös$(A,b)$ ist affiner Unterraum falls nicht leer
	\end{itemize}
\end{Beispiel}
\textbf{Problem:}\\
Gegeben: \begin{itemize}
	 \item[a)] affiner UR $L=v+U$
	 \item[b)] $b \in \RR^n$
	 \end{itemize}
Finde
\begin{align*}
\tag{MIN}
\hat{y} \in L \text{ so dass } \|\hat{y}-b\|_2 = \underset{y \in L}{min} \|y-b\|_2
\end{align*}

\begin{Satz}
Für einen affinen Unterraum $L=v+U\quad(v\in \RR^n, U\subset \RR^n UVR)$ und ein $b\in \RR^n$ gilt:\\
\begin{itemize}
\item[a)]$\hat{y} \in L$ erfüllt (MIN) $\Leftrightarrow \hat{y}-b\in U^{\perp}$
\item[b)]$\hat{y}$ mit (MIN) ist eindeutig bestimmt
\end{itemize}
\end{Satz}
\begin{proof}\hfill
\begin{itemize}
\item[a)] Sei $y \in L \Rightarrow y=v+u, u\in U$\\Wegen $\RR^n=U \oplus U^{\perp} \text{ gilt } b-v=u_1+U_2, \text{ wobei } u_1\in U, u_2 \in U^{\perp}$\\
$\|b-y\|_2^{2} =\|(b-v)-(y-v)\|_2^{2}=\|u_1+u_2-u\|_2^{2}=\|-(u+u_1)+u_2\|_2^{2} \overset{Pythagoras}{=}\|u_1-u\|_2^{2} + \|u_2\|^{2}$\\
Also gilt: $\|b-y\|_2^{2}$ minimal $\Leftrightarrow u_1=u \Leftrightarrow b-y = u_2 \in U^{\perp}$\\
\item[b)] Sei $\hat{y}$ Lösung von (MIN). Sei $y_2 \in L \text{ mit }\|\hat{y}-b\|_2 =\|y_2-b_2\| \overset{a)}{\Rightarrow} \hat{y}-b,y_2-b \in U^{\perp} \Rightarrow \hat{y}-y_2 \in U^{\perp}$.
Andererseits gilt $\hat{y}-y_2 \in U \Rightarrow \hat{y}-y_2=0$
\end{itemize}
\end{proof}
\begin{Satz}
$A \in \RR^{m \times n}$. Dann gilt:\\\begin{itemize}
\item[a)]$(\ker A)^{\perp} = \im A$ \quad \text{(Übungsaufgabe3)}
\item[b)]$A^TA$\; \text{ symmetrisch }
\item[c)]$\ker A^TA = \ker A$
\item[d)]$\im A^TA = \im A^T$
\item[e)]$A^TA \ge 0 \text{ mit } (A^TA>0 \Leftrightarrow ker A = {0})$
\item[f)] $\forall \lambda \in \RR\backslash\{0\}: \dim \ker(\lambda I-A^TA) = \dim \ker(\lambda I-AA^T)$
\end{itemize}
\end{Satz}
\begin{proof}\hfill
\begin{itemize}
\item[c)]$"\supset"$ \; \text{klar}\\
$" \subset" $ Sei $x \in \ker A^TA \Rightarrow 0=x^TA^TAx=\|Ax\|_2^{2} \Rightarrow Ax=0 \Rightarrow \in \ker A$
\item[d)]$\im AA^T \overset{a)}{=} (\ker(A^TA)^T)^{\perp} = (ker A^TA)^{\perp} \overset{c)}{=} (\ker A)^{\perp} \overset{a)}{=} im A^T$
\item[e)]$A^TA\ge 0  \; \text{klar}$\\$" \Rightarrow"\text{ Sei } \ker A^TA =\{0\} \overset{c)}{\Rightarrow}ker\{0\}$\\$"\Leftarrow" \text{ Sei } \ker A=\{0\}, x\in \RR^n\backslash\{0\} \Rightarrow x^TA^TAx = \|Ax\|_2^{2} > 0$
\item[f)] folgt aus ÜA 3
\end{itemize}
\end{proof}
\subsubsection{Zurück zum linearen Ausgleichsproblem}
\begin{align*}
&A \in \RR^{m \times n}, b \in \RR^n\\
&\text{(LA): Finde } \hat{x} \in \RR^n s.d. \|A\hat{x}-b\|_2 = \underset{x \in \RR}{\mathrm{min}}\|Ax-b\|_2\\
&\Leftrightarrow \hat{y}-b \in (\im A)^{\perp} = \ker A^T\\
&\Leftrightarrow A^T(\hat{y}-b)=0 \Leftrightarrow A^T(Ax-b)=0\\
&\Leftrightarrow A^TAx=A^Tb \quad \text{(NGL:  "`Normalengleichung"')}
\end{align*}
\begin{Bemerkung}
$A^TAx=A^Tb$ ist immer lösbar, da $\im A^TA=\im A^T$
\end{Bemerkung}
\begin{Satz}
$A \in \RR^{m \times n}, b \in \RR^m.$ Dann gilt:
\begin{itemize}
\item[a)]$\|A\hat{x}-b\|_2= \underset{x \in \RR^n}{\mathrm{min}}\|Ax-b\|_2 \Leftrightarrow A^TAx=A^Tb$ (NGL)
\item[b)] $\mathrm{NGL\;ist\;loesbar\;} (\Loes(A^TA,A^Tb) = \emptyset$)
\item[c)] $\mathrm{Es\;gibt\;ein\;eindeutiges\;} \hat{x_{*}} \in \RR^n$ mit $ \|\hat{x_{*}}\|_2 = \underset{\hat{x} \text{ erfüllt NGL }}{\mathrm{min}}\|\hat{x}\|$
\end{itemize}
\end{Satz}
\begin{proof}
	\begin{itemize}
		\item[a) und b)] folgt aus vorigen Überlegungen
		\item[c)] Es gilt $L=\{\hat{x}\in \RR^n:A^TA\hat{x}=A^Tb\}$ ist affiner UR, Rest folgt dann aus 2.9.1
	\end{itemize}
\end{proof}
\begin{Bemerkung}
	(LA) wird auch häufig als Lösung "Lösung im kleinste-Quadrat-Sinne"
	\textit{(least squares solution)} bezeichnet.
\end{Bemerkung}
\begin{Beispiel}
	Gesucht ist eine Gerade $y=\alpha x+\beta$ durch (0,1),(1,1),(2,1),(5,1)\\
	$\rightsquigarrow$ gibts nicht\\
	Suche stattdessen eine Gerade, die die Summe der Quadrate der Abstände zu den Punkten minimiert.\\
	Minimiere \begin{align*}
	&\left\|\underbrace{\begin{pmatrix}
	0 & 1\\1 & 1\\2 & 1\\ 5  & 1
	\end{pmatrix}}_{=A}\underbrace{\begin{pmatrix}
	\alpha\\\beta
	\end{pmatrix}}_{=x}-\underbrace{\begin{pmatrix}
	2\\1\\1\\1
	\end{pmatrix}}_{=b}\right\|_2\\\\
	&\overset{NGL}{\Rightarrow} \begin{pmatrix}
	30 & 8\\8 & 4
	\end{pmatrix}\begin{pmatrix}
	\alpha\\\beta
	\end{pmatrix}=\begin{pmatrix}
	8\\5
	\end{pmatrix}\quad
	Lsg.\begin{pmatrix}
	\frac{-1}{7}\\\frac{43}{28}
	\end{pmatrix}=\begin{pmatrix}
	\alpha\\\beta
	\end{pmatrix}
	\end{align*}
\end{Beispiel}
\begin{Bemerkung}
	Wenn $\rang A = n$, dann kann (LA) ohne (NGL) mit QR-Zerlegung gelöst werden
\end{Bemerkung}

\subsubsection{Singulärwertzerlegung}
\begin{Satz}
	Sei $A \in \RR^{m \times n}, r=rang(A)$. Dann existieren $U \in O_m(\RR), V \in O_n(\RR) \text{ sowie } \sigma_1 \ge \ldots \ge \sigma_r>0$, sodass für $\sum=\diag(\sigma_1,\ldots,\sigma_r)$ gilt $U^*AV=\begin{pmatrix}
	\sum & 0_{r,(n-1)}\\
	0_{m-r,r} & 0_{m-r,n-r}
	\end{pmatrix}$\\\glqq Singulärwertzerlegung"\\
	Die Singulärwerte $\sigma_1,\ldots,\sigma_r$ sind durch $A$ eindeutig bestimmt.
\end{Satz}
Vorbemerkungen zum Beweis:\\
\begin{itemize}
	\item[(i)]$\sigma(A^TA)\backslash\{0\} = \sigma(AA^T)\backslash\{0\}\\ \text{dim ker}(\lambda I-A^TA)=\text{dim ker}(\lambda I-AA^T)$
	\item[(ii)]$\rang(A^TA) = \rang(AA^T)=r\\ \sigma(A^TA) \subset [0,\infty)$
\end{itemize}
\begin{proof}
	Seien $\lambda_1 \ge \ldots \ge \lambda_r\ge\lambda_{r+1}=\ldots=\lambda_n=0$ die Eigenwerte von $A^TA$ (inkl. Vielfache)\\
	$v_1,\ldots,v_n$ sei ONS im $R^n$ mit $A^TAv_i=\lambda_iv_i \quad i=1,\ldots,n$\\
	Setze $\sigma_i =\sqrt{\lambda_i} \quad i=1,\ldots,r \text{ und } u_i=\frac{1}{\sigma_i}Av_i$\\
	\begin{itemize}
		\item[a)]Für $k=1,\ldots,r$ gilt $AA^Tu_i=\frac{1}{\sigma_i}A^TAv_i=\frac{1}{\sigma_i}A(\lambda_iv_i)=\lambda_i\left(\frac{1}{\sigma_i}Av_i\right)=\lambda_iu_i$
		\item[b)]Für $j,k=1,\ldots,r$ gilt $u_j^{T}u_k=\frac{1}{\sigma_j\sigma_k}v_j^{T}A^TAv_k=\frac{\lambda_k}{\sigma_j\sigma_k}v_j^{T}v_k=\delta_{j,k}$
	\end{itemize}
	$\Rightarrow \{u_1,\ldots,u_r\}$ ist ein ONS in $\RR^m$ bestehend aus EV von $AA^T$.\\\newline
	Ergänze zu ONB $\{u_1,\ldots,u_m\}$. Dann gilt $A^Tu_{r+1}=\ldots=A^Tu_m=0$.\\
	Betrachte $U=(u_1,\ldots,u_m) \in O_m(\RR), V=(v_1,\ldots,v_n) \in O_n(\RR)$\\
	Betrachte $U^TAv$: Es gilt $(U^TAv)_{i,j}=u_i^{T}Av_j$\\\\
	Für $i=r+1,\ldots,m$ gilt $u_i^{T}Av_j=\underset{=0}{(Au_i)^T}v_j=0$\\
	Für $i=1,\ldots,r $ gilt $u_i^{T}Av_j=\frac{1}{\sigma_i}(Av_i)^TAv_j=\frac{1}{\sigma_i}v_i^{T}A^TAv_j=\sigma_i\delta_{ij}$\\\newline
	Also $U^TAV= \left(\begin{array}{c|c}
		\begin{matrix}\sigma_1 & &  \\&\ddots & \\ & & \sigma_r\end{matrix} &\begin{matrix}& &\\ &\bigzero& \\&&\end{matrix}\\\hline
		\bigzero &\begin{matrix}& &\\ &\bigzero& \\&&\end{matrix}
		\end{array}\right)$
\end{proof}
