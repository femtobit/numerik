Es gibt also $U \in \U_m(\RR), V \in \U_n(\RR)$ mit
\begin{align*}
  & A = U \left(\begin{array}{c|c} \varSigma & 0  \\ \hline 0 & 0 \end{array} \right) V^T \; \text{ mit } \; \varSigma = \diag(\sigma_1, \cdots , \sigma_r) \text{ und }\\
  & r = \rang(A) \; ; \quad \sigma_1\ge \sigma_2\ge\sigma_1\ge  \cdots , \sigma_r > 0
\end{align*}
Noch zu zeigen: Die Singulärwerte sind eindeutig bestimmt durch $A$.
(Achtung! $U$ und $V$ sind nicht eindeutig., Das gitl nurr wenn die $\sigma_i$
paarweise verscheiden sind.)

Beweis: Sei
\begin{align*}
  & U_1 \left(\begin{array}{c|c} \varSigma_1 & 0  \\ \hline 0 & 0 \end{array} \right) V_1^T =
A = U_2 \left(\begin{array}{c|c} \varSigma_2 & 0  \\ \hline 0 & 0 \end{array} \right) V_2^T \\
& \Rightarrow A^T A = V_1^T \left(\begin{array}{c|c} \varSigma_1 & 0  \\ \hline 0 & 0 \end{array} \right) V_1 = V_2^T \left(\begin{array}{c|c} \varSigma_2 & 0  \\ \hline 0 & 0 \end{array} \right) V_2 \\
& \Rightarrow \; \text{Die Quadrate der Singulärwerte sind Eigenwerte von } A^T A\\
& \Rightarrow \; \varSigma_1^2 = \varSigma_1^2  \quad \text{Die Eigenwerte sind der Größe nach sortiert}\\
& \Rightarrow \;  \varSigma_1 = \varSigma_1 \quad \text{Da A positiv definit ist}
\end{align*}
Die Singulärwertzerlegung (SVD = single value decomposition) findet ihre Anwendung im
linesaren Ausgleichsproblem (LA)


Hier mache ich weiter: Michael

\subsection{Pseudoinverse (Moore-Penrose-Inverse)}

TODO

\subsection{Iterations-Verfahren}


TODO
