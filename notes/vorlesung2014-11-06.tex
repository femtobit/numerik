Es gibt also $U \in \U_m(\RR), V \in \U_n(\RR)$ mit
\begin{align*}
  & A = U \left(\begin{array}{c|c} \varSigma & 0  \\ \hline 0 & 0 \end{array} \right) V^T \; \text{ mit } \; \varSigma = \diag(\sigma_1, \cdots , \sigma_r) \text{ und }\\
  & r = \rang(A) \; ; \quad \sigma_1\ge \sigma_2\ge\sigma_1\ge  \cdots , \sigma_r > 0
\end{align*}
Noch zu zeigen: Die Singulärwerte sind eindeutig bestimmt durch $A$.
(Achtung! $U$ und $V$ sind nicht eindeutig., Das gitl nurr wenn die $\sigma_i$
paarweise verscheiden sind.)

Beweis: Sei
\begin{align*}
  & U_1 \left(\begin{array}{c|c} \varSigma_1 & 0  \\ \hline 0 & 0 \end{array} \right) V_1^T =
A = U_2 \left(\begin{array}{c|c} \varSigma_2 & 0  \\ \hline 0 & 0 \end{array} \right) V_2^T \\
& \Rightarrow A^T A = V_1^T \left(\begin{array}{c|c} \varSigma_1^2 & 0  \\ \hline 0 & 0 \end{array} \right) V_1 = V_2^T \left(\begin{array}{c|c} \varSigma_2^2 & 0  \\ \hline 0 & 0 \end{array} \right) V_2 \\
& \Rightarrow \; \text{Die Quadrate der Singulärwerte sind Eigenwerte von } A^T A\\
& \Rightarrow \; \varSigma_1^2 = \varSigma_1^2  \quad \text{Die Eigenwerte sind der Größe nach sortiert}\\
& \Rightarrow \;  \varSigma_1 = \varSigma_1 \quad \text{weil A positiv definit ist}
\end{align*}
Die Singulärwertzerlegung (SVD = single value decomposition) findet ihre Anwendung im
linearen Ausgleichsproblem (LA): Gesucht ist ein $\hat{x}\in \RR^n$ so dass
\begin{align*}
\tag{LA}
\|A\hat{x}-b\|_2 & = \min_{x \in \RR^n}\|Ax-b\|_2
\intertext{Mit geeigneten Abbildungen $U, V \in \U^n$ gilt dann, da $U, V$ linientreu abbilden:}
\|A\hat{x}-b\|_2^2 & = \| U^T A x - U^T b\|_2^2 =  \| U^T A V (V^T x) - U^T b\|_2^2
= \left\| \begin{pmatrix}\varSigma & 0 \\ 0 & 0 \end{pmatrix} V^T x - U^T b \right\|_2^2\\
& = \sum_{i = 1}^r \left(\sigma_i(V^T x)_i - (U^T b_i)\right)^2 + \sum_{i = r +1}^n \left(U^T b_i\right)^2\\
&= \sum_{i = 1}^r \left(\sigma_i v_i^T x_i - U^T b \right)^2 +  \sum_{i = r +1}^n \left(U^T b_i\right)^2
\end{align*}
Der Ausruck wird minimal wenn der erste Summand verschwindet. Folgerungen:
\begin{itemize}
  \item[a)]
  \begin{align*}
    & \hat{x} \text{ löst das lineare Ausgleichsproblem}\\
    & \Leftrightarrow \quad v_i^T \hat{x} = \frac{1}{\sigma_i} U^T b \quad i = 1, \cdots , r\\
    & \Leftrightarrow \quad V^T \hat{x} =
    \begin{pmatrix}
    \frac{1}{\sigma_1} u_1^T b\\ \vdots \\ \frac{1}{\sigma_r} u_r^T b \\
    \alpha_{r + 1} \\ \vdots \\ \alpha_n
    \end{pmatrix} \quad \text{für $\alpha_{r + 1}, \cdots , \alpha_n \in \RR$ beliebig}\\
    & \Leftrightarrow \quad \hat{x} = V \cdot
    \begin{pmatrix}
    \frac{1}{\sigma_1} u_1^T b\\ \vdots \\ \frac{1}{\sigma_r} u_r^T b \\
    \alpha_{r + 1} \\ \vdots \\ \alpha_n
    \end{pmatrix}
  \end{align*}
  \item[b)]
  \begin{align*}
    & \|\hat{x}\|_2^2 = \sum_{i = 1}^r \left(\frac{1}{\sigma_i} u_i^T b \right)^2 + \sum_{i = r + 1}^n \alpha_i^2
    \intertext{Daraus folgt: Die eindeutig bestimmte Lösung mit minimaler Norm ist}
    &\hat{x} = V \cdot
    \begin{pmatrix}
    \frac{1}{\sigma_1} u_1^T b\\ \vdots \\ \frac{1}{\sigma_r} u_r^T b \\
    0 \\ \vdots \\ 0
    \end{pmatrix} = V  \begin{pmatrix}\varSigma^{-1} & 0 \\ 0 & 0 \end{pmatrix} U^T b = A^+ b
  \end{align*}
\end{itemize}
Dabei ist $A^+$ die Pseudoinverse von $A$, siehe nächster Abschnitt.


\subsection{Pseudoinverse (Moore-Penrose-Inverse)}

Hier mache ich weiter: Michael

\subsection{Iterations-Verfahren}


TODO
