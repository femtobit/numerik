\begin{Satz}\label{satz:CG-Orthogonal}
Sei $A \in \RR^{n\times n}$, $A > 0$, $b \in \RR^n.$

Dann gilt beim CG-Verfahren: Ist $r^{(\ell)} \ne 0$, dann ist
$\{d^{(0)}, \ldots, d^{(\ell)}\}$ $A$-orthogonal, d.h. 
    $$d^{(i)\,\T} A d^{(j)} = 0\text{ für alle }i \ne j$$
und $d^{(0)}, \ldots, d^{(\ell)} \in \RR^n \setminus \{0\}.$
\end{Satz}
Beweis in STiNE.

\begin{Korollar}
Nach spätestens $n$ Schritten ist $r^{(\ell)} = 0.$
\end{Korollar}
\begin{proof}
Angenommen, $r^{(0)}, \ldots, r^{(n)} \ne 0.$ Dann gilt nach Satz \ref{satz:CG-Orthogonal}:
$\mathcal D := \{d^{(0)}, \ldots, d^{(n)}\}$ ist $A$-orthogonales System mit
$d^{(0)}, \ldots, d^{(n)} \in \RR^n \setminus \{0\}.$

Daraus folgt: $\mathcal D$ ist ein linear unabhängiges System der Größe $n+1$ im
$\RR^n$. Widerspruch.
\end{proof}

\begin{Bemerkungen}
\quad
\begin{itemize}
  \item[a)] Das Residuum ist häufig schon nach wenigen Schritten klein. Das
    Verfahren kann vorzeitig abgebrochen werden.
  \item[b)] (ohne Beweis) Sei $x^0 = 0$. Dann ist $x^k$ die Bestapproximation
    von $x = A^{-1}b$ im Raum
      $$\{b, Ab, A^2 b, \ldots, A^{k-1} b\} \quad\text{(Krylov-Raum)}$$
    bezüglich der Norm $\|\cdot\|_A$.
    Damit wird die Normdifferenz in jedem Schritt kleiner, genauer:
    $(r^k)$ fällt monoton.
  \item[c)] Der Aufwand in jedem Schritt beträgt $O(n^2)$ Multiplikationen.
\end{itemize}
\end{Bemerkungen}

[TODO: Beispiel]

\section{Nichtlineare Gleichungen}
\subsection{Einleitung}
Gegeben: $F\colon \RR^n \supset U \to \RR^n$. \\
Ziel: Finde $x_* \in U$ mit $F(x_*) = 0$. \\
Methode: Iterationsverfahren $x^{(k+1)} = \phi(x^{(k)})$.
\begin{Bemerkungen}[Erinnerungen]
\quad
\begin{itemize}
  \item[a)] $(x^{(k)})$ konvergiert mit Ordnung $p > 0$ gegen $x$,
    gdw. es ein $L > 0$ gibt, so dass $$\|x^{(k+1)} - x\| \le L\|x^{(k)} - x\|^p.$$
  \item[b)] $U \in \RR^n$ offen, $F: U \to \RR^n$ partiell differenzierbar.
    $$F'(x) := \left( \frac{\partial F_i}{\partial x_j}(x)\right) \quad \text{(Jacobi-Matrix)}$$
    $f\colon U \to \RR$ partiell differenzierbar:
      $$\nabla f(x) = (f'(x))^\T \quad \text{(Gradient)}.$$
\end{itemize}
\end{Bemerkungen}
TODO
